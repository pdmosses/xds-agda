% easychair.tex,v 3.5 2017/03/15

%\documentclass{easychair}
%\documentclass[EPiC]{easychair}
%\documentclass[EPiCempty]{easychair}
%\documentclass[debug]{easychair}
%\documentclass[verbose]{easychair}
%\documentclass[notimes]{easychair}
%\documentclass[withtimes]{easychair}
\documentclass[a4paper]{easychair}
%\documentclass[letterpaper]{easychair}

%\usepackage{doc}

% use this if you have a long article and want to create an index
% \usepackage{makeidx}

% In order to save space or manage large tables or figures in a
% landcape-like text, you can use the rotating and pdflscape
% packages. Uncomment the desired from the below.
%
% \usepackage{rotating}
% \usepackage{pdflscape}

\usepackage[T1]{fontenc}
\usepackage{microtype}
\DisableLigatures[-]{encoding = T1, family = tt* }

\usepackage{multicol}

\usepackage{doi}

\input{agda-macros}
\AgdaNoSpaceAroundCode{}

%\makeindex

%% Front Matter
%%
% Regular title as in the article class.
%
\title{Denotational Semantics of PCF in Agda \\[2ex]
\normalsize DRAFT (\today)}

% Authors are joined by \and. Their affiliations are given by \inst, which indexes
% into the list defined using \institute
%
\author{
Peter D. Mosses%\inst{1}\inst{2}
}

% Institutes for affiliations are also joined by \and,
\institute{
  Delft University of Technology, The Netherlands
  \\
  \email{p.d.mosses@tudelft.nl}
\\
   Swansea University, United Kingdom
%   \\
%   \email{p.d.mosses@swansea.ac.uk}
 }

%  \authorrunning{} has to be set for the shorter version of the authors' names;
% otherwise a warning will be rendered in the running heads. When processed by
% EasyChair, this command is mandatory: a document without \authorrunning
% will be rejected by EasyChair

\authorrunning{Peter Mosses}

% \titlerunning{} has to be set to either the main title or its shorter
% version for the running heads. When processed by
% EasyChair, this command is mandatory: a document without \titlerunning
% will be rejected by EasyChair

\titlerunning{Denotational Semantics of PCF}

\begin{document}

\maketitle

\begin{abstract}
In synthetic domain theory, all sets are predomains, domains are pointed sets, and functions are implicitly continuous.
The denotational semantics of PCF (Plotkin's version) presented here illustrates how it might look if synthetic domain theory can be implemented in Agda.
Currently, the code uses unsatisfiable postulates as a work-around, to allow Agda to type-check the definitions.

The Agda source code used to generate this document is currently available only in a private repository,
but will soon be made public.
\end{abstract}


\bigskip\hrule\bigskip


\begin{code}%
\>[0]\AgdaSymbol{\{-\#}\AgdaSpace{}%
\AgdaKeyword{OPTIONS}\AgdaSpace{}%
\AgdaPragma{--rewriting}\AgdaSpace{}%
\AgdaPragma{--confluence-check}\AgdaSpace{}%
\AgdaSymbol{\#-\}}\<%
\\
%
\\[\AgdaEmptyExtraSkip]%
\>[0]\AgdaKeyword{module}\AgdaSpace{}%
\AgdaModule{ULC.All}\AgdaSpace{}%
\AgdaKeyword{where}\<%
\\
%
\\[\AgdaEmptyExtraSkip]%
\>[0]\AgdaKeyword{import}\AgdaSpace{}%
\AgdaModule{ULC.Variables}\<%
\\
\>[0]\AgdaKeyword{import}\AgdaSpace{}%
\AgdaModule{ULC.Terms}\<%
\\
\>[0]\AgdaKeyword{import}\AgdaSpace{}%
\AgdaModule{ULC.Domains}\<%
\\
\>[0]\AgdaKeyword{import}\AgdaSpace{}%
\AgdaModule{ULC.Environments}\<%
\\
\>[0]\AgdaKeyword{import}\AgdaSpace{}%
\AgdaModule{ULC.Semantics}\<%
\\
\>[0]\AgdaKeyword{import}\AgdaSpace{}%
\AgdaModule{ULC.Checks}\<%
\end{code}
\clearpage
\begin{code}%
\>[0]\AgdaKeyword{module}\AgdaSpace{}%
\AgdaModule{Scheme.Domain-Notation}\AgdaSpace{}%
\AgdaKeyword{where}\<%
\\
%
\\[\AgdaEmptyExtraSkip]%
\>[0]\AgdaKeyword{open}\AgdaSpace{}%
\AgdaKeyword{import}\AgdaSpace{}%
\AgdaModule{Relation.Binary.PropositionalEquality.Core}\<%
\\
\>[0][@{}l@{\AgdaIndent{0}}]%
\>[2]\AgdaKeyword{using}\AgdaSpace{}%
\AgdaSymbol{(}\AgdaOperator{\AgdaDatatype{\AgdaUnderscore{}≡\AgdaUnderscore{}}}\AgdaSymbol{;}\AgdaSpace{}%
\AgdaInductiveConstructor{refl}\AgdaSymbol{)}\AgdaSpace{}%
\AgdaKeyword{public}\<%
\\
%
\\[\AgdaEmptyExtraSkip]%
\>[0]\AgdaComment{------------------------------------------------------------------------}\<%
\\
\>[0]\AgdaComment{--\ Agda\ requires\ Predomain\ and\ Domain\ to\ be\ sorts}\<%
\\
%
\\[\AgdaEmptyExtraSkip]%
\>[0]\AgdaFunction{Predomain}%
\>[11]\AgdaSymbol{=}\AgdaSpace{}%
\AgdaPrimitive{Set}\<%
\\
\>[0]\AgdaFunction{Domain}%
\>[11]\AgdaSymbol{=}\AgdaSpace{}%
\AgdaPrimitive{Set}\<%
\\
\>[0]\AgdaKeyword{variable}\<%
\\
\>[0][@{}l@{\AgdaIndent{0}}]%
\>[2]\AgdaGeneralizable{P}\AgdaSpace{}%
\AgdaGeneralizable{Q}%
\>[7]\AgdaSymbol{:}\AgdaSpace{}%
\AgdaFunction{Predomain}\<%
\\
%
\>[2]\AgdaGeneralizable{D}\AgdaSpace{}%
\AgdaGeneralizable{E}%
\>[7]\AgdaSymbol{:}\AgdaSpace{}%
\AgdaFunction{Domain}\<%
\\
%
\\[\AgdaEmptyExtraSkip]%
\>[0]\AgdaComment{--\ Domains\ are\ pointed}\<%
\\
\>[0]\AgdaKeyword{postulate}\<%
\\
\>[0][@{}l@{\AgdaIndent{0}}]%
\>[2]\AgdaPostulate{⊥}%
\>[12]\AgdaSymbol{:}\AgdaSpace{}%
\AgdaSymbol{\{}\AgdaBound{D}\AgdaSpace{}%
\AgdaSymbol{:}\AgdaSpace{}%
\AgdaFunction{Domain}\AgdaSymbol{\}}\AgdaSpace{}%
\AgdaSymbol{→}\AgdaSpace{}%
\AgdaBound{D}\<%
\\
%
\>[2]\AgdaPostulate{strict}%
\>[12]\AgdaSymbol{:}\AgdaSpace{}%
\AgdaSymbol{\{}\AgdaBound{D}\AgdaSpace{}%
\AgdaBound{E}\AgdaSpace{}%
\AgdaSymbol{:}\AgdaSpace{}%
\AgdaFunction{Domain}\AgdaSymbol{\}}\AgdaSpace{}%
\AgdaSymbol{→}\AgdaSpace{}%
\AgdaSymbol{(}\AgdaBound{D}\AgdaSpace{}%
\AgdaSymbol{→}\AgdaSpace{}%
\AgdaBound{E}\AgdaSymbol{)}\AgdaSpace{}%
\AgdaSymbol{→}\AgdaSpace{}%
\AgdaSymbol{(}\AgdaBound{D}\AgdaSpace{}%
\AgdaSymbol{→}\AgdaSpace{}%
\AgdaBound{E}\AgdaSymbol{)}\<%
\\
%
\\[\AgdaEmptyExtraSkip]%
%
\>[2]\AgdaComment{--\ Properties}\<%
\\
%
\>[2]\AgdaPostulate{strict-⊥}%
\>[12]\AgdaSymbol{:}\AgdaSpace{}%
\AgdaSymbol{∀}%
\>[31I]\AgdaSymbol{\{}\AgdaBound{D}\AgdaSpace{}%
\AgdaBound{E}\AgdaSymbol{\}}\AgdaSpace{}%
\AgdaSymbol{→}\AgdaSpace{}%
\AgdaSymbol{(}\AgdaBound{f}\AgdaSpace{}%
\AgdaSymbol{:}\AgdaSpace{}%
\AgdaBound{D}\AgdaSpace{}%
\AgdaSymbol{→}\AgdaSpace{}%
\AgdaBound{E}\AgdaSymbol{)}\AgdaSpace{}%
\AgdaSymbol{→}\<%
\\
\>[.][@{}l@{}]\<[31I]%
\>[16]\AgdaPostulate{strict}\AgdaSpace{}%
\AgdaBound{f}\AgdaSpace{}%
\AgdaPostulate{⊥}\AgdaSpace{}%
\AgdaOperator{\AgdaDatatype{≡}}\AgdaSpace{}%
\AgdaPostulate{⊥}\<%
\\
%
\\[\AgdaEmptyExtraSkip]%
\>[0]\AgdaComment{------------------------------------------------------------------------}\<%
\\
\>[0]\AgdaComment{--\ Fixed\ points\ of\ endofunctions\ on\ function\ domains}\<%
\\
%
\\[\AgdaEmptyExtraSkip]%
\>[0]\AgdaKeyword{postulate}\<%
\\
\>[0][@{}l@{\AgdaIndent{0}}]%
\>[2]\AgdaPostulate{fix}%
\>[12]\AgdaSymbol{:}\AgdaSpace{}%
\AgdaSymbol{∀}\AgdaSpace{}%
\AgdaSymbol{\{}\AgdaBound{D}\AgdaSpace{}%
\AgdaSymbol{:}\AgdaSpace{}%
\AgdaFunction{Domain}\AgdaSymbol{\}}\AgdaSpace{}%
\AgdaSymbol{→}\AgdaSpace{}%
\AgdaSymbol{(}\AgdaBound{D}\AgdaSpace{}%
\AgdaSymbol{→}\AgdaSpace{}%
\AgdaBound{D}\AgdaSymbol{)}\AgdaSpace{}%
\AgdaSymbol{→}\AgdaSpace{}%
\AgdaBound{D}\<%
\\
%
\\[\AgdaEmptyExtraSkip]%
%
\>[2]\AgdaComment{--\ Properties}\<%
\\
%
\>[2]\AgdaPostulate{fix-fix}%
\>[12]\AgdaSymbol{:}%
\>[54I]\AgdaSymbol{∀}\AgdaSpace{}%
\AgdaSymbol{\{}\AgdaBound{D}\AgdaSymbol{\}}\AgdaSpace{}%
\AgdaSymbol{(}\AgdaBound{f}\AgdaSpace{}%
\AgdaSymbol{:}\AgdaSpace{}%
\AgdaBound{D}\AgdaSpace{}%
\AgdaSymbol{→}\AgdaSpace{}%
\AgdaBound{D}\AgdaSymbol{)}\AgdaSpace{}%
\AgdaSymbol{→}\<%
\\
\>[54I][@{}l@{\AgdaIndent{0}}]%
\>[15]\AgdaPostulate{fix}\AgdaSpace{}%
\AgdaBound{f}\AgdaSpace{}%
\AgdaOperator{\AgdaDatatype{≡}}\AgdaSpace{}%
\AgdaBound{f}\AgdaSpace{}%
\AgdaSymbol{(}\AgdaPostulate{fix}\AgdaSpace{}%
\AgdaBound{f}\AgdaSymbol{)}\<%
\\
%
\>[2]\AgdaPostulate{fix-app}%
\>[12]\AgdaSymbol{:}%
\>[67I]\AgdaSymbol{∀}\AgdaSpace{}%
\AgdaSymbol{\{}\AgdaBound{P}\AgdaSpace{}%
\AgdaBound{D}\AgdaSymbol{\}}\AgdaSpace{}%
\AgdaSymbol{(}\AgdaBound{f}\AgdaSpace{}%
\AgdaSymbol{:}\AgdaSpace{}%
\AgdaSymbol{(}\AgdaBound{P}\AgdaSpace{}%
\AgdaSymbol{→}\AgdaSpace{}%
\AgdaBound{D}\AgdaSymbol{)}\AgdaSpace{}%
\AgdaSymbol{→}\AgdaSpace{}%
\AgdaSymbol{(}\AgdaBound{P}\AgdaSpace{}%
\AgdaSymbol{→}\AgdaSpace{}%
\AgdaBound{D}\AgdaSymbol{))}\AgdaSpace{}%
\AgdaSymbol{(}\AgdaBound{p}\AgdaSpace{}%
\AgdaSymbol{:}\AgdaSpace{}%
\AgdaBound{P}\AgdaSymbol{)}\AgdaSpace{}%
\AgdaSymbol{→}\<%
\\
\>[67I][@{}l@{\AgdaIndent{0}}]%
\>[15]\AgdaPostulate{fix}\AgdaSpace{}%
\AgdaBound{f}\AgdaSpace{}%
\AgdaBound{p}\AgdaSpace{}%
\AgdaOperator{\AgdaDatatype{≡}}\AgdaSpace{}%
\AgdaBound{f}\AgdaSpace{}%
\AgdaSymbol{(}\AgdaPostulate{fix}\AgdaSpace{}%
\AgdaBound{f}\AgdaSymbol{)}\AgdaSpace{}%
\AgdaBound{p}\<%
\\
%
\\[\AgdaEmptyExtraSkip]%
\>[0]\AgdaComment{------------------------------------------------------------------------}\<%
\\
\>[0]\AgdaComment{--\ Lifted\ domains}\<%
\\
%
\\[\AgdaEmptyExtraSkip]%
\>[0]\AgdaKeyword{postulate}\<%
\\
\>[0][@{}l@{\AgdaIndent{0}}]%
\>[2]\AgdaPostulate{𝕃}%
\>[12]\AgdaSymbol{:}\AgdaSpace{}%
\AgdaFunction{Predomain}\AgdaSpace{}%
\AgdaSymbol{→}\AgdaSpace{}%
\AgdaFunction{Domain}\<%
\\
%
\>[2]\AgdaPostulate{η}%
\>[12]\AgdaSymbol{:}\AgdaSpace{}%
\AgdaSymbol{∀}\AgdaSpace{}%
\AgdaSymbol{\{}\AgdaBound{P}\AgdaSymbol{\}}\AgdaSpace{}%
\AgdaSymbol{→}\AgdaSpace{}%
\AgdaBound{P}\AgdaSpace{}%
\AgdaSymbol{→}\AgdaSpace{}%
\AgdaPostulate{𝕃}\AgdaSpace{}%
\AgdaBound{P}\<%
\\
%
\>[2]\AgdaOperator{\AgdaPostulate{\AgdaUnderscore{}♯}}%
\>[12]\AgdaSymbol{:}\AgdaSpace{}%
\AgdaSymbol{∀}\AgdaSpace{}%
\AgdaSymbol{\{}\AgdaBound{P}\AgdaSymbol{\}}\AgdaSpace{}%
\AgdaSymbol{\{}\AgdaBound{D}\AgdaSpace{}%
\AgdaSymbol{:}\AgdaSpace{}%
\AgdaFunction{Domain}\AgdaSymbol{\}}\AgdaSpace{}%
\AgdaSymbol{→}\AgdaSpace{}%
\AgdaSymbol{(}\AgdaBound{P}\AgdaSpace{}%
\AgdaSymbol{→}\AgdaSpace{}%
\AgdaBound{D}\AgdaSymbol{)}\AgdaSpace{}%
\AgdaSymbol{→}\AgdaSpace{}%
\AgdaSymbol{(}\AgdaPostulate{𝕃}\AgdaSpace{}%
\AgdaBound{P}\AgdaSpace{}%
\AgdaSymbol{→}\AgdaSpace{}%
\AgdaBound{D}\AgdaSymbol{)}\<%
\\
%
\\[\AgdaEmptyExtraSkip]%
%
\>[2]\AgdaComment{--\ Properties}\<%
\\
%
\>[2]\AgdaPostulate{elim-♯-η}%
\>[12]\AgdaSymbol{:}\AgdaSpace{}%
\AgdaSymbol{∀}%
\>[115I]\AgdaSymbol{\{}\AgdaBound{P}\AgdaSpace{}%
\AgdaBound{D}\AgdaSymbol{\}}\AgdaSpace{}%
\AgdaSymbol{(}\AgdaBound{f}\AgdaSpace{}%
\AgdaSymbol{:}\AgdaSpace{}%
\AgdaBound{P}\AgdaSpace{}%
\AgdaSymbol{→}\AgdaSpace{}%
\AgdaBound{D}\AgdaSymbol{)}\AgdaSpace{}%
\AgdaSymbol{(}\AgdaBound{p}\AgdaSpace{}%
\AgdaSymbol{:}\AgdaSpace{}%
\AgdaBound{P}\AgdaSymbol{)}%
\>[43]\AgdaSymbol{→}\<%
\\
\>[.][@{}l@{}]\<[115I]%
\>[16]\AgdaSymbol{(}\AgdaBound{f}\AgdaSpace{}%
\AgdaOperator{\AgdaPostulate{♯}}\AgdaSymbol{)}\AgdaSpace{}%
\AgdaSymbol{(}\AgdaPostulate{η}\AgdaSpace{}%
\AgdaBound{p}\AgdaSymbol{)}\AgdaSpace{}%
\AgdaOperator{\AgdaDatatype{≡}}\AgdaSpace{}%
\AgdaBound{f}\AgdaSpace{}%
\AgdaBound{p}\<%
\\
%
\>[2]\AgdaPostulate{elim-♯-⊥}%
\>[12]\AgdaSymbol{:}\AgdaSpace{}%
\AgdaSymbol{∀}%
\>[132I]\AgdaSymbol{\{}\AgdaBound{P}\AgdaSpace{}%
\AgdaBound{D}\AgdaSymbol{\}}\AgdaSpace{}%
\AgdaSymbol{(}\AgdaBound{f}\AgdaSpace{}%
\AgdaSymbol{:}\AgdaSpace{}%
\AgdaBound{P}\AgdaSpace{}%
\AgdaSymbol{→}\AgdaSpace{}%
\AgdaBound{D}\AgdaSymbol{)}\AgdaSpace{}%
\AgdaSymbol{→}\<%
\\
\>[.][@{}l@{}]\<[132I]%
\>[16]\AgdaSymbol{(}\AgdaBound{f}\AgdaSpace{}%
\AgdaOperator{\AgdaPostulate{♯}}\AgdaSymbol{)}\AgdaSpace{}%
\AgdaPostulate{⊥}\AgdaSpace{}%
\AgdaOperator{\AgdaDatatype{≡}}\AgdaSpace{}%
\AgdaPostulate{⊥}\<%
\end{code}
\clearpage
\begin{code}%
\>[0]\AgdaComment{------------------------------------------------------------------------}\<%
\\
\>[0]\AgdaComment{--\ Flat\ domains}\<%
\\
%
\\[\AgdaEmptyExtraSkip]%
\>[0]\AgdaOperator{\AgdaFunction{\AgdaUnderscore{}+⊥}}%
\>[6]\AgdaSymbol{:}\AgdaSpace{}%
\AgdaPrimitive{Set}\AgdaSpace{}%
\AgdaSymbol{→}\AgdaSpace{}%
\AgdaFunction{Domain}\<%
\\
\>[0]\AgdaBound{S}\AgdaSpace{}%
\AgdaOperator{\AgdaFunction{+⊥}}%
\>[6]\AgdaSymbol{=}\AgdaSpace{}%
\AgdaPostulate{𝕃}\AgdaSpace{}%
\AgdaBound{S}\<%
\\
%
\\[\AgdaEmptyExtraSkip]%
\>[0]\AgdaComment{--\ Lifted\ operations\ on\ ℕ}\<%
\\
%
\\[\AgdaEmptyExtraSkip]%
\>[0]\AgdaKeyword{open}\AgdaSpace{}%
\AgdaKeyword{import}\AgdaSpace{}%
\AgdaModule{Agda.Builtin.Nat}\<%
\\
\>[0][@{}l@{\AgdaIndent{0}}]%
\>[2]\AgdaKeyword{using}\AgdaSpace{}%
\AgdaSymbol{(}\AgdaOperator{\AgdaPrimitive{\AgdaUnderscore{}==\AgdaUnderscore{}}}\AgdaSymbol{;}\AgdaSpace{}%
\AgdaOperator{\AgdaPrimitive{\AgdaUnderscore{}<\AgdaUnderscore{}}}\AgdaSymbol{)}\AgdaSpace{}%
\AgdaKeyword{public}\<%
\\
\>[0]\AgdaKeyword{open}\AgdaSpace{}%
\AgdaKeyword{import}\AgdaSpace{}%
\AgdaModule{Data.Nat.Base}\<%
\\
\>[0][@{}l@{\AgdaIndent{0}}]%
\>[2]\AgdaKeyword{using}\AgdaSpace{}%
\AgdaSymbol{(}\AgdaDatatype{ℕ}\AgdaSymbol{;}\AgdaSpace{}%
\AgdaInductiveConstructor{suc}\AgdaSymbol{;}\AgdaSpace{}%
\AgdaRecord{NonZero}\AgdaSymbol{;}\AgdaSpace{}%
\AgdaFunction{pred}\AgdaSymbol{)}\AgdaSpace{}%
\AgdaKeyword{public}\<%
\\
\>[0]\AgdaKeyword{open}\AgdaSpace{}%
\AgdaKeyword{import}\AgdaSpace{}%
\AgdaModule{Data.Bool.Base}\<%
\\
\>[0][@{}l@{\AgdaIndent{0}}]%
\>[2]\AgdaKeyword{using}\AgdaSpace{}%
\AgdaSymbol{(}\AgdaDatatype{Bool}\AgdaSymbol{)}\AgdaSpace{}%
\AgdaKeyword{public}\<%
\\
%
\\[\AgdaEmptyExtraSkip]%
\>[0]\AgdaComment{--\ ν\ ==⊥\ n\ :\ Bool\ +⊥}\<%
\\
%
\\[\AgdaEmptyExtraSkip]%
\>[0]\AgdaOperator{\AgdaFunction{\AgdaUnderscore{}==⊥\AgdaUnderscore{}}}\AgdaSpace{}%
\AgdaSymbol{:}\AgdaSpace{}%
\AgdaDatatype{ℕ}\AgdaSpace{}%
\AgdaOperator{\AgdaFunction{+⊥}}\AgdaSpace{}%
\AgdaSymbol{→}\AgdaSpace{}%
\AgdaDatatype{ℕ}\AgdaSpace{}%
\AgdaSymbol{→}\AgdaSpace{}%
\AgdaDatatype{Bool}\AgdaSpace{}%
\AgdaOperator{\AgdaFunction{+⊥}}\<%
\\
\>[0]\AgdaBound{ν}\AgdaSpace{}%
\AgdaOperator{\AgdaFunction{==⊥}}\AgdaSpace{}%
\AgdaBound{n}\AgdaSpace{}%
\AgdaSymbol{=}\AgdaSpace{}%
\AgdaSymbol{((λ}\AgdaSpace{}%
\AgdaBound{m}\AgdaSpace{}%
\AgdaSymbol{→}\AgdaSpace{}%
\AgdaPostulate{η}\AgdaSpace{}%
\AgdaSymbol{(}\AgdaBound{m}\AgdaSpace{}%
\AgdaOperator{\AgdaPrimitive{==}}\AgdaSpace{}%
\AgdaBound{n}\AgdaSymbol{))}\AgdaSpace{}%
\AgdaOperator{\AgdaPostulate{♯}}\AgdaSymbol{)}\AgdaSpace{}%
\AgdaBound{ν}\<%
\\
%
\\[\AgdaEmptyExtraSkip]%
\>[0]\AgdaComment{--\ ν\ >=⊥\ n\ :\ Bool\ +⊥}\<%
\\
%
\\[\AgdaEmptyExtraSkip]%
\>[0]\AgdaOperator{\AgdaFunction{\AgdaUnderscore{}>=⊥\AgdaUnderscore{}}}\AgdaSpace{}%
\AgdaSymbol{:}\AgdaSpace{}%
\AgdaDatatype{ℕ}\AgdaSpace{}%
\AgdaOperator{\AgdaFunction{+⊥}}\AgdaSpace{}%
\AgdaSymbol{→}\AgdaSpace{}%
\AgdaDatatype{ℕ}\AgdaSpace{}%
\AgdaSymbol{→}\AgdaSpace{}%
\AgdaDatatype{Bool}\AgdaSpace{}%
\AgdaOperator{\AgdaFunction{+⊥}}\<%
\\
\>[0]\AgdaBound{ν}\AgdaSpace{}%
\AgdaOperator{\AgdaFunction{>=⊥}}\AgdaSpace{}%
\AgdaBound{n}\AgdaSpace{}%
\AgdaSymbol{=}\AgdaSpace{}%
\AgdaSymbol{((λ}\AgdaSpace{}%
\AgdaBound{m}\AgdaSpace{}%
\AgdaSymbol{→}\AgdaSpace{}%
\AgdaPostulate{η}\AgdaSpace{}%
\AgdaSymbol{(}\AgdaBound{n}\AgdaSpace{}%
\AgdaOperator{\AgdaPrimitive{<}}\AgdaSpace{}%
\AgdaBound{m}\AgdaSymbol{))}\AgdaSpace{}%
\AgdaOperator{\AgdaPostulate{♯}}\AgdaSymbol{)}\AgdaSpace{}%
\AgdaBound{ν}\<%
\\
%
\\[\AgdaEmptyExtraSkip]%
\>[0]\AgdaComment{------------------------------------------------------------------------}\<%
\\
\>[0]\AgdaComment{--\ Products}\<%
\\
%
\\[\AgdaEmptyExtraSkip]%
\>[0]\AgdaComment{--\ Products\ of\ (pre)domains\ are\ Cartesian}\<%
\\
%
\\[\AgdaEmptyExtraSkip]%
\>[0]\AgdaKeyword{open}\AgdaSpace{}%
\AgdaKeyword{import}\AgdaSpace{}%
\AgdaModule{Data.Product.Base}\<%
\\
\>[0][@{}l@{\AgdaIndent{0}}]%
\>[2]\AgdaKeyword{using}\AgdaSpace{}%
\AgdaSymbol{(}\AgdaOperator{\AgdaFunction{\AgdaUnderscore{}×\AgdaUnderscore{}}}\AgdaSymbol{;}\AgdaSpace{}%
\AgdaOperator{\AgdaInductiveConstructor{\AgdaUnderscore{},\AgdaUnderscore{}}}\AgdaSymbol{)}\AgdaSpace{}%
\AgdaKeyword{renaming}\AgdaSpace{}%
\AgdaSymbol{(}\AgdaField{proj₁}\AgdaSpace{}%
\AgdaSymbol{to}\AgdaSpace{}%
\AgdaField{\AgdaUnderscore{}↓1}\AgdaSymbol{;}\AgdaSpace{}%
\AgdaField{proj₂}\AgdaSpace{}%
\AgdaSymbol{to}\AgdaSpace{}%
\AgdaField{\AgdaUnderscore{}↓2}\AgdaSymbol{)}\AgdaSpace{}%
\AgdaKeyword{public}\<%
\\
%
\\[\AgdaEmptyExtraSkip]%
\>[0]\AgdaComment{--\ (p₁\ ,\ ...\ ,\ pₙ)\ :\ P₁\ ×\ ...\ ×\ Pₙ\ \ (n\ ≥\ 2)}\<%
\\
\>[0]\AgdaComment{--\ \AgdaUnderscore{}↓1\ :\ P₁\ ×\ P₂\ →\ P₁}\<%
\\
\>[0]\AgdaComment{--\ \AgdaUnderscore{}↓2\ :\ P₁\ ×\ P₂\ →\ P₂}\<%
\\
%
\\[\AgdaEmptyExtraSkip]%
\>[0]\AgdaComment{------------------------------------------------------------------------}\<%
\\
\>[0]\AgdaComment{--\ Sum\ domains}\<%
\\
%
\\[\AgdaEmptyExtraSkip]%
\>[0]\AgdaComment{--\ Disjoint\ unions\ of\ (pre)domains\ are\ unpointed\ predomains}\<%
\\
\>[0]\AgdaComment{--\ Lifted\ disjoint\ unions\ of\ domains\ are\ separated\ sum\ domains}\<%
\\
%
\\[\AgdaEmptyExtraSkip]%
\>[0]\AgdaKeyword{open}\AgdaSpace{}%
\AgdaKeyword{import}\AgdaSpace{}%
\AgdaModule{Data.Sum.Base}\<%
\\
\>[0][@{}l@{\AgdaIndent{0}}]%
\>[2]\AgdaKeyword{using}\AgdaSpace{}%
\AgdaSymbol{(}\AgdaInductiveConstructor{inj₁}\AgdaSymbol{;}\AgdaSpace{}%
\AgdaInductiveConstructor{inj₂}\AgdaSymbol{)}\AgdaSpace{}%
\AgdaKeyword{renaming}\AgdaSpace{}%
\AgdaSymbol{(}\AgdaOperator{\AgdaDatatype{\AgdaUnderscore{}⊎\AgdaUnderscore{}}}\AgdaSpace{}%
\AgdaSymbol{to}\AgdaSpace{}%
\AgdaOperator{\AgdaDatatype{\AgdaUnderscore{}+\AgdaUnderscore{}}}\AgdaSymbol{;}\AgdaSpace{}%
\AgdaOperator{\AgdaFunction{[\AgdaUnderscore{},\AgdaUnderscore{}]′}}\AgdaSpace{}%
\AgdaSymbol{to}\AgdaSpace{}%
\AgdaOperator{\AgdaFunction{[\AgdaUnderscore{},\AgdaUnderscore{}]}}\AgdaSymbol{)}\AgdaSpace{}%
\AgdaKeyword{public}\<%
\\
%
\\[\AgdaEmptyExtraSkip]%
\>[0]\AgdaComment{--\ inj₁\ :\ P₁\ →\ P₁\ +\ P₂}\<%
\\
\>[0]\AgdaComment{--\ inj₂\ :\ P₂\ →\ P₁\ +\ P₂}\<%
\\
\>[0]\AgdaComment{--\ [\ f₁\ ,\ f₂\ ]\ :\ (P₁\ →\ P)\ →\ (P₂\ →\ P)\ →\ (P₁\ +\ P₂)\ →\ P}\<%
\end{code}
\clearpage
\begin{code}%
\>[0]\AgdaComment{------------------------------------------------------------------------}\<%
\\
\>[0]\AgdaComment{--\ Finite\ sequences}\<%
\\
%
\\[\AgdaEmptyExtraSkip]%
\>[0]\AgdaKeyword{open}\AgdaSpace{}%
\AgdaKeyword{import}\AgdaSpace{}%
\AgdaModule{Data.Vec.Recursive}\<%
\\
\>[0][@{}l@{\AgdaIndent{0}}]%
\>[2]\AgdaKeyword{using}\AgdaSpace{}%
\AgdaSymbol{(}\AgdaOperator{\AgdaFunction{\AgdaUnderscore{}\textasciicircum{}\AgdaUnderscore{}}}\AgdaSymbol{;}\AgdaSpace{}%
\AgdaInductiveConstructor{[]}\AgdaSymbol{)}\AgdaSpace{}%
\AgdaKeyword{public}\<%
\\
\>[0]\AgdaKeyword{open}\AgdaSpace{}%
\AgdaKeyword{import}\AgdaSpace{}%
\AgdaModule{Agda.Builtin.Sigma}\<%
\\
\>[0][@{}l@{\AgdaIndent{0}}]%
\>[2]\AgdaKeyword{using}\AgdaSpace{}%
\AgdaSymbol{(}\AgdaRecord{Σ}\AgdaSymbol{)}\<%
\\
%
\\[\AgdaEmptyExtraSkip]%
\>[0]\AgdaComment{--\ Sequence\ predomains}\<%
\\
\>[0]\AgdaComment{--\ P\ \textasciicircum{}\ n\ \ =\ P\ ×\ ...\ ×\ P\ \ (n\ ≥\ 0)}\<%
\\
\>[0]\AgdaComment{--\ P\ *\ \ \ \ =\ (P\ \textasciicircum{}\ 0)\ +\ ...\ +\ (P\ \textasciicircum{}\ n)\ +\ ...}\<%
\\
\>[0]\AgdaComment{--\ (n,\ p₁\ ,\ ...\ ,\ pₙ)\ :\ P\ *}\<%
\\
%
\\[\AgdaEmptyExtraSkip]%
\>[0]\AgdaOperator{\AgdaFunction{\AgdaUnderscore{}*}}%
\>[5]\AgdaSymbol{:}\AgdaSpace{}%
\AgdaFunction{Predomain}\AgdaSpace{}%
\AgdaSymbol{→}\AgdaSpace{}%
\AgdaFunction{Predomain}\<%
\\
\>[0]\AgdaBound{P}\AgdaSpace{}%
\AgdaOperator{\AgdaFunction{*}}%
\>[5]\AgdaSymbol{=}\AgdaSpace{}%
\AgdaRecord{Σ}\AgdaSpace{}%
\AgdaDatatype{ℕ}\AgdaSpace{}%
\AgdaSymbol{(}\AgdaBound{P}\AgdaSpace{}%
\AgdaOperator{\AgdaFunction{\textasciicircum{}\AgdaUnderscore{}}}\AgdaSymbol{)}\<%
\\
%
\\[\AgdaEmptyExtraSkip]%
\>[0]\AgdaComment{--\ \#′\ P\ *\ :\ ℕ}\<%
\\
%
\\[\AgdaEmptyExtraSkip]%
\>[0]\AgdaFunction{\#′}\AgdaSpace{}%
\AgdaSymbol{:}\AgdaSpace{}%
\AgdaSymbol{∀}\AgdaSpace{}%
\AgdaSymbol{\{}\AgdaBound{P}\AgdaSymbol{\}}\AgdaSpace{}%
\AgdaSymbol{→}\AgdaSpace{}%
\AgdaBound{P}\AgdaSpace{}%
\AgdaOperator{\AgdaFunction{*}}\AgdaSpace{}%
\AgdaSymbol{→}\AgdaSpace{}%
\AgdaDatatype{ℕ}\<%
\\
\>[0]\AgdaFunction{\#′}\AgdaSpace{}%
\AgdaSymbol{(}\AgdaBound{n}\AgdaSpace{}%
\AgdaOperator{\AgdaInductiveConstructor{,}}\AgdaSpace{}%
\AgdaSymbol{\AgdaUnderscore{})}\AgdaSpace{}%
\AgdaSymbol{=}\AgdaSpace{}%
\AgdaBound{n}\<%
\\
%
\\[\AgdaEmptyExtraSkip]%
\>[0]\AgdaOperator{\AgdaFunction{\AgdaUnderscore{}::′\AgdaUnderscore{}}}\AgdaSpace{}%
\AgdaSymbol{:}\AgdaSpace{}%
\AgdaSymbol{∀}\AgdaSpace{}%
\AgdaSymbol{\{}\AgdaBound{P}\AgdaSymbol{\}}\AgdaSpace{}%
\AgdaSymbol{→}\AgdaSpace{}%
\AgdaBound{P}\AgdaSpace{}%
\AgdaSymbol{→}\AgdaSpace{}%
\AgdaBound{P}\AgdaSpace{}%
\AgdaOperator{\AgdaFunction{*}}\AgdaSpace{}%
\AgdaSymbol{→}\AgdaSpace{}%
\AgdaBound{P}\AgdaSpace{}%
\AgdaOperator{\AgdaFunction{*}}\<%
\\
\>[0]\AgdaBound{p}\AgdaSpace{}%
\AgdaOperator{\AgdaFunction{::′}}\AgdaSpace{}%
\AgdaSymbol{(}\AgdaNumber{0}%
\>[14]\AgdaOperator{\AgdaInductiveConstructor{,}}\AgdaSpace{}%
\AgdaBound{ps}\AgdaSymbol{)}\AgdaSpace{}%
\AgdaSymbol{=}\AgdaSpace{}%
\AgdaSymbol{(}\AgdaNumber{1}\AgdaSpace{}%
\AgdaOperator{\AgdaInductiveConstructor{,}}\AgdaSpace{}%
\AgdaBound{p}\AgdaSymbol{)}\<%
\\
\>[0]\AgdaBound{p}\AgdaSpace{}%
\AgdaOperator{\AgdaFunction{::′}}\AgdaSpace{}%
\AgdaSymbol{(}\AgdaInductiveConstructor{suc}\AgdaSpace{}%
\AgdaBound{n}%
\>[14]\AgdaOperator{\AgdaInductiveConstructor{,}}\AgdaSpace{}%
\AgdaBound{ps}\AgdaSymbol{)}\AgdaSpace{}%
\AgdaSymbol{=}\AgdaSpace{}%
\AgdaSymbol{(}\AgdaInductiveConstructor{suc}\AgdaSpace{}%
\AgdaSymbol{(}\AgdaInductiveConstructor{suc}\AgdaSpace{}%
\AgdaBound{n}\AgdaSymbol{)}\AgdaSpace{}%
\AgdaOperator{\AgdaInductiveConstructor{,}}\AgdaSpace{}%
\AgdaBound{p}\AgdaSpace{}%
\AgdaOperator{\AgdaInductiveConstructor{,}}\AgdaSpace{}%
\AgdaBound{ps}\AgdaSymbol{)}\<%
\\
%
\\[\AgdaEmptyExtraSkip]%
\>[0]\AgdaOperator{\AgdaFunction{\AgdaUnderscore{}↓′\AgdaUnderscore{}}}\AgdaSpace{}%
\AgdaSymbol{:}\AgdaSpace{}%
\AgdaSymbol{∀}\AgdaSpace{}%
\AgdaSymbol{\{}\AgdaBound{P}\AgdaSymbol{\}}\AgdaSpace{}%
\AgdaSymbol{→}\AgdaSpace{}%
\AgdaBound{P}\AgdaSpace{}%
\AgdaOperator{\AgdaFunction{*}}\AgdaSpace{}%
\AgdaSymbol{→}\AgdaSpace{}%
\AgdaSymbol{(}\AgdaBound{n}\AgdaSpace{}%
\AgdaSymbol{:}\AgdaSpace{}%
\AgdaDatatype{ℕ}\AgdaSymbol{)}\AgdaSpace{}%
\AgdaSymbol{→}\AgdaSpace{}%
\AgdaSymbol{.\{\{}\AgdaBound{\AgdaUnderscore{}}\AgdaSpace{}%
\AgdaSymbol{:}\AgdaSpace{}%
\AgdaRecord{NonZero}\AgdaSpace{}%
\AgdaBound{n}\AgdaSymbol{\}\}}\AgdaSpace{}%
\AgdaSymbol{→}\AgdaSpace{}%
\AgdaPostulate{𝕃}\AgdaSpace{}%
\AgdaBound{P}\<%
\\
\>[0]\AgdaSymbol{(}\AgdaNumber{1}%
\>[14]\AgdaOperator{\AgdaInductiveConstructor{,}}\AgdaSpace{}%
\AgdaBound{p}\AgdaSymbol{)}%
\>[25]\AgdaOperator{\AgdaFunction{↓′}}\AgdaSpace{}%
\AgdaNumber{1}%
\>[41]\AgdaSymbol{=}\AgdaSpace{}%
\AgdaPostulate{η}\AgdaSpace{}%
\AgdaBound{p}\<%
\\
\>[0]\AgdaSymbol{(}\AgdaInductiveConstructor{suc}\AgdaSpace{}%
\AgdaSymbol{(}\AgdaInductiveConstructor{suc}\AgdaSpace{}%
\AgdaBound{n}\AgdaSymbol{)}%
\>[14]\AgdaOperator{\AgdaInductiveConstructor{,}}\AgdaSpace{}%
\AgdaBound{p}\AgdaSpace{}%
\AgdaOperator{\AgdaInductiveConstructor{,}}\AgdaSpace{}%
\AgdaBound{ps}\AgdaSymbol{)}%
\>[25]\AgdaOperator{\AgdaFunction{↓′}}\AgdaSpace{}%
\AgdaNumber{1}%
\>[41]\AgdaSymbol{=}\AgdaSpace{}%
\AgdaPostulate{η}\AgdaSpace{}%
\AgdaBound{p}\<%
\\
\>[0]\AgdaSymbol{(}\AgdaInductiveConstructor{suc}\AgdaSpace{}%
\AgdaSymbol{(}\AgdaInductiveConstructor{suc}\AgdaSpace{}%
\AgdaBound{n}\AgdaSymbol{)}%
\>[14]\AgdaOperator{\AgdaInductiveConstructor{,}}\AgdaSpace{}%
\AgdaBound{p}\AgdaSpace{}%
\AgdaOperator{\AgdaInductiveConstructor{,}}\AgdaSpace{}%
\AgdaBound{ps}\AgdaSymbol{)}%
\>[25]\AgdaOperator{\AgdaFunction{↓′}}\AgdaSpace{}%
\AgdaInductiveConstructor{suc}\AgdaSpace{}%
\AgdaSymbol{(}\AgdaInductiveConstructor{suc}\AgdaSpace{}%
\AgdaBound{i}\AgdaSymbol{)}%
\>[41]\AgdaSymbol{=}\AgdaSpace{}%
\AgdaSymbol{(}\AgdaInductiveConstructor{suc}\AgdaSpace{}%
\AgdaBound{n}\AgdaSpace{}%
\AgdaOperator{\AgdaInductiveConstructor{,}}\AgdaSpace{}%
\AgdaBound{ps}\AgdaSymbol{)}\AgdaSpace{}%
\AgdaOperator{\AgdaFunction{↓′}}\AgdaSpace{}%
\AgdaInductiveConstructor{suc}\AgdaSpace{}%
\AgdaBound{i}\<%
\\
\>[0]\AgdaCatchallClause{\AgdaSymbol{(\AgdaUnderscore{}}}%
\>[14]\AgdaCatchallClause{\AgdaOperator{\AgdaInductiveConstructor{,}}}\AgdaSpace{}%
\AgdaCatchallClause{\AgdaSymbol{\AgdaUnderscore{})}}%
\>[25]\AgdaCatchallClause{\AgdaOperator{\AgdaFunction{↓′}}}\AgdaSpace{}%
\AgdaCatchallClause{\AgdaSymbol{\AgdaUnderscore{}}}%
\>[41]\AgdaSymbol{=}\AgdaSpace{}%
\AgdaPostulate{⊥}\<%
\\
%
\\[\AgdaEmptyExtraSkip]%
\>[0]\AgdaOperator{\AgdaFunction{\AgdaUnderscore{}†′\AgdaUnderscore{}}}\AgdaSpace{}%
\AgdaSymbol{:}\AgdaSpace{}%
\AgdaSymbol{∀}\AgdaSpace{}%
\AgdaSymbol{\{}\AgdaBound{P}\AgdaSymbol{\}}\AgdaSpace{}%
\AgdaSymbol{→}\AgdaSpace{}%
\AgdaBound{P}\AgdaSpace{}%
\AgdaOperator{\AgdaFunction{*}}\AgdaSpace{}%
\AgdaSymbol{→}\AgdaSpace{}%
\AgdaSymbol{(}\AgdaBound{n}\AgdaSpace{}%
\AgdaSymbol{:}\AgdaSpace{}%
\AgdaDatatype{ℕ}\AgdaSymbol{)}\AgdaSpace{}%
\AgdaSymbol{→}\AgdaSpace{}%
\AgdaSymbol{.\{\{}\AgdaBound{\AgdaUnderscore{}}\AgdaSpace{}%
\AgdaSymbol{:}\AgdaSpace{}%
\AgdaRecord{NonZero}\AgdaSpace{}%
\AgdaBound{n}\AgdaSymbol{\}\}}\AgdaSpace{}%
\AgdaSymbol{→}\AgdaSpace{}%
\AgdaPostulate{𝕃}\AgdaSpace{}%
\AgdaSymbol{(}\AgdaBound{P}\AgdaSpace{}%
\AgdaOperator{\AgdaFunction{*}}\AgdaSymbol{)}\<%
\\
\>[0]\AgdaSymbol{(}\AgdaNumber{1}%
\>[14]\AgdaOperator{\AgdaInductiveConstructor{,}}\AgdaSpace{}%
\AgdaBound{p}\AgdaSymbol{)}%
\>[25]\AgdaOperator{\AgdaFunction{†′}}\AgdaSpace{}%
\AgdaNumber{1}%
\>[41]\AgdaSymbol{=}\AgdaSpace{}%
\AgdaPostulate{η}\AgdaSpace{}%
\AgdaSymbol{(}\AgdaNumber{0}\AgdaSpace{}%
\AgdaOperator{\AgdaInductiveConstructor{,}}\AgdaSpace{}%
\AgdaInductiveConstructor{[]}\AgdaSymbol{)}\<%
\\
\>[0]\AgdaSymbol{(}\AgdaInductiveConstructor{suc}\AgdaSpace{}%
\AgdaSymbol{(}\AgdaInductiveConstructor{suc}\AgdaSpace{}%
\AgdaBound{n}\AgdaSymbol{)}%
\>[14]\AgdaOperator{\AgdaInductiveConstructor{,}}\AgdaSpace{}%
\AgdaBound{p}\AgdaSpace{}%
\AgdaOperator{\AgdaInductiveConstructor{,}}\AgdaSpace{}%
\AgdaBound{ps}\AgdaSymbol{)}%
\>[25]\AgdaOperator{\AgdaFunction{†′}}\AgdaSpace{}%
\AgdaNumber{1}%
\>[41]\AgdaSymbol{=}\AgdaSpace{}%
\AgdaPostulate{η}\AgdaSpace{}%
\AgdaSymbol{(}\AgdaInductiveConstructor{suc}\AgdaSpace{}%
\AgdaBound{n}\AgdaSpace{}%
\AgdaOperator{\AgdaInductiveConstructor{,}}\AgdaSpace{}%
\AgdaBound{ps}\AgdaSymbol{)}\<%
\\
\>[0]\AgdaSymbol{(}\AgdaInductiveConstructor{suc}\AgdaSpace{}%
\AgdaSymbol{(}\AgdaInductiveConstructor{suc}\AgdaSpace{}%
\AgdaBound{n}\AgdaSymbol{)}%
\>[14]\AgdaOperator{\AgdaInductiveConstructor{,}}\AgdaSpace{}%
\AgdaBound{p}\AgdaSpace{}%
\AgdaOperator{\AgdaInductiveConstructor{,}}\AgdaSpace{}%
\AgdaBound{ps}\AgdaSymbol{)}%
\>[25]\AgdaOperator{\AgdaFunction{†′}}\AgdaSpace{}%
\AgdaInductiveConstructor{suc}\AgdaSpace{}%
\AgdaSymbol{(}\AgdaInductiveConstructor{suc}\AgdaSpace{}%
\AgdaBound{i}\AgdaSymbol{)}%
\>[41]\AgdaSymbol{=}\AgdaSpace{}%
\AgdaSymbol{(}\AgdaInductiveConstructor{suc}\AgdaSpace{}%
\AgdaBound{n}\AgdaSpace{}%
\AgdaOperator{\AgdaInductiveConstructor{,}}\AgdaSpace{}%
\AgdaBound{ps}\AgdaSymbol{)}\AgdaSpace{}%
\AgdaOperator{\AgdaFunction{†′}}\AgdaSpace{}%
\AgdaInductiveConstructor{suc}\AgdaSpace{}%
\AgdaBound{i}\<%
\\
\>[0]\AgdaCatchallClause{\AgdaSymbol{(\AgdaUnderscore{}}}%
\>[14]\AgdaCatchallClause{\AgdaOperator{\AgdaInductiveConstructor{,}}}\AgdaSpace{}%
\AgdaCatchallClause{\AgdaSymbol{\AgdaUnderscore{})}}%
\>[25]\AgdaCatchallClause{\AgdaOperator{\AgdaFunction{†′}}}\AgdaSpace{}%
\AgdaCatchallClause{\AgdaSymbol{\AgdaUnderscore{}}}%
\>[41]\AgdaSymbol{=}\AgdaSpace{}%
\AgdaPostulate{⊥}\<%
\\
%
\\[\AgdaEmptyExtraSkip]%
\>[0]\AgdaOperator{\AgdaFunction{\AgdaUnderscore{}§′\AgdaUnderscore{}}}\AgdaSpace{}%
\AgdaSymbol{:}\AgdaSpace{}%
\AgdaSymbol{∀}\AgdaSpace{}%
\AgdaSymbol{\{}\AgdaBound{P}\AgdaSymbol{\}}\AgdaSpace{}%
\AgdaSymbol{→}\AgdaSpace{}%
\AgdaBound{P}\AgdaSpace{}%
\AgdaOperator{\AgdaFunction{*}}\AgdaSpace{}%
\AgdaSymbol{→}\AgdaSpace{}%
\AgdaBound{P}\AgdaSpace{}%
\AgdaOperator{\AgdaFunction{*}}\AgdaSpace{}%
\AgdaSymbol{→}\AgdaSpace{}%
\AgdaBound{P}\AgdaSpace{}%
\AgdaOperator{\AgdaFunction{*}}\<%
\\
\>[0]\AgdaSymbol{(}\AgdaNumber{0}\AgdaSpace{}%
\AgdaOperator{\AgdaInductiveConstructor{,}}\AgdaSpace{}%
\AgdaSymbol{\AgdaUnderscore{})}\AgdaSpace{}%
\AgdaOperator{\AgdaFunction{§′}}\AgdaSpace{}%
\AgdaBound{p*}\AgdaSpace{}%
\AgdaSymbol{=}\AgdaSpace{}%
\AgdaBound{p*}\<%
\\
\>[0]\AgdaSymbol{(}\AgdaNumber{1}\AgdaSpace{}%
\AgdaOperator{\AgdaInductiveConstructor{,}}\AgdaSpace{}%
\AgdaBound{p}\AgdaSymbol{)}\AgdaSpace{}%
\AgdaOperator{\AgdaFunction{§′}}\AgdaSpace{}%
\AgdaBound{p*}\AgdaSpace{}%
\AgdaSymbol{=}\AgdaSpace{}%
\AgdaBound{p}\AgdaSpace{}%
\AgdaOperator{\AgdaFunction{::′}}\AgdaSpace{}%
\AgdaBound{p*}\<%
\\
\>[0]\AgdaSymbol{(}\AgdaInductiveConstructor{suc}\AgdaSpace{}%
\AgdaSymbol{(}\AgdaInductiveConstructor{suc}\AgdaSpace{}%
\AgdaBound{n}\AgdaSymbol{)}\AgdaSpace{}%
\AgdaOperator{\AgdaInductiveConstructor{,}}\AgdaSpace{}%
\AgdaBound{p}\AgdaSpace{}%
\AgdaOperator{\AgdaInductiveConstructor{,}}\AgdaSpace{}%
\AgdaBound{ps}\AgdaSymbol{)}\AgdaSpace{}%
\AgdaOperator{\AgdaFunction{§′}}\AgdaSpace{}%
\AgdaBound{p*}\AgdaSpace{}%
\AgdaSymbol{=}\AgdaSpace{}%
\AgdaBound{p}\AgdaSpace{}%
\AgdaOperator{\AgdaFunction{::′}}\AgdaSpace{}%
\AgdaSymbol{((}\AgdaInductiveConstructor{suc}\AgdaSpace{}%
\AgdaBound{n}\AgdaSpace{}%
\AgdaOperator{\AgdaInductiveConstructor{,}}\AgdaSpace{}%
\AgdaBound{ps}\AgdaSymbol{)}\AgdaSpace{}%
\AgdaOperator{\AgdaFunction{§′}}\AgdaSpace{}%
\AgdaBound{p*}\AgdaSymbol{)}\<%
\\
%
\\[\AgdaEmptyExtraSkip]%
\>[0]\AgdaComment{--\ Sequence\ domains}\<%
\\
\>[0]\AgdaComment{--\ D\ ⋆\ =\ 𝕃\ ((D\ \textasciicircum{}\ 0)\ +\ ...\ +\ (D\ \textasciicircum{}\ n)\ +\ ...)}\<%
\\
%
\\[\AgdaEmptyExtraSkip]%
\>[0]\AgdaOperator{\AgdaFunction{\AgdaUnderscore{}⋆}}%
\>[5]\AgdaSymbol{:}\AgdaSpace{}%
\AgdaFunction{Domain}\AgdaSpace{}%
\AgdaSymbol{→}\AgdaSpace{}%
\AgdaFunction{Domain}\<%
\\
\>[0]\AgdaBound{D}\AgdaSpace{}%
\AgdaOperator{\AgdaFunction{⋆}}%
\>[5]\AgdaSymbol{=}\AgdaSpace{}%
\AgdaPostulate{𝕃}\AgdaSpace{}%
\AgdaSymbol{(}\AgdaRecord{Σ}\AgdaSpace{}%
\AgdaDatatype{ℕ}\AgdaSpace{}%
\AgdaSymbol{(}\AgdaBound{D}\AgdaSpace{}%
\AgdaOperator{\AgdaFunction{\textasciicircum{}\AgdaUnderscore{}}}\AgdaSymbol{))}\<%
\\
%
\\[\AgdaEmptyExtraSkip]%
\>[0]\AgdaComment{--\ ⟨⟩\ :\ D\ ⋆}\<%
\\
%
\\[\AgdaEmptyExtraSkip]%
\>[0]\AgdaFunction{⟨⟩}\AgdaSpace{}%
\AgdaSymbol{:}\AgdaSpace{}%
\AgdaSymbol{∀}\AgdaSpace{}%
\AgdaSymbol{\{}\AgdaBound{D}\AgdaSymbol{\}}\AgdaSpace{}%
\AgdaSymbol{→}\AgdaSpace{}%
\AgdaBound{D}\AgdaSpace{}%
\AgdaOperator{\AgdaFunction{⋆}}\<%
\\
\>[0]\AgdaFunction{⟨⟩}\AgdaSpace{}%
\AgdaSymbol{=}\AgdaSpace{}%
\AgdaPostulate{η}\AgdaSpace{}%
\AgdaSymbol{(}\AgdaNumber{0}\AgdaSpace{}%
\AgdaOperator{\AgdaInductiveConstructor{,}}\AgdaSpace{}%
\AgdaInductiveConstructor{[]}\AgdaSymbol{)}\<%
\\
%
\\[\AgdaEmptyExtraSkip]%
\>[0]\AgdaComment{--\ ⟨\ d₁\ ,\ ...\ ,\ dₙ\ ⟩\ :\ D\ ⋆}\<%
\\
%
\\[\AgdaEmptyExtraSkip]%
\>[0]\AgdaOperator{\AgdaFunction{⟨\AgdaUnderscore{}⟩}}\AgdaSpace{}%
\AgdaSymbol{:}\AgdaSpace{}%
\AgdaSymbol{∀}\AgdaSpace{}%
\AgdaSymbol{\{}\AgdaBound{n}\AgdaSpace{}%
\AgdaBound{D}\AgdaSymbol{\}}\AgdaSpace{}%
\AgdaSymbol{→}\AgdaSpace{}%
\AgdaBound{D}\AgdaSpace{}%
\AgdaOperator{\AgdaFunction{\textasciicircum{}}}\AgdaSpace{}%
\AgdaInductiveConstructor{suc}\AgdaSpace{}%
\AgdaBound{n}\AgdaSpace{}%
\AgdaSymbol{→}\AgdaSpace{}%
\AgdaBound{D}\AgdaSpace{}%
\AgdaOperator{\AgdaFunction{⋆}}\<%
\\
\>[0]\AgdaOperator{\AgdaFunction{⟨\AgdaUnderscore{}⟩}}\AgdaSpace{}%
\AgdaSymbol{\{}\AgdaArgument{n}\AgdaSpace{}%
\AgdaSymbol{=}\AgdaSpace{}%
\AgdaBound{n}\AgdaSymbol{\}}\AgdaSpace{}%
\AgdaBound{ds}\AgdaSpace{}%
\AgdaSymbol{=}\AgdaSpace{}%
\AgdaPostulate{η}\AgdaSpace{}%
\AgdaSymbol{(}\AgdaInductiveConstructor{suc}\AgdaSpace{}%
\AgdaBound{n}\AgdaSpace{}%
\AgdaOperator{\AgdaInductiveConstructor{,}}\AgdaSpace{}%
\AgdaBound{ds}\AgdaSymbol{)}\<%
\end{code}
\clearpage
\begin{code}%
\>[0]\AgdaComment{--\ \#\ D\ ⋆\ :\ ℕ\ +⊥}\<%
\\
%
\\[\AgdaEmptyExtraSkip]%
\>[0]\AgdaFunction{\#}\AgdaSpace{}%
\AgdaSymbol{:}\AgdaSpace{}%
\AgdaSymbol{∀}\AgdaSpace{}%
\AgdaSymbol{\{}\AgdaBound{D}\AgdaSymbol{\}}\AgdaSpace{}%
\AgdaSymbol{→}\AgdaSpace{}%
\AgdaBound{D}\AgdaSpace{}%
\AgdaOperator{\AgdaFunction{⋆}}\AgdaSpace{}%
\AgdaSymbol{→}\AgdaSpace{}%
\AgdaDatatype{ℕ}\AgdaSpace{}%
\AgdaOperator{\AgdaFunction{+⊥}}\<%
\\
\>[0]\AgdaFunction{\#}\AgdaSpace{}%
\AgdaBound{d⋆}\AgdaSpace{}%
\AgdaSymbol{=}\AgdaSpace{}%
\AgdaSymbol{((λ}\AgdaSpace{}%
\AgdaBound{p*}\AgdaSpace{}%
\AgdaSymbol{→}\AgdaSpace{}%
\AgdaPostulate{η}\AgdaSpace{}%
\AgdaSymbol{(}\AgdaFunction{\#′}\AgdaSpace{}%
\AgdaBound{p*}\AgdaSymbol{))}\AgdaSpace{}%
\AgdaOperator{\AgdaPostulate{♯}}\AgdaSymbol{)}\AgdaSpace{}%
\AgdaBound{d⋆}\<%
\\
%
\\[\AgdaEmptyExtraSkip]%
\>[0]\AgdaComment{--\ d⋆₁\ §\ d⋆₂\ :\ D\ ⋆}\<%
\\
%
\\[\AgdaEmptyExtraSkip]%
\>[0]\AgdaOperator{\AgdaFunction{\AgdaUnderscore{}§\AgdaUnderscore{}}}\AgdaSpace{}%
\AgdaSymbol{:}\AgdaSpace{}%
\AgdaSymbol{∀}\AgdaSpace{}%
\AgdaSymbol{\{}\AgdaBound{D}\AgdaSymbol{\}}\AgdaSpace{}%
\AgdaSymbol{→}\AgdaSpace{}%
\AgdaBound{D}\AgdaSpace{}%
\AgdaOperator{\AgdaFunction{⋆}}\AgdaSpace{}%
\AgdaSymbol{→}\AgdaSpace{}%
\AgdaBound{D}\AgdaSpace{}%
\AgdaOperator{\AgdaFunction{⋆}}\AgdaSpace{}%
\AgdaSymbol{→}\AgdaSpace{}%
\AgdaBound{D}\AgdaSpace{}%
\AgdaOperator{\AgdaFunction{⋆}}\<%
\\
\>[0]\AgdaBound{d⋆₁}\AgdaSpace{}%
\AgdaOperator{\AgdaFunction{§}}\AgdaSpace{}%
\AgdaBound{d⋆₂}\AgdaSpace{}%
\AgdaSymbol{=}\AgdaSpace{}%
\AgdaSymbol{((λ}\AgdaSpace{}%
\AgdaBound{p*₁}\AgdaSpace{}%
\AgdaSymbol{→}\AgdaSpace{}%
\AgdaSymbol{((λ}\AgdaSpace{}%
\AgdaBound{p*₂}\AgdaSpace{}%
\AgdaSymbol{→}\AgdaSpace{}%
\AgdaPostulate{η}\AgdaSpace{}%
\AgdaSymbol{(}\AgdaBound{p*₁}\AgdaSpace{}%
\AgdaOperator{\AgdaFunction{§′}}\AgdaSpace{}%
\AgdaBound{p*₂}\AgdaSymbol{))}\AgdaSpace{}%
\AgdaOperator{\AgdaPostulate{♯}}\AgdaSymbol{)}\AgdaSpace{}%
\AgdaBound{d⋆₂}\AgdaSymbol{)}\AgdaSpace{}%
\AgdaOperator{\AgdaPostulate{♯}}\AgdaSymbol{)}\AgdaSpace{}%
\AgdaBound{d⋆₁}\<%
\\
%
\\[\AgdaEmptyExtraSkip]%
\>[0]\AgdaKeyword{open}\AgdaSpace{}%
\AgdaKeyword{import}\AgdaSpace{}%
\AgdaModule{Function}\<%
\\
\>[0][@{}l@{\AgdaIndent{0}}]%
\>[2]\AgdaKeyword{using}\AgdaSpace{}%
\AgdaSymbol{(}\AgdaFunction{id}\AgdaSymbol{;}\AgdaSpace{}%
\AgdaOperator{\AgdaFunction{\AgdaUnderscore{}∘\AgdaUnderscore{}}}\AgdaSymbol{)}\AgdaSpace{}%
\AgdaKeyword{public}\<%
\\
%
\\[\AgdaEmptyExtraSkip]%
\>[0]\AgdaComment{--\ d⋆\ ↓\ k\ :\ D\ \ (k\ ≥\ 1;\ k\ <\ \#\ d⋆)}\<%
\\
%
\\[\AgdaEmptyExtraSkip]%
\>[0]\AgdaOperator{\AgdaFunction{\AgdaUnderscore{}↓\AgdaUnderscore{}}}\AgdaSpace{}%
\AgdaSymbol{:}\AgdaSpace{}%
\AgdaSymbol{∀}\AgdaSpace{}%
\AgdaSymbol{\{}\AgdaBound{D}\AgdaSymbol{\}}\AgdaSpace{}%
\AgdaSymbol{→}\AgdaSpace{}%
\AgdaBound{D}\AgdaSpace{}%
\AgdaOperator{\AgdaFunction{⋆}}\AgdaSpace{}%
\AgdaSymbol{→}\AgdaSpace{}%
\AgdaSymbol{(}\AgdaBound{n}\AgdaSpace{}%
\AgdaSymbol{:}\AgdaSpace{}%
\AgdaDatatype{ℕ}\AgdaSymbol{)}\AgdaSpace{}%
\AgdaSymbol{→}\AgdaSpace{}%
\AgdaSymbol{.\{\{}\AgdaBound{\AgdaUnderscore{}}\AgdaSpace{}%
\AgdaSymbol{:}\AgdaSpace{}%
\AgdaRecord{NonZero}\AgdaSpace{}%
\AgdaBound{n}\AgdaSymbol{\}\}}\AgdaSpace{}%
\AgdaSymbol{→}\AgdaSpace{}%
\AgdaBound{D}\<%
\\
\>[0]\AgdaBound{d⋆}\AgdaSpace{}%
\AgdaOperator{\AgdaFunction{↓}}\AgdaSpace{}%
\AgdaBound{n}\AgdaSpace{}%
\AgdaSymbol{=}\AgdaSpace{}%
\AgdaSymbol{(}\AgdaFunction{id}\AgdaSpace{}%
\AgdaOperator{\AgdaPostulate{♯}}\AgdaSymbol{)}\AgdaSpace{}%
\AgdaSymbol{(((λ}\AgdaSpace{}%
\AgdaBound{p*}\AgdaSpace{}%
\AgdaSymbol{→}\AgdaSpace{}%
\AgdaBound{p*}\AgdaSpace{}%
\AgdaOperator{\AgdaFunction{↓′}}\AgdaSpace{}%
\AgdaBound{n}\AgdaSymbol{)}\AgdaSpace{}%
\AgdaOperator{\AgdaPostulate{♯}}\AgdaSymbol{)}\AgdaSpace{}%
\AgdaBound{d⋆}\AgdaSymbol{)}\<%
\\
%
\\[\AgdaEmptyExtraSkip]%
\>[0]\AgdaComment{--\ d⋆\ †\ k\ :\ D\ ⋆\ \ (k\ ≥\ 1)}\<%
\\
%
\\[\AgdaEmptyExtraSkip]%
\>[0]\AgdaOperator{\AgdaFunction{\AgdaUnderscore{}†\AgdaUnderscore{}}}\AgdaSpace{}%
\AgdaSymbol{:}\AgdaSpace{}%
\AgdaSymbol{∀}\AgdaSpace{}%
\AgdaSymbol{\{}\AgdaBound{D}\AgdaSymbol{\}}\AgdaSpace{}%
\AgdaSymbol{→}\AgdaSpace{}%
\AgdaBound{D}\AgdaSpace{}%
\AgdaOperator{\AgdaFunction{⋆}}\AgdaSpace{}%
\AgdaSymbol{→}\AgdaSpace{}%
\AgdaSymbol{(}\AgdaBound{n}\AgdaSpace{}%
\AgdaSymbol{:}\AgdaSpace{}%
\AgdaDatatype{ℕ}\AgdaSymbol{)}\AgdaSpace{}%
\AgdaSymbol{→}\AgdaSpace{}%
\AgdaSymbol{.\{\{}\AgdaBound{\AgdaUnderscore{}}\AgdaSpace{}%
\AgdaSymbol{:}\AgdaSpace{}%
\AgdaRecord{NonZero}\AgdaSpace{}%
\AgdaBound{n}\AgdaSymbol{\}\}}\AgdaSpace{}%
\AgdaSymbol{→}\AgdaSpace{}%
\AgdaBound{D}\AgdaSpace{}%
\AgdaOperator{\AgdaFunction{⋆}}\<%
\\
\>[0]\AgdaBound{d⋆}\AgdaSpace{}%
\AgdaOperator{\AgdaFunction{†}}\AgdaSpace{}%
\AgdaBound{n}\AgdaSpace{}%
\AgdaSymbol{=}\AgdaSpace{}%
\AgdaSymbol{(}\AgdaFunction{id}\AgdaSpace{}%
\AgdaOperator{\AgdaPostulate{♯}}\AgdaSymbol{)}\AgdaSpace{}%
\AgdaSymbol{(((λ}\AgdaSpace{}%
\AgdaBound{p*}\AgdaSpace{}%
\AgdaSymbol{→}\AgdaSpace{}%
\AgdaPostulate{η}\AgdaSpace{}%
\AgdaSymbol{(}\AgdaBound{p*}\AgdaSpace{}%
\AgdaOperator{\AgdaFunction{†′}}\AgdaSpace{}%
\AgdaBound{n}\AgdaSymbol{))}\AgdaSpace{}%
\AgdaOperator{\AgdaPostulate{♯}}\AgdaSymbol{)}\AgdaSpace{}%
\AgdaBound{d⋆}\AgdaSymbol{)}\<%
\\
%
\\[\AgdaEmptyExtraSkip]%
\>[0]\AgdaComment{------------------------------------------------------------------------}\<%
\\
\>[0]\AgdaComment{--\ McCarthy\ conditional}\<%
\\
%
\\[\AgdaEmptyExtraSkip]%
\>[0]\AgdaComment{--\ t\ ⟶\ d₁\ ,\ d₂\ :\ D\ \ (t\ :\ Bool\ +⊥\ ;\ d₁,\ d₂\ :\ D)}\<%
\\
%
\\[\AgdaEmptyExtraSkip]%
\>[0]\AgdaKeyword{open}\AgdaSpace{}%
\AgdaKeyword{import}\AgdaSpace{}%
\AgdaModule{Data.Bool.Base}\<%
\\
\>[0][@{}l@{\AgdaIndent{0}}]%
\>[2]\AgdaKeyword{using}\AgdaSpace{}%
\AgdaSymbol{(}\AgdaDatatype{Bool}\AgdaSymbol{;}\AgdaSpace{}%
\AgdaInductiveConstructor{true}\AgdaSymbol{;}\AgdaSpace{}%
\AgdaInductiveConstructor{false}\AgdaSymbol{;}\AgdaSpace{}%
\AgdaOperator{\AgdaFunction{if\AgdaUnderscore{}then\AgdaUnderscore{}else\AgdaUnderscore{}}}\AgdaSymbol{)}\AgdaSpace{}%
\AgdaKeyword{public}\<%
\\
%
\\[\AgdaEmptyExtraSkip]%
\>[0]\AgdaKeyword{postulate}\<%
\\
\>[0][@{}l@{\AgdaIndent{0}}]%
\>[2]\AgdaOperator{\AgdaPostulate{\AgdaUnderscore{}⟶\AgdaUnderscore{},\AgdaUnderscore{}}}\AgdaSpace{}%
\AgdaSymbol{:}\AgdaSpace{}%
\AgdaSymbol{\{}\AgdaBound{D}\AgdaSpace{}%
\AgdaSymbol{:}\AgdaSpace{}%
\AgdaFunction{Domain}\AgdaSymbol{\}}\AgdaSpace{}%
\AgdaSymbol{→}\AgdaSpace{}%
\AgdaDatatype{Bool}\AgdaSpace{}%
\AgdaOperator{\AgdaFunction{+⊥}}\AgdaSpace{}%
\AgdaSymbol{→}\AgdaSpace{}%
\AgdaBound{D}\AgdaSpace{}%
\AgdaSymbol{→}\AgdaSpace{}%
\AgdaBound{D}\AgdaSpace{}%
\AgdaSymbol{→}\AgdaSpace{}%
\AgdaBound{D}\<%
\\
%
\\[\AgdaEmptyExtraSkip]%
%
\>[2]\AgdaComment{--\ Properties}\<%
\\
%
\>[2]\AgdaPostulate{true-cond}%
\>[15]\AgdaSymbol{:}\AgdaSpace{}%
\AgdaSymbol{∀}\AgdaSpace{}%
\AgdaSymbol{\{}\AgdaBound{D}\AgdaSymbol{\}}\AgdaSpace{}%
\AgdaSymbol{\{}\AgdaBound{d₁}\AgdaSpace{}%
\AgdaBound{d₂}\AgdaSpace{}%
\AgdaSymbol{:}\AgdaSpace{}%
\AgdaBound{D}\AgdaSymbol{\}}\AgdaSpace{}%
\AgdaSymbol{→}\AgdaSpace{}%
\AgdaSymbol{(}\AgdaPostulate{η}\AgdaSpace{}%
\AgdaInductiveConstructor{true}\AgdaSpace{}%
\AgdaOperator{\AgdaPostulate{⟶}}\AgdaSpace{}%
\AgdaBound{d₁}\AgdaSpace{}%
\AgdaOperator{\AgdaPostulate{,}}\AgdaSpace{}%
\AgdaBound{d₂}\AgdaSymbol{)}%
\>[57]\AgdaOperator{\AgdaDatatype{≡}}\AgdaSpace{}%
\AgdaBound{d₁}\<%
\\
%
\>[2]\AgdaPostulate{false-cond}%
\>[15]\AgdaSymbol{:}\AgdaSpace{}%
\AgdaSymbol{∀}\AgdaSpace{}%
\AgdaSymbol{\{}\AgdaBound{D}\AgdaSymbol{\}}\AgdaSpace{}%
\AgdaSymbol{\{}\AgdaBound{d₁}\AgdaSpace{}%
\AgdaBound{d₂}\AgdaSpace{}%
\AgdaSymbol{:}\AgdaSpace{}%
\AgdaBound{D}\AgdaSymbol{\}}\AgdaSpace{}%
\AgdaSymbol{→}\AgdaSpace{}%
\AgdaSymbol{(}\AgdaPostulate{η}\AgdaSpace{}%
\AgdaInductiveConstructor{false}\AgdaSpace{}%
\AgdaOperator{\AgdaPostulate{⟶}}\AgdaSpace{}%
\AgdaBound{d₁}\AgdaSpace{}%
\AgdaOperator{\AgdaPostulate{,}}\AgdaSpace{}%
\AgdaBound{d₂}\AgdaSymbol{)}\AgdaSpace{}%
\AgdaOperator{\AgdaDatatype{≡}}\AgdaSpace{}%
\AgdaBound{d₂}\<%
\\
%
\>[2]\AgdaPostulate{bottom-cond}%
\>[15]\AgdaSymbol{:}\AgdaSpace{}%
\AgdaSymbol{∀}\AgdaSpace{}%
\AgdaSymbol{\{}\AgdaBound{D}\AgdaSymbol{\}}\AgdaSpace{}%
\AgdaSymbol{\{}\AgdaBound{d₁}\AgdaSpace{}%
\AgdaBound{d₂}\AgdaSpace{}%
\AgdaSymbol{:}\AgdaSpace{}%
\AgdaBound{D}\AgdaSymbol{\}}\AgdaSpace{}%
\AgdaSymbol{→}\AgdaSpace{}%
\AgdaSymbol{(}\AgdaPostulate{⊥}\AgdaSpace{}%
\AgdaOperator{\AgdaPostulate{⟶}}\AgdaSpace{}%
\AgdaBound{d₁}\AgdaSpace{}%
\AgdaOperator{\AgdaPostulate{,}}\AgdaSpace{}%
\AgdaBound{d₂}\AgdaSymbol{)}%
\>[57]\AgdaOperator{\AgdaDatatype{≡}}\AgdaSpace{}%
\AgdaPostulate{⊥}\<%
\\
%
\\[\AgdaEmptyExtraSkip]%
\>[0]\AgdaComment{------------------------------------------------------------------------}\<%
\\
\>[0]\AgdaComment{--\ Meta-Strings}\<%
\\
%
\\[\AgdaEmptyExtraSkip]%
\>[0]\AgdaKeyword{open}\AgdaSpace{}%
\AgdaKeyword{import}\AgdaSpace{}%
\AgdaModule{Data.String.Base}\<%
\\
\>[0][@{}l@{\AgdaIndent{0}}]%
\>[2]\AgdaKeyword{using}\AgdaSpace{}%
\AgdaSymbol{(}\AgdaPostulate{String}\AgdaSymbol{)}\AgdaSpace{}%
\AgdaKeyword{public}\<%
\\
\>[0]\<%
\end{code}  
\clearpage
\begin{code}%
\>[0]\AgdaKeyword{module}\AgdaSpace{}%
\AgdaModule{PCF.Types}\AgdaSpace{}%
\AgdaKeyword{where}\<%
\\
%
\\[\AgdaEmptyExtraSkip]%
\>[0]\AgdaKeyword{open}\AgdaSpace{}%
\AgdaKeyword{import}\AgdaSpace{}%
\AgdaModule{Data.Bool.Base}\<%
\\
\>[0][@{}l@{\AgdaIndent{0}}]%
\>[2]\AgdaKeyword{using}\AgdaSpace{}%
\AgdaSymbol{(}\AgdaDatatype{Bool}\AgdaSymbol{)}\<%
\\
\>[0]\AgdaKeyword{open}\AgdaSpace{}%
\AgdaKeyword{import}\AgdaSpace{}%
\AgdaModule{Agda.Builtin.Nat}\<%
\\
\>[0][@{}l@{\AgdaIndent{0}}]%
\>[2]\AgdaKeyword{using}\AgdaSpace{}%
\AgdaSymbol{(}\AgdaDatatype{Nat}\AgdaSymbol{)}\<%
\\
%
\\[\AgdaEmptyExtraSkip]%
\>[0]\AgdaKeyword{open}\AgdaSpace{}%
\AgdaKeyword{import}\AgdaSpace{}%
\AgdaModule{PCF.Domain-Notation}\<%
\\
\>[0][@{}l@{\AgdaIndent{0}}]%
\>[2]\AgdaKeyword{using}\AgdaSpace{}%
\AgdaSymbol{(}\AgdaFunction{Domain}\AgdaSymbol{;}\AgdaSpace{}%
\AgdaOperator{\AgdaFunction{\AgdaUnderscore{}+⊥}}\AgdaSymbol{)}\<%
\\
%
\\[\AgdaEmptyExtraSkip]%
\>[0]\AgdaComment{--\ Syntax}\<%
\\
%
\\[\AgdaEmptyExtraSkip]%
\>[0]\AgdaKeyword{data}\AgdaSpace{}%
\AgdaDatatype{Types}\AgdaSpace{}%
\AgdaSymbol{:}\AgdaSpace{}%
\AgdaPrimitive{Set}\AgdaSpace{}%
\AgdaKeyword{where}\<%
\\
\>[0][@{}l@{\AgdaIndent{0}}]%
\>[2]\AgdaInductiveConstructor{ι}%
\>[7]\AgdaSymbol{:}\AgdaSpace{}%
\AgdaDatatype{Types}%
\>[32]\AgdaComment{--\ natural\ numbers}\<%
\\
%
\>[2]\AgdaInductiveConstructor{o}%
\>[7]\AgdaSymbol{:}\AgdaSpace{}%
\AgdaDatatype{Types}%
\>[32]\AgdaComment{--\ Boolean\ truthvalues}\<%
\\
%
\>[2]\AgdaOperator{\AgdaInductiveConstructor{\AgdaUnderscore{}⇒\AgdaUnderscore{}}}%
\>[7]\AgdaSymbol{:}\AgdaSpace{}%
\AgdaDatatype{Types}\AgdaSpace{}%
\AgdaSymbol{→}\AgdaSpace{}%
\AgdaDatatype{Types}\AgdaSpace{}%
\AgdaSymbol{→}\AgdaSpace{}%
\AgdaDatatype{Types}%
\>[32]\AgdaComment{--\ functions}\<%
\\
%
\\[\AgdaEmptyExtraSkip]%
\>[0]\AgdaKeyword{variable}\AgdaSpace{}%
\AgdaGeneralizable{σ}\AgdaSpace{}%
\AgdaGeneralizable{τ}\AgdaSpace{}%
\AgdaSymbol{:}\AgdaSpace{}%
\AgdaDatatype{Types}\<%
\\
%
\\[\AgdaEmptyExtraSkip]%
\>[0]\AgdaKeyword{infixr}\AgdaSpace{}%
\AgdaNumber{1}\AgdaSpace{}%
\AgdaOperator{\AgdaInductiveConstructor{\AgdaUnderscore{}⇒\AgdaUnderscore{}}}\<%
\\
%
\\[\AgdaEmptyExtraSkip]%
\>[0]\AgdaComment{--\ Semantics\ 𝒟}\<%
\\
%
\\[\AgdaEmptyExtraSkip]%
\>[0]\AgdaFunction{𝒟}\AgdaSpace{}%
\AgdaSymbol{:}\AgdaSpace{}%
\AgdaDatatype{Types}\AgdaSpace{}%
\AgdaSymbol{→}\AgdaSpace{}%
\AgdaFunction{Domain}\<%
\\
%
\\[\AgdaEmptyExtraSkip]%
\>[0]\AgdaFunction{𝒟}\AgdaSpace{}%
\AgdaInductiveConstructor{ι}%
\>[11]\AgdaSymbol{=}\AgdaSpace{}%
\AgdaDatatype{Nat}%
\>[18]\AgdaOperator{\AgdaFunction{+⊥}}\<%
\\
\>[0]\AgdaFunction{𝒟}\AgdaSpace{}%
\AgdaInductiveConstructor{o}%
\>[11]\AgdaSymbol{=}\AgdaSpace{}%
\AgdaDatatype{Bool}\AgdaSpace{}%
\AgdaOperator{\AgdaFunction{+⊥}}\<%
\\
\>[0]\AgdaFunction{𝒟}\AgdaSpace{}%
\AgdaSymbol{(}\AgdaBound{σ}\AgdaSpace{}%
\AgdaOperator{\AgdaInductiveConstructor{⇒}}\AgdaSpace{}%
\AgdaBound{τ}\AgdaSymbol{)}%
\>[11]\AgdaSymbol{=}\AgdaSpace{}%
\AgdaFunction{𝒟}\AgdaSpace{}%
\AgdaBound{σ}\AgdaSpace{}%
\AgdaSymbol{→}\AgdaSpace{}%
\AgdaFunction{𝒟}\AgdaSpace{}%
\AgdaBound{τ}\<%
\\
%
\\[\AgdaEmptyExtraSkip]%
\>[0]\AgdaKeyword{variable}\AgdaSpace{}%
\AgdaGeneralizable{x}\AgdaSpace{}%
\AgdaGeneralizable{y}\AgdaSpace{}%
\AgdaGeneralizable{z}\AgdaSpace{}%
\AgdaSymbol{:}\AgdaSpace{}%
\AgdaFunction{𝒟}\AgdaSpace{}%
\AgdaGeneralizable{σ}\<%
\end{code}
\clearpage
\begin{code}%
\>[0]\AgdaKeyword{module}\AgdaSpace{}%
\AgdaModule{PCF.Constants}\AgdaSpace{}%
\AgdaKeyword{where}\<%
\\
%
\\[\AgdaEmptyExtraSkip]%
\>[0]\AgdaKeyword{open}\AgdaSpace{}%
\AgdaKeyword{import}\AgdaSpace{}%
\AgdaModule{Data.Bool.Base}\<%
\\
\>[0][@{}l@{\AgdaIndent{0}}]%
\>[2]\AgdaKeyword{using}\AgdaSpace{}%
\AgdaSymbol{(}\AgdaDatatype{Bool}\AgdaSymbol{;}\AgdaSpace{}%
\AgdaInductiveConstructor{true}\AgdaSymbol{;}\AgdaSpace{}%
\AgdaInductiveConstructor{false}\AgdaSymbol{;}\AgdaSpace{}%
\AgdaOperator{\AgdaFunction{if\AgdaUnderscore{}then\AgdaUnderscore{}else\AgdaUnderscore{}}}\AgdaSymbol{)}\<%
\\
\>[0]\AgdaKeyword{open}\AgdaSpace{}%
\AgdaKeyword{import}\AgdaSpace{}%
\AgdaModule{Agda.Builtin.Nat}\<%
\\
\>[0][@{}l@{\AgdaIndent{0}}]%
\>[2]\AgdaKeyword{using}\AgdaSpace{}%
\AgdaSymbol{(}\AgdaDatatype{Nat}\AgdaSymbol{;}\AgdaSpace{}%
\AgdaOperator{\AgdaPrimitive{\AgdaUnderscore{}+\AgdaUnderscore{}}}\AgdaSymbol{;}\AgdaSpace{}%
\AgdaOperator{\AgdaPrimitive{\AgdaUnderscore{}-\AgdaUnderscore{}}}\AgdaSymbol{;}\AgdaSpace{}%
\AgdaOperator{\AgdaPrimitive{\AgdaUnderscore{}==\AgdaUnderscore{}}}\AgdaSymbol{)}\<%
\\
%
\\[\AgdaEmptyExtraSkip]%
\>[0]\AgdaKeyword{open}\AgdaSpace{}%
\AgdaKeyword{import}\AgdaSpace{}%
\AgdaModule{PCF.Domain-Notation}\<%
\\
\>[0][@{}l@{\AgdaIndent{0}}]%
\>[2]\AgdaKeyword{using}\AgdaSpace{}%
\AgdaSymbol{(}\AgdaPostulate{η}\AgdaSymbol{;}\AgdaSpace{}%
\AgdaOperator{\AgdaPostulate{\AgdaUnderscore{}♯}}\AgdaSymbol{;}\AgdaSpace{}%
\AgdaPostulate{fix}\AgdaSymbol{;}\AgdaSpace{}%
\AgdaPostulate{⊥}\AgdaSymbol{;}\AgdaSpace{}%
\AgdaOperator{\AgdaPostulate{\AgdaUnderscore{}⟶\AgdaUnderscore{},\AgdaUnderscore{}}}\AgdaSymbol{)}\<%
\\
\>[0]\AgdaKeyword{open}\AgdaSpace{}%
\AgdaKeyword{import}\AgdaSpace{}%
\AgdaModule{PCF.Types}\<%
\\
\>[0][@{}l@{\AgdaIndent{0}}]%
\>[2]\AgdaKeyword{using}\AgdaSpace{}%
\AgdaSymbol{(}\AgdaDatatype{Types}\AgdaSymbol{;}\AgdaSpace{}%
\AgdaInductiveConstructor{o}\AgdaSymbol{;}\AgdaSpace{}%
\AgdaInductiveConstructor{ι}\AgdaSymbol{;}\AgdaSpace{}%
\AgdaOperator{\AgdaInductiveConstructor{\AgdaUnderscore{}⇒\AgdaUnderscore{}}}\AgdaSymbol{;}\AgdaSpace{}%
\AgdaGeneralizable{σ}\AgdaSymbol{;}\AgdaSpace{}%
\AgdaFunction{𝒟}\AgdaSymbol{)}\<%
\\
%
\\[\AgdaEmptyExtraSkip]%
\>[0]\AgdaComment{--\ Syntax}\<%
\\
%
\\[\AgdaEmptyExtraSkip]%
\>[0]\AgdaKeyword{data}\AgdaSpace{}%
\AgdaDatatype{ℒ}\AgdaSpace{}%
\AgdaSymbol{:}\AgdaSpace{}%
\AgdaDatatype{Types}\AgdaSpace{}%
\AgdaSymbol{→}\AgdaSpace{}%
\AgdaPrimitive{Set}\AgdaSpace{}%
\AgdaKeyword{where}\<%
\\
\>[0][@{}l@{\AgdaIndent{0}}]%
\>[2]\AgdaInductiveConstructor{tt}%
\>[7]\AgdaSymbol{:}\AgdaSpace{}%
\AgdaDatatype{ℒ}\AgdaSpace{}%
\AgdaInductiveConstructor{o}\<%
\\
%
\>[2]\AgdaInductiveConstructor{ff}%
\>[7]\AgdaSymbol{:}\AgdaSpace{}%
\AgdaDatatype{ℒ}\AgdaSpace{}%
\AgdaInductiveConstructor{o}\<%
\\
%
\>[2]\AgdaInductiveConstructor{⊃ᵢ}%
\>[7]\AgdaSymbol{:}\AgdaSpace{}%
\AgdaDatatype{ℒ}\AgdaSpace{}%
\AgdaSymbol{(}\AgdaInductiveConstructor{o}\AgdaSpace{}%
\AgdaOperator{\AgdaInductiveConstructor{⇒}}\AgdaSpace{}%
\AgdaInductiveConstructor{ι}\AgdaSpace{}%
\AgdaOperator{\AgdaInductiveConstructor{⇒}}\AgdaSpace{}%
\AgdaInductiveConstructor{ι}\AgdaSpace{}%
\AgdaOperator{\AgdaInductiveConstructor{⇒}}\AgdaSpace{}%
\AgdaInductiveConstructor{ι}\AgdaSymbol{)}\<%
\\
%
\>[2]\AgdaInductiveConstructor{⊃ₒ}%
\>[7]\AgdaSymbol{:}\AgdaSpace{}%
\AgdaDatatype{ℒ}\AgdaSpace{}%
\AgdaSymbol{(}\AgdaInductiveConstructor{o}\AgdaSpace{}%
\AgdaOperator{\AgdaInductiveConstructor{⇒}}\AgdaSpace{}%
\AgdaInductiveConstructor{o}\AgdaSpace{}%
\AgdaOperator{\AgdaInductiveConstructor{⇒}}\AgdaSpace{}%
\AgdaInductiveConstructor{o}\AgdaSpace{}%
\AgdaOperator{\AgdaInductiveConstructor{⇒}}\AgdaSpace{}%
\AgdaInductiveConstructor{o}\AgdaSymbol{)}\<%
\\
%
\>[2]\AgdaInductiveConstructor{Y}%
\>[7]\AgdaSymbol{:}\AgdaSpace{}%
\AgdaSymbol{\{}\AgdaBound{σ}\AgdaSpace{}%
\AgdaSymbol{:}\AgdaSpace{}%
\AgdaDatatype{Types}\AgdaSymbol{\}}\AgdaSpace{}%
\AgdaSymbol{→}\AgdaSpace{}%
\AgdaDatatype{ℒ}\AgdaSpace{}%
\AgdaSymbol{((}\AgdaBound{σ}\AgdaSpace{}%
\AgdaOperator{\AgdaInductiveConstructor{⇒}}\AgdaSpace{}%
\AgdaBound{σ}\AgdaSymbol{)}\AgdaSpace{}%
\AgdaOperator{\AgdaInductiveConstructor{⇒}}\AgdaSpace{}%
\AgdaBound{σ}\AgdaSymbol{)}\<%
\\
%
\>[2]\AgdaInductiveConstructor{k}%
\>[7]\AgdaSymbol{:}\AgdaSpace{}%
\AgdaSymbol{(}\AgdaBound{n}\AgdaSpace{}%
\AgdaSymbol{:}\AgdaSpace{}%
\AgdaDatatype{Nat}\AgdaSymbol{)}\AgdaSpace{}%
\AgdaSymbol{→}\AgdaSpace{}%
\AgdaDatatype{ℒ}\AgdaSpace{}%
\AgdaInductiveConstructor{ι}\<%
\\
%
\>[2]\AgdaInductiveConstructor{+1′}%
\>[7]\AgdaSymbol{:}\AgdaSpace{}%
\AgdaDatatype{ℒ}\AgdaSpace{}%
\AgdaSymbol{(}\AgdaInductiveConstructor{ι}\AgdaSpace{}%
\AgdaOperator{\AgdaInductiveConstructor{⇒}}\AgdaSpace{}%
\AgdaInductiveConstructor{ι}\AgdaSymbol{)}\<%
\\
%
\>[2]\AgdaInductiveConstructor{-1′}%
\>[7]\AgdaSymbol{:}\AgdaSpace{}%
\AgdaDatatype{ℒ}\AgdaSpace{}%
\AgdaSymbol{(}\AgdaInductiveConstructor{ι}\AgdaSpace{}%
\AgdaOperator{\AgdaInductiveConstructor{⇒}}\AgdaSpace{}%
\AgdaInductiveConstructor{ι}\AgdaSymbol{)}\<%
\\
%
\>[2]\AgdaInductiveConstructor{Z}%
\>[7]\AgdaSymbol{:}\AgdaSpace{}%
\AgdaDatatype{ℒ}\AgdaSpace{}%
\AgdaSymbol{(}\AgdaInductiveConstructor{ι}\AgdaSpace{}%
\AgdaOperator{\AgdaInductiveConstructor{⇒}}\AgdaSpace{}%
\AgdaInductiveConstructor{o}\AgdaSymbol{)}\<%
\\
%
\\[\AgdaEmptyExtraSkip]%
\>[0]\AgdaKeyword{variable}\AgdaSpace{}%
\AgdaGeneralizable{c}\AgdaSpace{}%
\AgdaSymbol{:}\AgdaSpace{}%
\AgdaDatatype{ℒ}\AgdaSpace{}%
\AgdaGeneralizable{σ}\<%
\\
%
\\[\AgdaEmptyExtraSkip]%
\>[0]\AgdaComment{--\ Semantics}\<%
\\
%
\\[\AgdaEmptyExtraSkip]%
\>[0]\AgdaOperator{\AgdaFunction{𝒜⟦\AgdaUnderscore{}⟧}}\AgdaSpace{}%
\AgdaSymbol{:}\AgdaSpace{}%
\AgdaDatatype{ℒ}\AgdaSpace{}%
\AgdaGeneralizable{σ}\AgdaSpace{}%
\AgdaSymbol{→}\AgdaSpace{}%
\AgdaFunction{𝒟}\AgdaSpace{}%
\AgdaGeneralizable{σ}\<%
\\
%
\\[\AgdaEmptyExtraSkip]%
\>[0]\AgdaOperator{\AgdaFunction{𝒜⟦}}\AgdaSpace{}%
\AgdaInductiveConstructor{tt}%
\>[8]\AgdaOperator{\AgdaFunction{⟧}}\AgdaSpace{}%
\AgdaSymbol{=}%
\>[13]\AgdaPostulate{η}\AgdaSpace{}%
\AgdaInductiveConstructor{true}\<%
\\
\>[0]\AgdaOperator{\AgdaFunction{𝒜⟦}}\AgdaSpace{}%
\AgdaInductiveConstructor{ff}%
\>[8]\AgdaOperator{\AgdaFunction{⟧}}\AgdaSpace{}%
\AgdaSymbol{=}%
\>[13]\AgdaPostulate{η}\AgdaSpace{}%
\AgdaInductiveConstructor{false}\<%
\\
\>[0]\AgdaOperator{\AgdaFunction{𝒜⟦}}\AgdaSpace{}%
\AgdaInductiveConstructor{⊃ᵢ}%
\>[8]\AgdaOperator{\AgdaFunction{⟧}}\AgdaSpace{}%
\AgdaSymbol{=}%
\>[13]\AgdaOperator{\AgdaPostulate{\AgdaUnderscore{}⟶\AgdaUnderscore{},\AgdaUnderscore{}}}\<%
\\
\>[0]\AgdaOperator{\AgdaFunction{𝒜⟦}}\AgdaSpace{}%
\AgdaInductiveConstructor{⊃ₒ}%
\>[8]\AgdaOperator{\AgdaFunction{⟧}}\AgdaSpace{}%
\AgdaSymbol{=}%
\>[13]\AgdaOperator{\AgdaPostulate{\AgdaUnderscore{}⟶\AgdaUnderscore{},\AgdaUnderscore{}}}\<%
\\
\>[0]\AgdaOperator{\AgdaFunction{𝒜⟦}}\AgdaSpace{}%
\AgdaInductiveConstructor{Y}%
\>[8]\AgdaOperator{\AgdaFunction{⟧}}\AgdaSpace{}%
\AgdaSymbol{=}%
\>[13]\AgdaPostulate{fix}\<%
\\
\>[0]\AgdaOperator{\AgdaFunction{𝒜⟦}}\AgdaSpace{}%
\AgdaInductiveConstructor{k}\AgdaSpace{}%
\AgdaBound{n}%
\>[8]\AgdaOperator{\AgdaFunction{⟧}}\AgdaSpace{}%
\AgdaSymbol{=}%
\>[13]\AgdaPostulate{η}\AgdaSpace{}%
\AgdaBound{n}\<%
\\
\>[0]\AgdaOperator{\AgdaFunction{𝒜⟦}}\AgdaSpace{}%
\AgdaInductiveConstructor{+1′}%
\>[8]\AgdaOperator{\AgdaFunction{⟧}}\AgdaSpace{}%
\AgdaSymbol{=}%
\>[13]\AgdaSymbol{(λ}\AgdaSpace{}%
\AgdaBound{n}\AgdaSpace{}%
\AgdaSymbol{→}\AgdaSpace{}%
\AgdaPostulate{η}\AgdaSpace{}%
\AgdaSymbol{(}\AgdaBound{n}\AgdaSpace{}%
\AgdaOperator{\AgdaPrimitive{+}}\AgdaSpace{}%
\AgdaNumber{1}\AgdaSymbol{))}\AgdaSpace{}%
\AgdaOperator{\AgdaPostulate{♯}}\<%
\\
\>[0]\AgdaOperator{\AgdaFunction{𝒜⟦}}\AgdaSpace{}%
\AgdaInductiveConstructor{-1′}%
\>[8]\AgdaOperator{\AgdaFunction{⟧}}\AgdaSpace{}%
\AgdaSymbol{=}%
\>[13]\AgdaSymbol{(λ}\AgdaSpace{}%
\AgdaBound{n}\AgdaSpace{}%
\AgdaSymbol{→}\AgdaSpace{}%
\AgdaOperator{\AgdaFunction{if}}\AgdaSpace{}%
\AgdaBound{n}\AgdaSpace{}%
\AgdaOperator{\AgdaPrimitive{==}}\AgdaSpace{}%
\AgdaNumber{0}\AgdaSpace{}%
\AgdaOperator{\AgdaFunction{then}}\AgdaSpace{}%
\AgdaPostulate{⊥}\AgdaSpace{}%
\AgdaOperator{\AgdaFunction{else}}\AgdaSpace{}%
\AgdaPostulate{η}\AgdaSpace{}%
\AgdaSymbol{(}\AgdaBound{n}\AgdaSpace{}%
\AgdaOperator{\AgdaPrimitive{-}}\AgdaSpace{}%
\AgdaNumber{1}\AgdaSymbol{))}\AgdaSpace{}%
\AgdaOperator{\AgdaPostulate{♯}}\<%
\\
\>[0]\AgdaOperator{\AgdaFunction{𝒜⟦}}\AgdaSpace{}%
\AgdaInductiveConstructor{Z}%
\>[8]\AgdaOperator{\AgdaFunction{⟧}}\AgdaSpace{}%
\AgdaSymbol{=}%
\>[13]\AgdaSymbol{(λ}\AgdaSpace{}%
\AgdaBound{n}\AgdaSpace{}%
\AgdaSymbol{→}\AgdaSpace{}%
\AgdaPostulate{η}\AgdaSpace{}%
\AgdaSymbol{(}\AgdaBound{n}\AgdaSpace{}%
\AgdaOperator{\AgdaPrimitive{==}}\AgdaSpace{}%
\AgdaNumber{0}\AgdaSymbol{))}\AgdaSpace{}%
\AgdaOperator{\AgdaPostulate{♯}}\<%
\end{code}
\clearpage
\begin{code}%
\>[0]\AgdaKeyword{module}\AgdaSpace{}%
\AgdaModule{ULC.Variables}\AgdaSpace{}%
\AgdaKeyword{where}\<%
\\
\>[0]\<%
\\
\>[0]\AgdaKeyword{open}\AgdaSpace{}%
\AgdaKeyword{import}\AgdaSpace{}%
\AgdaModule{Data.Bool}\AgdaSpace{}%
\AgdaKeyword{using}\AgdaSpace{}%
\AgdaSymbol{(}\AgdaDatatype{Bool}\AgdaSymbol{)}\<%
\\
\>[0]\AgdaKeyword{open}\AgdaSpace{}%
\AgdaKeyword{import}\AgdaSpace{}%
\AgdaModule{Data.Nat}%
\>[22]\AgdaKeyword{using}\AgdaSpace{}%
\AgdaSymbol{(}\AgdaDatatype{ℕ}\AgdaSymbol{;}\AgdaSpace{}%
\AgdaOperator{\AgdaPrimitive{\AgdaUnderscore{}≡ᵇ\AgdaUnderscore{}}}\AgdaSymbol{)}\<%
\\
%
\\[\AgdaEmptyExtraSkip]%
\>[0]\AgdaKeyword{data}\AgdaSpace{}%
\AgdaDatatype{Var}\AgdaSpace{}%
\AgdaSymbol{:}\AgdaSpace{}%
\AgdaPrimitive{Set}\AgdaSpace{}%
\AgdaKeyword{where}\<%
\\
\>[0][@{}l@{\AgdaIndent{0}}]%
\>[2]\AgdaInductiveConstructor{x}\AgdaSpace{}%
\AgdaSymbol{:}\AgdaSpace{}%
\AgdaDatatype{ℕ}\AgdaSpace{}%
\AgdaSymbol{→}\AgdaSpace{}%
\AgdaDatatype{Var}%
\>[15]\AgdaComment{--\ variables}\<%
\\
%
\\[\AgdaEmptyExtraSkip]%
\>[0]\AgdaKeyword{variable}\AgdaSpace{}%
\AgdaGeneralizable{v}\AgdaSpace{}%
\AgdaSymbol{:}\AgdaSpace{}%
\AgdaDatatype{Var}\<%
\\
%
\\[\AgdaEmptyExtraSkip]%
\>[0]\AgdaOperator{\AgdaFunction{\AgdaUnderscore{}==\AgdaUnderscore{}}}\AgdaSpace{}%
\AgdaSymbol{:}\AgdaSpace{}%
\AgdaDatatype{Var}\AgdaSpace{}%
\AgdaSymbol{→}\AgdaSpace{}%
\AgdaDatatype{Var}\AgdaSpace{}%
\AgdaSymbol{→}\AgdaSpace{}%
\AgdaDatatype{Bool}\<%
\\
\>[0]\AgdaInductiveConstructor{x}\AgdaSpace{}%
\AgdaBound{n}\AgdaSpace{}%
\AgdaOperator{\AgdaFunction{==}}\AgdaSpace{}%
\AgdaInductiveConstructor{x}\AgdaSpace{}%
\AgdaBound{n′}\AgdaSpace{}%
\AgdaSymbol{=}\AgdaSpace{}%
\AgdaSymbol{(}\AgdaBound{n}\AgdaSpace{}%
\AgdaOperator{\AgdaPrimitive{≡ᵇ}}\AgdaSpace{}%
\AgdaBound{n′}\AgdaSymbol{)}\<%
\end{code}
\clearpage
\begin{code}%
\>[0]\AgdaKeyword{module}\AgdaSpace{}%
\AgdaModule{ULC.Environments}\AgdaSpace{}%
\AgdaKeyword{where}\<%
\\
%
\\[\AgdaEmptyExtraSkip]%
\>[0]\AgdaKeyword{open}\AgdaSpace{}%
\AgdaKeyword{import}\AgdaSpace{}%
\AgdaModule{ULC.Variables}\<%
\\
\>[0]\AgdaKeyword{open}\AgdaSpace{}%
\AgdaKeyword{import}\AgdaSpace{}%
\AgdaModule{ULC.Domains}\<%
\\
\>[0]\AgdaKeyword{open}\AgdaSpace{}%
\AgdaKeyword{import}\AgdaSpace{}%
\AgdaModule{Data.Bool}\AgdaSpace{}%
\AgdaKeyword{using}\AgdaSpace{}%
\AgdaSymbol{(}\AgdaOperator{\AgdaFunction{if\AgdaUnderscore{}then\AgdaUnderscore{}else\AgdaUnderscore{}}}\AgdaSymbol{)}\<%
\\
%
\\[\AgdaEmptyExtraSkip]%
\>[0]\AgdaFunction{Env}\AgdaSpace{}%
\AgdaSymbol{:}\AgdaSpace{}%
\AgdaFunction{Domain}\<%
\\
\>[0]\AgdaFunction{Env}\AgdaSpace{}%
\AgdaSymbol{=}\AgdaSpace{}%
\AgdaDatatype{Var}\AgdaSpace{}%
\AgdaSymbol{→}\AgdaSpace{}%
\AgdaPostulate{D∞}\<%
\\
\>[0]\AgdaComment{--\ the\ initial\ environment\ for\ a\ closed\ term\ is\ λ\ v\ →\ ⊥}\<%
\\
%
\\[\AgdaEmptyExtraSkip]%
\>[0]\AgdaKeyword{variable}\AgdaSpace{}%
\AgdaGeneralizable{ρ}\AgdaSpace{}%
\AgdaSymbol{:}\AgdaSpace{}%
\AgdaFunction{Env}\<%
\\
%
\\[\AgdaEmptyExtraSkip]%
\>[0]\AgdaOperator{\AgdaFunction{\AgdaUnderscore{}[\AgdaUnderscore{}/\AgdaUnderscore{}]}}\AgdaSpace{}%
\AgdaSymbol{:}\AgdaSpace{}%
\AgdaFunction{Env}\AgdaSpace{}%
\AgdaSymbol{→}\AgdaSpace{}%
\AgdaPostulate{D∞}\AgdaSpace{}%
\AgdaSymbol{→}\AgdaSpace{}%
\AgdaDatatype{Var}\AgdaSpace{}%
\AgdaSymbol{→}\AgdaSpace{}%
\AgdaFunction{Env}\<%
\\
\>[0]\AgdaBound{ρ}\AgdaSpace{}%
\AgdaOperator{\AgdaFunction{[}}\AgdaSpace{}%
\AgdaBound{d}\AgdaSpace{}%
\AgdaOperator{\AgdaFunction{/}}\AgdaSpace{}%
\AgdaBound{v}\AgdaSpace{}%
\AgdaOperator{\AgdaFunction{]}}\AgdaSpace{}%
\AgdaSymbol{=}\AgdaSpace{}%
\AgdaSymbol{λ}\AgdaSpace{}%
\AgdaBound{v′}\AgdaSpace{}%
\AgdaSymbol{→}\AgdaSpace{}%
\AgdaOperator{\AgdaFunction{if}}\AgdaSpace{}%
\AgdaBound{v}\AgdaSpace{}%
\AgdaOperator{\AgdaFunction{==}}\AgdaSpace{}%
\AgdaBound{v′}\AgdaSpace{}%
\AgdaOperator{\AgdaFunction{then}}\AgdaSpace{}%
\AgdaBound{d}\AgdaSpace{}%
\AgdaOperator{\AgdaFunction{else}}\AgdaSpace{}%
\AgdaBound{ρ}\AgdaSpace{}%
\AgdaBound{v′}\<%
\end{code}
\clearpage
\begin{code}%
\>[0]\AgdaKeyword{module}\AgdaSpace{}%
\AgdaModule{PCF.Terms}\AgdaSpace{}%
\AgdaKeyword{where}\<%
\\
%
\\[\AgdaEmptyExtraSkip]%
\>[0]\AgdaKeyword{open}\AgdaSpace{}%
\AgdaKeyword{import}\AgdaSpace{}%
\AgdaModule{PCF.Types}\<%
\\
\>[0][@{}l@{\AgdaIndent{0}}]%
\>[2]\AgdaKeyword{using}\AgdaSpace{}%
\AgdaSymbol{(}\AgdaDatatype{Types}\AgdaSymbol{;}\AgdaSpace{}%
\AgdaOperator{\AgdaInductiveConstructor{\AgdaUnderscore{}⇒\AgdaUnderscore{}}}\AgdaSymbol{;}\AgdaSpace{}%
\AgdaGeneralizable{σ}\AgdaSymbol{;}\AgdaSpace{}%
\AgdaFunction{𝒟}\AgdaSymbol{)}\<%
\\
\>[0]\AgdaKeyword{open}\AgdaSpace{}%
\AgdaKeyword{import}\AgdaSpace{}%
\AgdaModule{PCF.Constants}\<%
\\
\>[0][@{}l@{\AgdaIndent{0}}]%
\>[2]\AgdaKeyword{using}\AgdaSpace{}%
\AgdaSymbol{(}\AgdaDatatype{ℒ}\AgdaSymbol{;}\AgdaSpace{}%
\AgdaOperator{\AgdaFunction{𝒜⟦\AgdaUnderscore{}⟧}}\AgdaSymbol{;}\AgdaSpace{}%
\AgdaGeneralizable{c}\AgdaSymbol{)}\<%
\\
\>[0]\AgdaKeyword{open}\AgdaSpace{}%
\AgdaKeyword{import}\AgdaSpace{}%
\AgdaModule{PCF.Variables}\<%
\\
\>[0][@{}l@{\AgdaIndent{0}}]%
\>[2]\AgdaKeyword{using}\AgdaSpace{}%
\AgdaSymbol{(}\AgdaDatatype{𝒱}\AgdaSymbol{;}\AgdaSpace{}%
\AgdaFunction{Env}\AgdaSymbol{;}\AgdaSpace{}%
\AgdaOperator{\AgdaFunction{\AgdaUnderscore{}⟦\AgdaUnderscore{}⟧}}\AgdaSymbol{)}\<%
\\
\>[0]\AgdaKeyword{open}\AgdaSpace{}%
\AgdaKeyword{import}\AgdaSpace{}%
\AgdaModule{PCF.Environments}\<%
\\
\>[0][@{}l@{\AgdaIndent{0}}]%
\>[2]\AgdaKeyword{using}\AgdaSpace{}%
\AgdaSymbol{(}\AgdaOperator{\AgdaFunction{\AgdaUnderscore{}[\AgdaUnderscore{}/\AgdaUnderscore{}]}}\AgdaSymbol{)}\<%
\\
%
\\[\AgdaEmptyExtraSkip]%
\>[0]\AgdaComment{--\ Syntax}\<%
\\
%
\\[\AgdaEmptyExtraSkip]%
\>[0]\AgdaKeyword{data}\AgdaSpace{}%
\AgdaDatatype{Terms}\AgdaSpace{}%
\AgdaSymbol{:}\AgdaSpace{}%
\AgdaDatatype{Types}\AgdaSpace{}%
\AgdaSymbol{→}\AgdaSpace{}%
\AgdaPrimitive{Set}\AgdaSpace{}%
\AgdaKeyword{where}\<%
\\
\>[0][@{}l@{\AgdaIndent{0}}]%
\>[2]\AgdaInductiveConstructor{𝑉}%
\>[8]\AgdaSymbol{:}\AgdaSpace{}%
\AgdaSymbol{\{}\AgdaBound{σ}%
\>[15]\AgdaSymbol{:}\AgdaSpace{}%
\AgdaDatatype{Types}\AgdaSymbol{\}}\AgdaSpace{}%
\AgdaSymbol{→}\AgdaSpace{}%
\AgdaDatatype{𝒱}\AgdaSpace{}%
\AgdaBound{σ}\AgdaSpace{}%
\AgdaSymbol{→}\AgdaSpace{}%
\AgdaDatatype{Terms}\AgdaSpace{}%
\AgdaBound{σ}%
\>[61]\AgdaComment{--\ variables}\<%
\\
%
\>[2]\AgdaInductiveConstructor{𝐿}%
\>[8]\AgdaSymbol{:}\AgdaSpace{}%
\AgdaSymbol{\{}\AgdaBound{σ}%
\>[15]\AgdaSymbol{:}\AgdaSpace{}%
\AgdaDatatype{Types}\AgdaSymbol{\}}\AgdaSpace{}%
\AgdaSymbol{→}\AgdaSpace{}%
\AgdaDatatype{ℒ}\AgdaSpace{}%
\AgdaBound{σ}\AgdaSpace{}%
\AgdaSymbol{→}\AgdaSpace{}%
\AgdaDatatype{Terms}\AgdaSpace{}%
\AgdaBound{σ}%
\>[61]\AgdaComment{--\ constants}\<%
\\
%
\>[2]\AgdaOperator{\AgdaInductiveConstructor{\AgdaUnderscore{}˜\AgdaUnderscore{}}}%
\>[8]\AgdaSymbol{:}\AgdaSpace{}%
\AgdaSymbol{\{}\AgdaBound{σ}\AgdaSpace{}%
\AgdaBound{τ}\AgdaSpace{}%
\AgdaSymbol{:}\AgdaSpace{}%
\AgdaDatatype{Types}\AgdaSymbol{\}}\AgdaSpace{}%
\AgdaSymbol{→}\AgdaSpace{}%
\AgdaDatatype{Terms}\AgdaSpace{}%
\AgdaSymbol{(}\AgdaBound{σ}\AgdaSpace{}%
\AgdaOperator{\AgdaInductiveConstructor{⇒}}\AgdaSpace{}%
\AgdaBound{τ}\AgdaSymbol{)}\AgdaSpace{}%
\AgdaSymbol{→}\AgdaSpace{}%
\AgdaDatatype{Terms}\AgdaSpace{}%
\AgdaBound{σ}\AgdaSpace{}%
\AgdaSymbol{→}\AgdaSpace{}%
\AgdaDatatype{Terms}\AgdaSpace{}%
\AgdaBound{τ}%
\>[61]\AgdaComment{--\ application}\<%
\\
%
\>[2]\AgdaOperator{\AgdaInductiveConstructor{ƛ\AgdaUnderscore{}˜\AgdaUnderscore{}}}%
\>[8]\AgdaSymbol{:}\AgdaSpace{}%
\AgdaSymbol{\{}\AgdaBound{σ}\AgdaSpace{}%
\AgdaBound{τ}\AgdaSpace{}%
\AgdaSymbol{:}\AgdaSpace{}%
\AgdaDatatype{Types}\AgdaSymbol{\}}\AgdaSpace{}%
\AgdaSymbol{→}\AgdaSpace{}%
\AgdaDatatype{𝒱}\AgdaSpace{}%
\AgdaBound{σ}\AgdaSpace{}%
\AgdaSymbol{→}\AgdaSpace{}%
\AgdaDatatype{Terms}\AgdaSpace{}%
\AgdaBound{τ}\AgdaSpace{}%
\AgdaSymbol{→}\AgdaSpace{}%
\AgdaDatatype{Terms}\AgdaSpace{}%
\AgdaSymbol{(}\AgdaBound{σ}\AgdaSpace{}%
\AgdaOperator{\AgdaInductiveConstructor{⇒}}\AgdaSpace{}%
\AgdaBound{τ}\AgdaSymbol{)}%
\>[61]\AgdaComment{--\ λ-abstraction}\<%
\\
%
\\[\AgdaEmptyExtraSkip]%
\>[0]\AgdaKeyword{variable}\AgdaSpace{}%
\AgdaGeneralizable{M}\AgdaSpace{}%
\AgdaGeneralizable{N}\AgdaSpace{}%
\AgdaSymbol{:}\AgdaSpace{}%
\AgdaDatatype{Terms}\AgdaSpace{}%
\AgdaGeneralizable{σ}\<%
\\
\>[0]\AgdaKeyword{infixl}\AgdaSpace{}%
\AgdaNumber{20}\AgdaSpace{}%
\AgdaOperator{\AgdaInductiveConstructor{\AgdaUnderscore{}˜\AgdaUnderscore{}}}\<%
\\
%
\\[\AgdaEmptyExtraSkip]%
\>[0]\AgdaComment{--\ Semantics}\<%
\\
%
\\[\AgdaEmptyExtraSkip]%
\>[0]\AgdaOperator{\AgdaFunction{𝒜′⟦\AgdaUnderscore{}⟧}}\AgdaSpace{}%
\AgdaSymbol{:}\AgdaSpace{}%
\AgdaDatatype{Terms}\AgdaSpace{}%
\AgdaGeneralizable{σ}\AgdaSpace{}%
\AgdaSymbol{→}\AgdaSpace{}%
\AgdaFunction{Env}\AgdaSpace{}%
\AgdaSymbol{→}\AgdaSpace{}%
\AgdaFunction{𝒟}\AgdaSpace{}%
\AgdaGeneralizable{σ}\<%
\\
%
\\[\AgdaEmptyExtraSkip]%
\>[0]\AgdaOperator{\AgdaFunction{𝒜′⟦}}\AgdaSpace{}%
\AgdaInductiveConstructor{𝑉}\AgdaSpace{}%
\AgdaBound{α}%
\>[13]\AgdaOperator{\AgdaFunction{⟧}}\AgdaSpace{}%
\AgdaBound{ρ}\AgdaSpace{}%
\AgdaSymbol{=}\AgdaSpace{}%
\AgdaBound{ρ}\AgdaSpace{}%
\AgdaOperator{\AgdaFunction{⟦}}\AgdaSpace{}%
\AgdaBound{α}\AgdaSpace{}%
\AgdaOperator{\AgdaFunction{⟧}}\<%
\\
\>[0]\AgdaOperator{\AgdaFunction{𝒜′⟦}}\AgdaSpace{}%
\AgdaInductiveConstructor{𝐿}\AgdaSpace{}%
\AgdaBound{c}%
\>[13]\AgdaOperator{\AgdaFunction{⟧}}\AgdaSpace{}%
\AgdaBound{ρ}\AgdaSpace{}%
\AgdaSymbol{=}\AgdaSpace{}%
\AgdaOperator{\AgdaFunction{𝒜⟦}}\AgdaSpace{}%
\AgdaBound{c}\AgdaSpace{}%
\AgdaOperator{\AgdaFunction{⟧}}\<%
\\
\>[0]\AgdaOperator{\AgdaFunction{𝒜′⟦}}\AgdaSpace{}%
\AgdaBound{M}\AgdaSpace{}%
\AgdaOperator{\AgdaInductiveConstructor{˜}}\AgdaSpace{}%
\AgdaBound{N}%
\>[13]\AgdaOperator{\AgdaFunction{⟧}}\AgdaSpace{}%
\AgdaBound{ρ}\AgdaSpace{}%
\AgdaSymbol{=}\AgdaSpace{}%
\AgdaOperator{\AgdaFunction{𝒜′⟦}}\AgdaSpace{}%
\AgdaBound{M}\AgdaSpace{}%
\AgdaOperator{\AgdaFunction{⟧}}\AgdaSpace{}%
\AgdaBound{ρ}\AgdaSpace{}%
\AgdaSymbol{(}\AgdaOperator{\AgdaFunction{𝒜′⟦}}\AgdaSpace{}%
\AgdaBound{N}\AgdaSpace{}%
\AgdaOperator{\AgdaFunction{⟧}}\AgdaSpace{}%
\AgdaBound{ρ}\AgdaSymbol{)}\<%
\\
\>[0]\AgdaOperator{\AgdaFunction{𝒜′⟦}}\AgdaSpace{}%
\AgdaOperator{\AgdaInductiveConstructor{ƛ}}\AgdaSpace{}%
\AgdaBound{α}\AgdaSpace{}%
\AgdaOperator{\AgdaInductiveConstructor{˜}}\AgdaSpace{}%
\AgdaBound{M}%
\>[13]\AgdaOperator{\AgdaFunction{⟧}}\AgdaSpace{}%
\AgdaBound{ρ}\AgdaSpace{}%
\AgdaSymbol{=}\AgdaSpace{}%
\AgdaSymbol{λ}\AgdaSpace{}%
\AgdaBound{x}\AgdaSpace{}%
\AgdaSymbol{→}\AgdaSpace{}%
\AgdaOperator{\AgdaFunction{𝒜′⟦}}\AgdaSpace{}%
\AgdaBound{M}\AgdaSpace{}%
\AgdaOperator{\AgdaFunction{⟧}}\AgdaSpace{}%
\AgdaSymbol{(}\AgdaBound{ρ}\AgdaSpace{}%
\AgdaOperator{\AgdaFunction{[}}\AgdaSpace{}%
\AgdaBound{x}\AgdaSpace{}%
\AgdaOperator{\AgdaFunction{/}}\AgdaSpace{}%
\AgdaBound{α}\AgdaSpace{}%
\AgdaOperator{\AgdaFunction{]}}\AgdaSymbol{)}\<%
\end{code}
\clearpage
\begin{code}%
\>[0]\AgdaSymbol{\{-\#}\AgdaSpace{}%
\AgdaKeyword{OPTIONS}\AgdaSpace{}%
\AgdaPragma{--rewriting}\AgdaSpace{}%
\AgdaPragma{--confluence-check}\AgdaSpace{}%
\AgdaSymbol{\#-\}}\<%
\\
\>[0]\AgdaKeyword{open}\AgdaSpace{}%
\AgdaKeyword{import}\AgdaSpace{}%
\AgdaModule{Agda.Builtin.Equality}\<%
\\
\>[0]\AgdaKeyword{open}\AgdaSpace{}%
\AgdaKeyword{import}\AgdaSpace{}%
\AgdaModule{Agda.Builtin.Equality.Rewrite}\<%
\\
%
\\[\AgdaEmptyExtraSkip]%
\>[0]\AgdaKeyword{module}\AgdaSpace{}%
\AgdaModule{PCF.Checks}\AgdaSpace{}%
\AgdaKeyword{where}\<%
\\
%
\\[\AgdaEmptyExtraSkip]%
\>[0]\AgdaKeyword{open}\AgdaSpace{}%
\AgdaKeyword{import}\AgdaSpace{}%
\AgdaModule{Data.Bool.Base}\<%
\\
\>[0]\AgdaKeyword{open}\AgdaSpace{}%
\AgdaKeyword{import}\AgdaSpace{}%
\AgdaModule{Agda.Builtin.Nat}\<%
\\
\>[0]\AgdaKeyword{open}\AgdaSpace{}%
\AgdaKeyword{import}\AgdaSpace{}%
\AgdaModule{Relation.Binary.PropositionalEquality.Core}\<%
\\
\>[0][@{}l@{\AgdaIndent{0}}]%
\>[2]\AgdaKeyword{using}\AgdaSpace{}%
\AgdaSymbol{(}\AgdaOperator{\AgdaDatatype{\AgdaUnderscore{}≡\AgdaUnderscore{}}}\AgdaSymbol{;}\AgdaSpace{}%
\AgdaInductiveConstructor{refl}\AgdaSymbol{)}\<%
\\
%
\\[\AgdaEmptyExtraSkip]%
\>[0]\AgdaKeyword{open}\AgdaSpace{}%
\AgdaKeyword{import}\AgdaSpace{}%
\AgdaModule{PCF.Domain-Notation}\<%
\\
\>[0]\AgdaKeyword{open}\AgdaSpace{}%
\AgdaKeyword{import}\AgdaSpace{}%
\AgdaModule{PCF.Types}\<%
\\
\>[0]\AgdaKeyword{open}\AgdaSpace{}%
\AgdaKeyword{import}\AgdaSpace{}%
\AgdaModule{PCF.Constants}\<%
\\
\>[0]\AgdaKeyword{open}\AgdaSpace{}%
\AgdaKeyword{import}\AgdaSpace{}%
\AgdaModule{PCF.Variables}\<%
\\
\>[0]\AgdaKeyword{open}\AgdaSpace{}%
\AgdaKeyword{import}\AgdaSpace{}%
\AgdaModule{PCF.Environments}\<%
\\
\>[0]\AgdaKeyword{open}\AgdaSpace{}%
\AgdaKeyword{import}\AgdaSpace{}%
\AgdaModule{PCF.Terms}\<%
\\
%
\\[\AgdaEmptyExtraSkip]%
\>[0]\AgdaKeyword{postulate}\<%
\\
\>[0][@{}l@{\AgdaIndent{0}}]%
\>[2]\AgdaSymbol{\{-\#}\AgdaSpace{}%
\AgdaKeyword{REWRITE}\AgdaSpace{}%
\AgdaPostulate{fix-app}\AgdaSpace{}%
\AgdaPostulate{elim-♯-η}\AgdaSpace{}%
\AgdaPostulate{elim-♯-⊥}\AgdaSpace{}%
\AgdaPostulate{true-cond}\AgdaSpace{}%
\AgdaPostulate{false-cond}\AgdaSpace{}%
\AgdaSymbol{\#-\}}\<%
\\
%
\\[\AgdaEmptyExtraSkip]%
\>[0]\AgdaComment{--\ Constants}\<%
\\
\>[0]\AgdaKeyword{pattern}\AgdaSpace{}%
\AgdaInductiveConstructor{𝑁}\AgdaSpace{}%
\AgdaBound{n}%
\>[15]\AgdaSymbol{=}\AgdaSpace{}%
\AgdaInductiveConstructor{𝐿}\AgdaSpace{}%
\AgdaSymbol{(}\AgdaInductiveConstructor{k}\AgdaSpace{}%
\AgdaBound{n}\AgdaSymbol{)}\<%
\\
\>[0]\AgdaKeyword{pattern}\AgdaSpace{}%
\AgdaInductiveConstructor{succ}%
\>[15]\AgdaSymbol{=}\AgdaSpace{}%
\AgdaInductiveConstructor{𝐿}\AgdaSpace{}%
\AgdaInductiveConstructor{+1′}\<%
\\
\>[0]\AgdaKeyword{pattern}\AgdaSpace{}%
\AgdaInductiveConstructor{pred⊥}%
\>[15]\AgdaSymbol{=}\AgdaSpace{}%
\AgdaInductiveConstructor{𝐿}\AgdaSpace{}%
\AgdaInductiveConstructor{-1′}\<%
\\
\>[0]\AgdaKeyword{pattern}\AgdaSpace{}%
\AgdaInductiveConstructor{if}%
\>[15]\AgdaSymbol{=}\AgdaSpace{}%
\AgdaInductiveConstructor{𝐿}\AgdaSpace{}%
\AgdaInductiveConstructor{⊃ᵢ}\<%
\\
\>[0]\AgdaKeyword{pattern}\AgdaSpace{}%
\AgdaInductiveConstructor{𝑌}%
\>[15]\AgdaSymbol{=}\AgdaSpace{}%
\AgdaInductiveConstructor{𝐿}\AgdaSpace{}%
\AgdaInductiveConstructor{Y}\<%
\\
\>[0]\AgdaKeyword{pattern}\AgdaSpace{}%
\AgdaInductiveConstructor{𝑍}%
\>[14]\AgdaSymbol{=}\AgdaSpace{}%
\AgdaInductiveConstructor{𝐿}\AgdaSpace{}%
\AgdaInductiveConstructor{Z}\<%
\\
%
\\[\AgdaEmptyExtraSkip]%
\>[0]\AgdaComment{--\ Variables}\<%
\\
\>[0]\AgdaFunction{f}%
\>[3]\AgdaSymbol{=}\AgdaSpace{}%
\AgdaInductiveConstructor{var}\AgdaSpace{}%
\AgdaNumber{0}\AgdaSpace{}%
\AgdaInductiveConstructor{ι}\<%
\\
\>[0]\AgdaFunction{g}%
\>[3]\AgdaSymbol{=}\AgdaSpace{}%
\AgdaInductiveConstructor{var}\AgdaSpace{}%
\AgdaNumber{1}\AgdaSpace{}%
\AgdaSymbol{(}\AgdaInductiveConstructor{ι}\AgdaSpace{}%
\AgdaOperator{\AgdaInductiveConstructor{⇒}}\AgdaSpace{}%
\AgdaInductiveConstructor{ι}\AgdaSymbol{)}\<%
\\
\>[0]\AgdaFunction{h}%
\>[3]\AgdaSymbol{=}\AgdaSpace{}%
\AgdaInductiveConstructor{var}\AgdaSpace{}%
\AgdaNumber{2}\AgdaSpace{}%
\AgdaSymbol{(}\AgdaInductiveConstructor{ι}\AgdaSpace{}%
\AgdaOperator{\AgdaInductiveConstructor{⇒}}\AgdaSpace{}%
\AgdaInductiveConstructor{ι}\AgdaSpace{}%
\AgdaOperator{\AgdaInductiveConstructor{⇒}}\AgdaSpace{}%
\AgdaInductiveConstructor{ι}\AgdaSymbol{)}\<%
\\
\>[0]\AgdaFunction{a}%
\>[3]\AgdaSymbol{=}\AgdaSpace{}%
\AgdaInductiveConstructor{var}\AgdaSpace{}%
\AgdaNumber{3}\AgdaSpace{}%
\AgdaInductiveConstructor{ι}\<%
\\
\>[0]\AgdaFunction{b}%
\>[3]\AgdaSymbol{=}\AgdaSpace{}%
\AgdaInductiveConstructor{var}\AgdaSpace{}%
\AgdaNumber{4}\AgdaSpace{}%
\AgdaInductiveConstructor{ι}\<%
\\
%
\\[\AgdaEmptyExtraSkip]%
\>[0]\AgdaComment{--\ Arithmetic}\<%
\\
\>[0]\AgdaFunction{check-41+1}\AgdaSpace{}%
\AgdaSymbol{:}\AgdaSpace{}%
\AgdaOperator{\AgdaFunction{𝒜′⟦}}\AgdaSpace{}%
\AgdaInductiveConstructor{succ}\AgdaSpace{}%
\AgdaOperator{\AgdaInductiveConstructor{˜}}\AgdaSpace{}%
\AgdaInductiveConstructor{𝑁}\AgdaSpace{}%
\AgdaNumber{41}\AgdaSpace{}%
\AgdaOperator{\AgdaFunction{⟧}}\AgdaSpace{}%
\AgdaFunction{ρ⊥}\AgdaSpace{}%
\AgdaOperator{\AgdaDatatype{≡}}\AgdaSpace{}%
\AgdaPostulate{η}\AgdaSpace{}%
\AgdaNumber{42}\<%
\\
\>[0]\AgdaFunction{check-41+1}\AgdaSpace{}%
\AgdaSymbol{=}\AgdaSpace{}%
\AgdaInductiveConstructor{refl}\<%
\\
%
\\[\AgdaEmptyExtraSkip]%
\>[0]\AgdaFunction{check-43-1}\AgdaSpace{}%
\AgdaSymbol{:}\AgdaSpace{}%
\AgdaOperator{\AgdaFunction{𝒜′⟦}}\AgdaSpace{}%
\AgdaInductiveConstructor{pred⊥}\AgdaSpace{}%
\AgdaOperator{\AgdaInductiveConstructor{˜}}\AgdaSpace{}%
\AgdaInductiveConstructor{𝑁}\AgdaSpace{}%
\AgdaNumber{43}\AgdaSpace{}%
\AgdaOperator{\AgdaFunction{⟧}}\AgdaSpace{}%
\AgdaFunction{ρ⊥}\AgdaSpace{}%
\AgdaOperator{\AgdaDatatype{≡}}\AgdaSpace{}%
\AgdaPostulate{η}\AgdaSpace{}%
\AgdaNumber{42}\<%
\\
\>[0]\AgdaFunction{check-43-1}\AgdaSpace{}%
\AgdaSymbol{=}\AgdaSpace{}%
\AgdaInductiveConstructor{refl}\<%
\\
%
\\[\AgdaEmptyExtraSkip]%
\>[0]\AgdaComment{--\ Binding}\<%
\\
\>[0]\AgdaFunction{check-id}\AgdaSpace{}%
\AgdaSymbol{:}\AgdaSpace{}%
\AgdaOperator{\AgdaFunction{𝒜′⟦}}\AgdaSpace{}%
\AgdaSymbol{(}\AgdaOperator{\AgdaInductiveConstructor{ƛ}}\AgdaSpace{}%
\AgdaFunction{a}\AgdaSpace{}%
\AgdaOperator{\AgdaInductiveConstructor{˜}}\AgdaSpace{}%
\AgdaInductiveConstructor{𝑉}\AgdaSpace{}%
\AgdaFunction{a}\AgdaSymbol{)}\AgdaSpace{}%
\AgdaOperator{\AgdaInductiveConstructor{˜}}\AgdaSpace{}%
\AgdaInductiveConstructor{𝑁}\AgdaSpace{}%
\AgdaNumber{42}\AgdaSpace{}%
\AgdaOperator{\AgdaFunction{⟧}}\AgdaSpace{}%
\AgdaFunction{ρ⊥}\AgdaSpace{}%
\AgdaOperator{\AgdaDatatype{≡}}\AgdaSpace{}%
\AgdaPostulate{η}\AgdaSpace{}%
\AgdaNumber{42}\<%
\\
\>[0]\AgdaFunction{check-id}\AgdaSpace{}%
\AgdaSymbol{=}\AgdaSpace{}%
\AgdaInductiveConstructor{refl}\<%
\\
%
\\[\AgdaEmptyExtraSkip]%
\>[0]\AgdaFunction{check-k}\AgdaSpace{}%
\AgdaSymbol{:}\AgdaSpace{}%
\AgdaOperator{\AgdaFunction{𝒜′⟦}}\AgdaSpace{}%
\AgdaSymbol{(}\AgdaOperator{\AgdaInductiveConstructor{ƛ}}\AgdaSpace{}%
\AgdaFunction{a}\AgdaSpace{}%
\AgdaOperator{\AgdaInductiveConstructor{˜}}\AgdaSpace{}%
\AgdaOperator{\AgdaInductiveConstructor{ƛ}}\AgdaSpace{}%
\AgdaFunction{b}\AgdaSpace{}%
\AgdaOperator{\AgdaInductiveConstructor{˜}}\AgdaSpace{}%
\AgdaInductiveConstructor{𝑉}\AgdaSpace{}%
\AgdaFunction{a}\AgdaSymbol{)}\AgdaSpace{}%
\AgdaOperator{\AgdaInductiveConstructor{˜}}\AgdaSpace{}%
\AgdaInductiveConstructor{𝑁}\AgdaSpace{}%
\AgdaNumber{42}\AgdaSpace{}%
\AgdaOperator{\AgdaInductiveConstructor{˜}}\AgdaSpace{}%
\AgdaInductiveConstructor{𝑁}\AgdaSpace{}%
\AgdaNumber{41}\AgdaSpace{}%
\AgdaOperator{\AgdaFunction{⟧}}\AgdaSpace{}%
\AgdaFunction{ρ⊥}\AgdaSpace{}%
\AgdaOperator{\AgdaDatatype{≡}}\AgdaSpace{}%
\AgdaPostulate{η}\AgdaSpace{}%
\AgdaNumber{42}\<%
\\
\>[0]\AgdaFunction{check-k}\AgdaSpace{}%
\AgdaSymbol{=}\AgdaSpace{}%
\AgdaInductiveConstructor{refl}\<%
\\
%
\\[\AgdaEmptyExtraSkip]%
\>[0]\AgdaFunction{check-ki}\AgdaSpace{}%
\AgdaSymbol{:}\AgdaSpace{}%
\AgdaOperator{\AgdaFunction{𝒜′⟦}}\AgdaSpace{}%
\AgdaSymbol{(}\AgdaOperator{\AgdaInductiveConstructor{ƛ}}\AgdaSpace{}%
\AgdaFunction{a}\AgdaSpace{}%
\AgdaOperator{\AgdaInductiveConstructor{˜}}\AgdaSpace{}%
\AgdaOperator{\AgdaInductiveConstructor{ƛ}}\AgdaSpace{}%
\AgdaFunction{b}\AgdaSpace{}%
\AgdaOperator{\AgdaInductiveConstructor{˜}}\AgdaSpace{}%
\AgdaInductiveConstructor{𝑉}\AgdaSpace{}%
\AgdaFunction{b}\AgdaSymbol{)}\AgdaSpace{}%
\AgdaOperator{\AgdaInductiveConstructor{˜}}\AgdaSpace{}%
\AgdaInductiveConstructor{𝑁}\AgdaSpace{}%
\AgdaNumber{41}\AgdaSpace{}%
\AgdaOperator{\AgdaInductiveConstructor{˜}}\AgdaSpace{}%
\AgdaInductiveConstructor{𝑁}\AgdaSpace{}%
\AgdaNumber{42}\AgdaSpace{}%
\AgdaOperator{\AgdaFunction{⟧}}\AgdaSpace{}%
\AgdaFunction{ρ⊥}\AgdaSpace{}%
\AgdaOperator{\AgdaDatatype{≡}}\AgdaSpace{}%
\AgdaPostulate{η}\AgdaSpace{}%
\AgdaNumber{42}\<%
\\
\>[0]\AgdaFunction{check-ki}\AgdaSpace{}%
\AgdaSymbol{=}\AgdaSpace{}%
\AgdaInductiveConstructor{refl}\<%
\\
\>[0]\<%
\end{code}
\clearpage
\begin{code}%
\>[0]\AgdaFunction{check-suc-41}\AgdaSpace{}%
\AgdaSymbol{:}\AgdaSpace{}%
\AgdaOperator{\AgdaFunction{𝒜′⟦}}\AgdaSpace{}%
\AgdaSymbol{(}\AgdaOperator{\AgdaInductiveConstructor{ƛ}}\AgdaSpace{}%
\AgdaFunction{a}\AgdaSpace{}%
\AgdaOperator{\AgdaInductiveConstructor{˜}}\AgdaSpace{}%
\AgdaSymbol{(}\AgdaInductiveConstructor{succ}\AgdaSpace{}%
\AgdaOperator{\AgdaInductiveConstructor{˜}}\AgdaSpace{}%
\AgdaInductiveConstructor{𝑉}\AgdaSpace{}%
\AgdaFunction{a}\AgdaSpace{}%
\AgdaSymbol{))}\AgdaSpace{}%
\AgdaOperator{\AgdaInductiveConstructor{˜}}\AgdaSpace{}%
\AgdaInductiveConstructor{𝑁}\AgdaSpace{}%
\AgdaNumber{41}\AgdaSpace{}%
\AgdaOperator{\AgdaFunction{⟧}}\AgdaSpace{}%
\AgdaFunction{ρ⊥}\AgdaSpace{}%
\AgdaOperator{\AgdaDatatype{≡}}\AgdaSpace{}%
\AgdaPostulate{η}\AgdaSpace{}%
\AgdaNumber{42}\<%
\\
\>[0]\AgdaFunction{check-suc-41}\AgdaSpace{}%
\AgdaSymbol{=}\AgdaSpace{}%
\AgdaInductiveConstructor{refl}\<%
\\
%
\\[\AgdaEmptyExtraSkip]%
\>[0]\AgdaFunction{check-pred-42}\AgdaSpace{}%
\AgdaSymbol{:}\AgdaSpace{}%
\AgdaOperator{\AgdaFunction{𝒜′⟦}}\AgdaSpace{}%
\AgdaSymbol{(}\AgdaOperator{\AgdaInductiveConstructor{ƛ}}\AgdaSpace{}%
\AgdaFunction{a}\AgdaSpace{}%
\AgdaOperator{\AgdaInductiveConstructor{˜}}\AgdaSpace{}%
\AgdaSymbol{(}\AgdaInductiveConstructor{pred⊥}\AgdaSpace{}%
\AgdaOperator{\AgdaInductiveConstructor{˜}}\AgdaSpace{}%
\AgdaInductiveConstructor{𝑉}\AgdaSpace{}%
\AgdaFunction{a}\AgdaSymbol{))}\AgdaSpace{}%
\AgdaOperator{\AgdaInductiveConstructor{˜}}\AgdaSpace{}%
\AgdaInductiveConstructor{𝑁}\AgdaSpace{}%
\AgdaNumber{43}\AgdaSpace{}%
\AgdaOperator{\AgdaFunction{⟧}}\AgdaSpace{}%
\AgdaFunction{ρ⊥}\AgdaSpace{}%
\AgdaOperator{\AgdaDatatype{≡}}\AgdaSpace{}%
\AgdaPostulate{η}\AgdaSpace{}%
\AgdaNumber{42}\<%
\\
\>[0]\AgdaFunction{check-pred-42}\AgdaSpace{}%
\AgdaSymbol{=}\AgdaSpace{}%
\AgdaInductiveConstructor{refl}\<%
\\
%
\\[\AgdaEmptyExtraSkip]%
\>[0]\AgdaFunction{check-if-zero}\AgdaSpace{}%
\AgdaSymbol{:}\AgdaSpace{}%
\AgdaOperator{\AgdaFunction{𝒜′⟦}}\AgdaSpace{}%
\AgdaInductiveConstructor{if}\AgdaSpace{}%
\AgdaOperator{\AgdaInductiveConstructor{˜}}\AgdaSpace{}%
\AgdaSymbol{(}\AgdaInductiveConstructor{𝑍}\AgdaSpace{}%
\AgdaOperator{\AgdaInductiveConstructor{˜}}\AgdaSpace{}%
\AgdaInductiveConstructor{𝑁}\AgdaSpace{}%
\AgdaNumber{0}\AgdaSymbol{)}\AgdaSpace{}%
\AgdaOperator{\AgdaInductiveConstructor{˜}}\AgdaSpace{}%
\AgdaInductiveConstructor{𝑁}\AgdaSpace{}%
\AgdaNumber{42}\AgdaSpace{}%
\AgdaOperator{\AgdaInductiveConstructor{˜}}\AgdaSpace{}%
\AgdaInductiveConstructor{𝑁}\AgdaSpace{}%
\AgdaNumber{0}\AgdaSpace{}%
\AgdaOperator{\AgdaFunction{⟧}}\AgdaSpace{}%
\AgdaFunction{ρ⊥}\AgdaSpace{}%
\AgdaOperator{\AgdaDatatype{≡}}\AgdaSpace{}%
\AgdaPostulate{η}\AgdaSpace{}%
\AgdaNumber{42}\<%
\\
\>[0]\AgdaFunction{check-if-zero}\AgdaSpace{}%
\AgdaSymbol{=}\AgdaSpace{}%
\AgdaInductiveConstructor{refl}\<%
\\
%
\\[\AgdaEmptyExtraSkip]%
\>[0]\AgdaFunction{check-if-nonzero}\AgdaSpace{}%
\AgdaSymbol{:}\AgdaSpace{}%
\AgdaOperator{\AgdaFunction{𝒜′⟦}}\AgdaSpace{}%
\AgdaInductiveConstructor{if}\AgdaSpace{}%
\AgdaOperator{\AgdaInductiveConstructor{˜}}\AgdaSpace{}%
\AgdaSymbol{(}\AgdaInductiveConstructor{𝑍}\AgdaSpace{}%
\AgdaOperator{\AgdaInductiveConstructor{˜}}\AgdaSpace{}%
\AgdaInductiveConstructor{𝑁}\AgdaSpace{}%
\AgdaNumber{42}\AgdaSymbol{)}\AgdaSpace{}%
\AgdaOperator{\AgdaInductiveConstructor{˜}}\AgdaSpace{}%
\AgdaInductiveConstructor{𝑁}\AgdaSpace{}%
\AgdaNumber{0}\AgdaSpace{}%
\AgdaOperator{\AgdaInductiveConstructor{˜}}\AgdaSpace{}%
\AgdaInductiveConstructor{𝑁}\AgdaSpace{}%
\AgdaNumber{42}\AgdaSpace{}%
\AgdaOperator{\AgdaFunction{⟧}}\AgdaSpace{}%
\AgdaFunction{ρ⊥}\AgdaSpace{}%
\AgdaOperator{\AgdaDatatype{≡}}\AgdaSpace{}%
\AgdaPostulate{η}\AgdaSpace{}%
\AgdaNumber{42}\<%
\\
\>[0]\AgdaFunction{check-if-nonzero}\AgdaSpace{}%
\AgdaSymbol{=}\AgdaSpace{}%
\AgdaInductiveConstructor{refl}\<%
\\
%
\\[\AgdaEmptyExtraSkip]%
\>[0]\AgdaComment{--\ fix\ (λf.\ 42)\ ≡\ 42}\<%
\\
\>[0]\AgdaFunction{check-fix-const}\AgdaSpace{}%
\AgdaSymbol{:}\<%
\\
\>[0][@{}l@{\AgdaIndent{0}}]%
\>[2]\AgdaOperator{\AgdaFunction{𝒜′⟦}}\AgdaSpace{}%
\AgdaInductiveConstructor{𝑌}\AgdaSpace{}%
\AgdaOperator{\AgdaInductiveConstructor{˜}}\AgdaSpace{}%
\AgdaSymbol{(}\AgdaOperator{\AgdaInductiveConstructor{ƛ}}\AgdaSpace{}%
\AgdaFunction{f}\AgdaSpace{}%
\AgdaOperator{\AgdaInductiveConstructor{˜}}\AgdaSpace{}%
\AgdaInductiveConstructor{𝑁}\AgdaSpace{}%
\AgdaNumber{42}\AgdaSymbol{)}\AgdaSpace{}%
\AgdaOperator{\AgdaFunction{⟧}}\AgdaSpace{}%
\AgdaFunction{ρ⊥}\<%
\\
%
\>[2]\AgdaOperator{\AgdaDatatype{≡}}\AgdaSpace{}%
\AgdaPostulate{η}\AgdaSpace{}%
\AgdaNumber{42}\<%
\\
\>[0]\AgdaFunction{check-fix-const}\AgdaSpace{}%
\AgdaSymbol{=}\AgdaSpace{}%
\AgdaPostulate{fix-fix}\AgdaSpace{}%
\AgdaSymbol{(λ}\AgdaSpace{}%
\AgdaBound{x}\AgdaSpace{}%
\AgdaSymbol{→}\AgdaSpace{}%
\AgdaPostulate{η}\AgdaSpace{}%
\AgdaNumber{42}\AgdaSymbol{)}\<%
\\
%
\\[\AgdaEmptyExtraSkip]%
\>[0]\AgdaComment{--\ fix\ (λg.\ λa.\ 42)\ 2\ ≡\ 42}\<%
\\
\>[0]\AgdaFunction{check-fix-lambda}\AgdaSpace{}%
\AgdaSymbol{:}\<%
\\
\>[0][@{}l@{\AgdaIndent{0}}]%
\>[2]\AgdaOperator{\AgdaFunction{𝒜′⟦}}\AgdaSpace{}%
\AgdaInductiveConstructor{𝑌}\AgdaSpace{}%
\AgdaOperator{\AgdaInductiveConstructor{˜}}\AgdaSpace{}%
\AgdaSymbol{(}\AgdaOperator{\AgdaInductiveConstructor{ƛ}}\AgdaSpace{}%
\AgdaFunction{g}\AgdaSpace{}%
\AgdaOperator{\AgdaInductiveConstructor{˜}}\AgdaSpace{}%
\AgdaOperator{\AgdaInductiveConstructor{ƛ}}\AgdaSpace{}%
\AgdaFunction{a}\AgdaSpace{}%
\AgdaOperator{\AgdaInductiveConstructor{˜}}\AgdaSpace{}%
\AgdaInductiveConstructor{𝑁}\AgdaSpace{}%
\AgdaNumber{42}\AgdaSymbol{)}\AgdaSpace{}%
\AgdaOperator{\AgdaInductiveConstructor{˜}}\AgdaSpace{}%
\AgdaInductiveConstructor{𝑁}\AgdaSpace{}%
\AgdaNumber{2}\AgdaSpace{}%
\AgdaOperator{\AgdaFunction{⟧}}\AgdaSpace{}%
\AgdaFunction{ρ⊥}\<%
\\
%
\>[2]\AgdaOperator{\AgdaDatatype{≡}}\AgdaSpace{}%
\AgdaPostulate{η}\AgdaSpace{}%
\AgdaNumber{42}\<%
\\
\>[0]\AgdaFunction{check-fix-lambda}\AgdaSpace{}%
\AgdaSymbol{=}\AgdaSpace{}%
\AgdaInductiveConstructor{refl}\<%
\\
%
\\[\AgdaEmptyExtraSkip]%
\>[0]\AgdaComment{--\ fix\ (λg.\ λa.\ ifz\ a\ then\ 42\ else\ g\ (pred\ a))\ 101\ ≡\ 42}\<%
\\
\>[0]\AgdaFunction{check-countdown}\AgdaSpace{}%
\AgdaSymbol{:}\<%
\\
\>[0][@{}l@{\AgdaIndent{0}}]%
\>[2]\AgdaOperator{\AgdaFunction{𝒜′⟦}}%
\>[288I]\AgdaInductiveConstructor{𝑌}\AgdaSpace{}%
\AgdaOperator{\AgdaInductiveConstructor{˜}}\AgdaSpace{}%
\AgdaSymbol{(}\AgdaOperator{\AgdaInductiveConstructor{ƛ}}%
\>[291I]\AgdaFunction{g}\AgdaSpace{}%
\AgdaOperator{\AgdaInductiveConstructor{˜}}\AgdaSpace{}%
\AgdaOperator{\AgdaInductiveConstructor{ƛ}}\AgdaSpace{}%
\AgdaFunction{a}\AgdaSpace{}%
\AgdaOperator{\AgdaInductiveConstructor{˜}}\<%
\\
\>[291I][@{}l@{\AgdaIndent{0}}]%
\>[14]\AgdaSymbol{(}\AgdaInductiveConstructor{if}\AgdaSpace{}%
\AgdaOperator{\AgdaInductiveConstructor{˜}}\AgdaSpace{}%
\AgdaSymbol{(}\AgdaInductiveConstructor{𝑍}\AgdaSpace{}%
\AgdaOperator{\AgdaInductiveConstructor{˜}}\AgdaSpace{}%
\AgdaInductiveConstructor{𝑉}\AgdaSpace{}%
\AgdaFunction{a}\AgdaSymbol{)}\AgdaSpace{}%
\AgdaOperator{\AgdaInductiveConstructor{˜}}\AgdaSpace{}%
\AgdaInductiveConstructor{𝑁}\AgdaSpace{}%
\AgdaNumber{42}\AgdaSpace{}%
\AgdaOperator{\AgdaInductiveConstructor{˜}}\AgdaSpace{}%
\AgdaSymbol{(}\AgdaInductiveConstructor{𝑉}\AgdaSpace{}%
\AgdaFunction{g}\AgdaSpace{}%
\AgdaOperator{\AgdaInductiveConstructor{˜}}\AgdaSpace{}%
\AgdaSymbol{(}\AgdaInductiveConstructor{pred⊥}\AgdaSpace{}%
\AgdaOperator{\AgdaInductiveConstructor{˜}}\AgdaSpace{}%
\AgdaInductiveConstructor{𝑉}\AgdaSpace{}%
\AgdaFunction{a}\AgdaSymbol{))))}\<%
\\
\>[.][@{}l@{}]\<[288I]%
\>[6]\AgdaOperator{\AgdaInductiveConstructor{˜}}\AgdaSpace{}%
\AgdaInductiveConstructor{𝑁}\AgdaSpace{}%
\AgdaNumber{101}\<%
\\
\>[2][@{}l@{\AgdaIndent{0}}]%
\>[4]\AgdaOperator{\AgdaFunction{⟧}}\AgdaSpace{}%
\AgdaFunction{ρ⊥}\<%
\\
%
\>[2]\AgdaOperator{\AgdaDatatype{≡}}\AgdaSpace{}%
\AgdaPostulate{η}\AgdaSpace{}%
\AgdaNumber{42}\<%
\\
\>[0]\AgdaFunction{check-countdown}\AgdaSpace{}%
\AgdaSymbol{=}\AgdaSpace{}%
\AgdaInductiveConstructor{refl}\<%
\\
%
\\[\AgdaEmptyExtraSkip]%
\>[0]\AgdaComment{--\ fix\ (λh.\ λa.\ λb.\ ifz\ a\ then\ b\ else\ h\ (pred\ a)\ (succ\ b))\ 4\ 38\ ≡\ 42}\<%
\\
\>[0]\AgdaFunction{check-sum-42}\AgdaSpace{}%
\AgdaSymbol{:}\<%
\\
\>[0][@{}l@{\AgdaIndent{0}}]%
\>[2]\AgdaOperator{\AgdaFunction{𝒜′⟦}}%
\>[320I]\AgdaSymbol{(}\AgdaInductiveConstructor{𝑌}\AgdaSpace{}%
\AgdaOperator{\AgdaInductiveConstructor{˜}}\AgdaSpace{}%
\AgdaSymbol{(}\AgdaOperator{\AgdaInductiveConstructor{ƛ}}%
\>[323I]\AgdaFunction{h}\AgdaSpace{}%
\AgdaOperator{\AgdaInductiveConstructor{˜}}\AgdaSpace{}%
\AgdaOperator{\AgdaInductiveConstructor{ƛ}}\AgdaSpace{}%
\AgdaFunction{a}\AgdaSpace{}%
\AgdaOperator{\AgdaInductiveConstructor{˜}}\AgdaSpace{}%
\AgdaOperator{\AgdaInductiveConstructor{ƛ}}\AgdaSpace{}%
\AgdaFunction{b}\AgdaSpace{}%
\AgdaOperator{\AgdaInductiveConstructor{˜}}\<%
\\
\>[.][@{}l@{}]\<[323I]%
\>[14]\AgdaSymbol{(}\AgdaInductiveConstructor{if}\AgdaSpace{}%
\AgdaOperator{\AgdaInductiveConstructor{˜}}\AgdaSpace{}%
\AgdaSymbol{(}\AgdaInductiveConstructor{𝑍}\AgdaSpace{}%
\AgdaOperator{\AgdaInductiveConstructor{˜}}\AgdaSpace{}%
\AgdaInductiveConstructor{𝑉}\AgdaSpace{}%
\AgdaFunction{a}\AgdaSymbol{)}\AgdaSpace{}%
\AgdaOperator{\AgdaInductiveConstructor{˜}}\AgdaSpace{}%
\AgdaInductiveConstructor{𝑉}\AgdaSpace{}%
\AgdaFunction{b}\AgdaSpace{}%
\AgdaOperator{\AgdaInductiveConstructor{˜}}\AgdaSpace{}%
\AgdaSymbol{(}\AgdaInductiveConstructor{𝑉}\AgdaSpace{}%
\AgdaFunction{h}\AgdaSpace{}%
\AgdaOperator{\AgdaInductiveConstructor{˜}}\AgdaSpace{}%
\AgdaSymbol{(}\AgdaInductiveConstructor{pred⊥}\AgdaSpace{}%
\AgdaOperator{\AgdaInductiveConstructor{˜}}\AgdaSpace{}%
\AgdaInductiveConstructor{𝑉}\AgdaSpace{}%
\AgdaFunction{a}\AgdaSymbol{)}\AgdaSpace{}%
\AgdaOperator{\AgdaInductiveConstructor{˜}}\AgdaSpace{}%
\AgdaSymbol{(}\AgdaInductiveConstructor{succ}\AgdaSpace{}%
\AgdaOperator{\AgdaInductiveConstructor{˜}}\AgdaSpace{}%
\AgdaInductiveConstructor{𝑉}\AgdaSpace{}%
\AgdaFunction{b}\AgdaSymbol{)))))}\<%
\\
\>[.][@{}l@{}]\<[320I]%
\>[6]\AgdaOperator{\AgdaInductiveConstructor{˜}}\AgdaSpace{}%
\AgdaInductiveConstructor{𝑁}\AgdaSpace{}%
\AgdaNumber{4}\AgdaSpace{}%
\AgdaOperator{\AgdaInductiveConstructor{˜}}\AgdaSpace{}%
\AgdaInductiveConstructor{𝑁}\AgdaSpace{}%
\AgdaNumber{38}\<%
\\
\>[2][@{}l@{\AgdaIndent{0}}]%
\>[4]\AgdaOperator{\AgdaFunction{⟧}}\AgdaSpace{}%
\AgdaFunction{ρ⊥}\<%
\\
%
\>[2]\AgdaOperator{\AgdaDatatype{≡}}\AgdaSpace{}%
\AgdaPostulate{η}\AgdaSpace{}%
\AgdaNumber{42}\<%
\\
\>[0]\AgdaFunction{check-sum-42}\AgdaSpace{}%
\AgdaSymbol{=}\AgdaSpace{}%
\AgdaInductiveConstructor{refl}\<%
\\
\>[0]\AgdaComment{--\ Exponential\ in\ first\ arg?}\<%
\end{code}

\bibliographystyle{ACM-Reference-Format}
%\bibliography{PCF}

\end{document}




\begin{abstract}
  This is a progress report on ideas for improving support for Scott–Strachey denotational semantics in Agda.
  It starts by recalling relevant features of denotational semantics and Agda.
  It then 
  \end{abstract}
  
\section{Introduction}

## Domains in Denotational Semantics

In the Scott–Strachey style of denotational semantics:

- types of denotations are (Scott-)domains;
- domains are cpos with least elements, and can be defined recursively;
- denotations are defined in λ-notation, and functions are continuous;
- the isomorphisms between domains and their definitions are left implicit.

## Domains in Agda

In Agda, the [DomainTheory] modules from the [TypeTopology] library provide
well-developed support for domains.

- Domains `D` are tuples `(⟪D⟫, ⊥, _⊑_, axioms)` where:
  
  - `⟪D⟫` is a type of elements,
  - `⊥` is a distinguished element of `⟪D⟫`,
  - `_⊑_` is a partial order on `⟪D⟫`, and
  - `axioms` prove that `(⟪D⟫, _⊑_)` is a directed-complete poset (dcpo)
    with `⊥` least.

- Continuous functions `c` from domain `D` to domain `E`  are pairs
  `(f, axioms)` where:

  - `f` is an underlying function from `⟪D⟫` to `⟪E⟫`, and
  - `axioms` prove that `f` preserves limits of directed sets.

- Domains are defined recursively as bilimits of diagrams.

- Elements of domains are defined in λ-notation, where:

  - λ-abstractions need to be paired with continuity proofs,
  - applications need to select the underlying functions, and
  - the isomorphisms between domains and their definitions are explicit.

## Example: PCF

> PCF is a call-by-name simply-typed λ-calculus equipped with one or two base
> types (usually natural numbers and Booleans) and a fixed point combinator.
> In essence, PCF is the simplest lazy, purely functional programming language.
> ([PLS Lab])

PCF and its denotational semantics were originally defined by Dana Scott in 1969
([Scott 1993]) including combinators (`S`, `K`) instead of λ-abstraction.
Gordon Plotkin subsequently defined a denotational semantics for PCF including 
λ-abstraction ([Plotkin 1977]).

A direct transcription of Plotkin's definition to Agda is available in this
repo at [PCF.Plotkin.Semantics] (see also the [generated PDF]). The code
imports various modules from the [standard Agda library v2.1], and typechecks
with Agda v2.6.4.3 using the (blatantly unsafe) workarounds and postulates in
the [Domains] module.

Many authors have defined denotational semantics for PCF ([PLS Lab]). Recently,
Tom de Jong has given a definition in Agda ([DomainTheory.ScottModelOfPCF])
based on the [DomainTheory] modules from the [TypeTopology] library. The syntax
follows ([Scott 1993]) by including combinators instead of λ-abstraction.

De Jong's semantics of PCF was extended to PCF with variables and λ-abstraction
by Brendan Hart in a final year project supervised by Martín Escardó and
De Jong ([Hart 2020]).[^1] Hart used De Bruin indices for variables.

[^1]: Both De Jong and Hart deviate slightly from Scott's original semantics
    of the predecessor function by defining it to return zero when applied to
    zero, instead of being undefined.

De Jong has also given an example of how to defined a domain recursively, see
[DomainTheory.Bilimits.Dinfinity].

The definitions of the denotations of PCF terms given by De Jong and Hart
show the notational overhead that arises in Agda when using the [DomainTheory]
modules, compared to the definitions given by Scott and Plotkin.

## Extending Agda with Scott-domains

The purpose of this repo is to experiment with extending Agda to allow
denotational semantics to be defined more straightforwardly.

The main idea is to distinguish between declarations of types that correspond
to domains and declarations of ordinary types – e.g., by introducing a universe
(hierarchy) for domains.

A domain type would implicitly be a cpo, with built-in notation for its partial
order and least element. A type that corresponds to the domain of continuous 
functions from a domain to itself would also have a least fixed-point function.

The meaning of a definition in the proposed extension of Agda might be given by
translation to standard Agda, together with import of [DomainTheory] modules
from the [TypeTopology] library. Constructing the required continuity proofs
for λ-abstractions automatically could be challenging.

An alternative approach might be to typecheck code in the extension using
Agda's support for reflection and meta-programming.

Advice and suggestions are welcome, e.g., by posting to the repo [Discussions].

Peter Mosses <p.d.mosses@tudelft.nl>

[PLS Lab]: https://www.pls-lab.org/PCF "Web page"
[Scott 1993]: https://doi.org/10.1016/0304-3975(93)90095-B "TCS paper DOI"
[Plotkin 1977]: https://doi.org/10.1016/0304-3975(77)90044-5 "TCS paper DOI"
[PCF.Plotkin.Semantics]: PCF/Plotkin/Semantics.lagda "Agda module"
[Generated PDF]: latex/PCF.pdf "PDF generated by Agda"
[Domains]: Domains.lagda "Agda module"
[standard Agda library version 2.1]: https://agda.github.io/agda-stdlib/v2.1 "Agda library"
[DomainTheory]: https://www.cs.bham.ac.uk/~mhe/TypeTopology/DomainTheory.index.html "Agda modules"
[TypeTopology]: https://www.cs.bham.ac.uk/~mhe/TypeTopology "Agda library"
[DomainTheory.ScottModelOfPCF]: https://martinescardo.github.io/TypeTopology/DomainTheory.ScottModelOfPCF.ScottModelOfPCF.html "Agda module"
[Hart 2020]: https://github.com/BrendanHart/Investigating-Properties-of-PCF "GitHub repo"
[DomainTheory.Bilimits.Dinfinity]: https://martinescardo.github.io/TypeTopology/DomainTheory.Bilimits.Dinfinity.html  "Agda module"
[Discussions]: https://github.com/pdmosses/xds-agda/discussions
\section{Agda Code}
