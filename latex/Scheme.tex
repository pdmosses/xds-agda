% easychair.tex,v 3.5 2017/03/15

%\documentclass{easychair}
%\documentclass[EPiC]{easychair}
%\documentclass[EPiCempty]{easychair}
%\documentclass[debug]{easychair}
%\documentclass[verbose]{easychair}
%\documentclass[notimes]{easychair}
%\documentclass[withtimes]{easychair}
\documentclass[a4paper]{easychair}
%\documentclass[letterpaper]{easychair}

%\usepackage{doc}

% use this if you have a long article and want to create an index
% \usepackage{makeidx}

% In order to save space or manage large tables or figures in a
% landcape-like text, you can use the rotating and pdflscape
% packages. Uncomment the desired from the below.
%
% \usepackage{rotating}
% \usepackage{pdflscape}

\usepackage[T1]{fontenc}
\usepackage{microtype}
\DisableLigatures[-]{encoding = T1, family = tt* }

\usepackage{multicol}

\usepackage{doi}

% Agda macros

\usepackage[conor]{agda}


% For drafting inline use of Agda symbols:

%\newcommand{\AgdaInline}[2][]{{\setlength\fboxsep{0pt}\colorbox{mygray}{\strut\AgdaFontStyle{#1}\strut}}}
\renewcommand{\AgdaRef}[2][]{{\setlength\fboxsep{0pt}\colorbox{mygray}{\strut\AgdaFontStyle{#2}\strut}}}
%\renewcommand{\AgdaRef}[2][]{\textcolor{AgdaBound}{\AgdaFontStyle{#2}}}
%\renewenvironment{AgdaAlign}{}{}
%\renewcommand{\AgdaCodeStyle}{\small}


% For suppressing color boxes that indicate warnings and errors:

\renewcommand{\AgdaUnsolvedMeta}      [1]{#1}
%    {\AgdaFontStyle{\colorbox{AgdaUnsolvedMeta}{#1}}}
\renewcommand{\AgdaUnsolvedConstraint}[1]{#1}
%    {\AgdaFontStyle{\colorbox{AgdaUnsolvedConstraint}{#1}}}
\renewcommand{\AgdaTerminationProblem}[1]{#1}
%    {\AgdaFontStyle{\colorbox{AgdaTerminationProblem}{#1}}}
\renewcommand{\AgdaIncompletePattern} [1]{#1}
%	{\colorbox{AgdaIncompletePattern}{#1}}
\renewcommand{\AgdaErrorWarning}      [1]{#1}
%	{\colorbox{AgdaErrorWarning}{#1}}
\renewcommand{\AgdaError}             [1]{#1}
%    {\textcolor{AgdaError}{\AgdaFontStyle{\underline{#1}}}}


% Fonts for math symbols

%\usepackage{alphabeta} 
%\usepackage{sansmathfonts} % requires OT1 fontenc
\usepackage[bb=bboldx]{mathalpha}
\usepackage{bm}
\usepackage{stmaryrd}
\usepackage[nohelv, nott]{newtx} % for \lambdabar

% Fonts for Agda highlighting

\renewcommand{\AgdaStringFontStyle}[1]{\textsf{#1}}
\renewcommand{\AgdaCommentFontStyle}[1]{\texttt{\small#1}}
\renewcommand{\AgdaBoundFontStyle}[1]{\textsf{#1}}


% This handles the translation of unicode to latex:

\usepackage{ucs}
\usepackage{newunicodechar}

% MATH SYMBOLS

\newcommand{\newucsmath}[3]{\newunicodechar{#1}{\ensuremath{#2{#3}}}}

% \newucsmath {→} \mathnormal \rightarrow
\newucsmath {⇒} \mathnormal \Rightarrow
\newucsmath {⟶} \mathnormal \longrightarrow
\newucsmath {↔} \mathnormal \leftrightarrow
\newucsmath {▻} \mathnormal \triangleright
\newucsmath {◅} \mathnormal \triangleleft
\newucsmath {∀} \mathnormal \forall
\newucsmath {∪} \mathnormal \cup
\newucsmath {⊃} \mathnormal \supset
\newucsmath {∈} \mathnormal \in
\newucsmath {≤} \mathnormal \leq
\newucsmath {≥} \mathnormal \geq
\newucsmath {≡} \mathnormal \equiv
\newucsmath {∎} \mathnormal \square
\newucsmath {⊑} \mathnormal \sqsubseteq
\newucsmath {⟦} \mathnormal \llbracket
\newucsmath {⟧} \mathnormal \rrbracket
\newucsmath {⦅} \mathnormal {\llparenthesis\,}
\newucsmath {⦆} \mathnormal {\,\rrparenthesis}
\newucsmath {⊎} \mathnormal \uplus
\newucsmath {⊕} \mathnormal \oplus
\newucsmath {∘} \mathnormal \circ
\newucsmath {⊥} \mathnormal \bot
\newucsmath {⊤} \mathnormal \top
\newucsmath {′} \mathnormal {'}
\newucsmath {″} \mathnormal {''}

% GREEK LETTERS

\newucsmath {α} \mathnormal \alpha
\newucsmath {γ} \mathnormal \gamma
\newucsmath {δ} \mathnormal \delta
\newucsmath {ϵ} \mathnormal \epsilon
\newucsmath {φ} \mathnormal \phi
\newucsmath {ψ} \mathnormal \psi
\newucsmath {θ} \mathnormal \theta
\newucsmath {η} \mathnormal \eta
\newucsmath {ι} \mathnormal \iota
\newucsmath {κ} \mathnormal \kappa
\newucsmath {λ} \mathnormal \lambda
\newucsmath {ƛ} \mathnormal \lambdabar
\newucsmath {μ} \mathnormal \mu
\newucsmath {ν} \mathnormal \nu
\newucsmath {π} \mathnormal \pi
\newucsmath {ρ} \mathnormal \rho
\newucsmath {σ} \mathnormal \sigma
\newucsmath {τ} \mathnormal \tau
\newucsmath {ζ} \mathnormal \zeta
\newucsmath {Γ} \mathrm \Gamma
\newucsmath {Σ} \mathrm \Sigma

% BLACKBOARD BOLD

\newucsmath {𝟙} \mathbb 1
\newucsmath {𝕃} \mathbb L
\newucsmath {ℕ} \mathbb N
\newucsmath {𝕋} \mathbb T

% CALLIGRAPHIC

\newucsmath {𝒜} \mathcal A
\newucsmath {𝒞} \mathcal C
\newucsmath {𝒟} \mathcal D
\newucsmath {ℰ} \mathcal E
\newucsmath {𝒦} \mathcal K
\newucsmath {ℒ} \mathcal L
\newucsmath {𝒫} \mathcal P
\newucsmath {𝒱} \mathcal V

% SUBSCRIPTS

\newcommand{\newucssub}[2]{\newunicodechar{#1}{\ensuremath{{}_#2}}}

\newucssub {∞} \infty
\newucssub {₀} 0
\newucssub {₁} 1
\newucssub {₂} 2
\newucssub {ᵢ} {\mathrm i}
\newucssub {ₙ} {\mathrm n}
\newucssub {ₒ} {\mathrm o}

% SUPERSCRIPTS

\newcommand{\newucssup}[2]{\newunicodechar{#1}{\ensuremath{{}^#2}}}

\newucssup {⋆} * % \ast seems to give the same
\newucssup {♯} \sharp
\newucssup {ᵇ} {\mathrm b}
\newucssup {ᶜ} {\mathrm c}
\newucssup {ᵈ} {\mathrm d}
\newucssup {ᶠ} {\mathrm f}
\newucssup {ᵍ} {\mathrm g}
\newucssup {ᴵ} {\mathrm I}
\newucssup {ᴸ} {\mathrm L}
\newucssup {ˡ} {\mathrm l}
\newucssup {ᵖ} {\mathrm p}
\newucssup {ᵒ} {\mathrm o}
\newucssup {ʳ} {\mathrm r}
\newucssup {ⱽ} {\mathrm V}
\newucssup {ᵛ} {\mathrm v}

% TEXT

\newcommand{\newucstext}[3]{\newunicodechar{#1}{#2{#3}}}

\newucstext {∙} \textbf {.}
\newucstext {+} \textbf {+}
% \newucstext {–} \textbf {--}

\newucstext {𝐀} \textbf A
\newucstext {𝐂} \textbf C
\newucstext {𝐄} \textbf E
\newucstext {𝐅} \textbf F
\newucstext {𝐇} \textbf H
\newucstext {𝐊} \textbf K
\newucstext {𝐋} \textbf L
\newucstext {𝐌} \textbf M
\newucstext {𝐍} \textbf N
\newucstext {𝐐} \textbf Q
\newucstext {𝐑} \textbf R
\newucstext {𝐒} \textbf S
\newucstext {𝐓} \textbf T
\newucstext {𝐔} \textbf U
\newucstext {𝐗} \textbf X
\newucstext {𝐩} \textbf p
\newucstext {𝐬} \textbf s
\newucstext {𝐯} \textbf v

\newucstext {𝐿} \textit L
\newucstext {𝑁} \textit N
\newucstext {𝑉} \textit V
\newucstext {𝑌} \textit Y
\newucstext {𝑍} \textit Z

% MISC

\newucsmath {ℓ} \mathnormal \ell

\newunicodechar{⍄}{\ensuremath{\mathnormal{\mbox{\,\setlength\fboxsep{1pt}\fbox{$>$}\,}}}}

\newunicodechar{‵}{\ensuremath{\mathnormal{\mkern 2mu}}} % for \/ in code comments: -- send‵′ 

\AgdaNoSpaceAroundCode{}

%\makeindex

%% Front Matter
%%
% Regular title as in the article class.
%
\title{Denotational Semantics of Scheme R${}^5$ in Agda \\[2ex]
\normalsize DRAFT (\today)}

% Authors are joined by \and. Their affiliations are given by \inst, which indexes
% into the list defined using \institute
%
\author{
Peter D. Mosses%\inst{1}\inst{2}
}

% Institutes for affiliations are also joined by \and,
\institute{
  Delft University of Technology, The Netherlands
  \\
  \email{p.d.mosses@tudelft.nl}
\\
   Swansea University, United Kingdom
%   \\
%   \email{p.d.mosses@swansea.ac.uk}
 }

%  \authorrunning{} has to be set for the shorter version of the authors' names;
% otherwise a warning will be rendered in the running heads. When processed by
% EasyChair, this command is mandatory: a document without \authorrunning
% will be rejected by EasyChair

\authorrunning{Peter Mosses}

% \titlerunning{} has to be set to either the main title or its shorter
% version for the running heads. When processed by
% EasyChair, this command is mandatory: a document without \titlerunning
% will be rejected by EasyChair

\titlerunning{Denotational Semantics of Scheme R${}^5$}

\begin{document}

\maketitle

\begin{abstract}
In synthetic domain theory, all sets are predomains, domains are pointed sets, and functions are implicitly continuous.
The denotational semantics of Scheme (R${}^5$) presented here illustrates how it might look if synthetic domain theory can be implemented in Agda.
As a work-around, the code presented here uses unsatisfiable postulates to allow Agda to type-check the definitions.

The (currently illiterate) Agda source code used to generate this document can be downloaded from
\url{https://github.com/pdmosses/xds-agda}, and browsed with hyperlinks and highlighting at
\url{https://pdmosses.github.io/xds-agda/}.
\end{abstract}


\bigskip\hrule\bigskip

\begin{code}%
\>[0]\AgdaSymbol{\{-\#}\AgdaSpace{}%
\AgdaKeyword{OPTIONS}\AgdaSpace{}%
\AgdaPragma{--rewriting}\AgdaSpace{}%
\AgdaPragma{--confluence-check}\AgdaSpace{}%
\AgdaSymbol{\#-\}}\<%
\\
%
\\[\AgdaEmptyExtraSkip]%
\>[0]\AgdaKeyword{module}\AgdaSpace{}%
\AgdaModule{PCF.All}\AgdaSpace{}%
\AgdaKeyword{where}\<%
\\
%
\\[\AgdaEmptyExtraSkip]%
\>[0]\AgdaKeyword{import}\AgdaSpace{}%
\AgdaModule{PCF.Domain-Notation}\<%
\\
\>[0]\AgdaKeyword{import}\AgdaSpace{}%
\AgdaModule{PCF.Types}\<%
\\
\>[0]\AgdaKeyword{import}\AgdaSpace{}%
\AgdaModule{PCF.Constants}\<%
\\
\>[0]\AgdaKeyword{import}\AgdaSpace{}%
\AgdaModule{PCF.Variables}\<%
\\
\>[0]\AgdaKeyword{import}\AgdaSpace{}%
\AgdaModule{PCF.Environments}\<%
\\
\>[0]\AgdaKeyword{import}\AgdaSpace{}%
\AgdaModule{PCF.Terms}\<%
\\
\>[0]\AgdaKeyword{import}\AgdaSpace{}%
\AgdaModule{PCF.Checks}\<%
\end{code}

\clearpage

\begin{code}%
\>[0]\AgdaKeyword{module}\AgdaSpace{}%
\AgdaModule{Scheme.Domain-Notation}\AgdaSpace{}%
\AgdaKeyword{where}\<%
\\
%
\\[\AgdaEmptyExtraSkip]%
\>[0]\AgdaKeyword{open}\AgdaSpace{}%
\AgdaKeyword{import}\AgdaSpace{}%
\AgdaModule{Relation.Binary.PropositionalEquality.Core}\<%
\\
\>[0][@{}l@{\AgdaIndent{0}}]%
\>[2]\AgdaKeyword{using}\AgdaSpace{}%
\AgdaSymbol{(}\AgdaOperator{\AgdaDatatype{\AgdaUnderscore{}≡\AgdaUnderscore{}}}\AgdaSymbol{;}\AgdaSpace{}%
\AgdaInductiveConstructor{refl}\AgdaSymbol{)}\AgdaSpace{}%
\AgdaKeyword{public}\<%
\\
%
\\[\AgdaEmptyExtraSkip]%
\>[0]\AgdaComment{------------------------------------------------------------------------}\<%
\\
\>[0]\AgdaComment{--\ Agda\ requires\ Predomain\ and\ Domain\ to\ be\ sorts}\<%
\\
%
\\[\AgdaEmptyExtraSkip]%
\>[0]\AgdaFunction{Predomain}%
\>[11]\AgdaSymbol{=}\AgdaSpace{}%
\AgdaPrimitive{Set}\<%
\\
\>[0]\AgdaFunction{Domain}%
\>[11]\AgdaSymbol{=}\AgdaSpace{}%
\AgdaPrimitive{Set}\<%
\\
\>[0]\AgdaKeyword{variable}\<%
\\
\>[0][@{}l@{\AgdaIndent{0}}]%
\>[2]\AgdaGeneralizable{P}%
\>[7]\AgdaSymbol{:}\AgdaSpace{}%
\AgdaFunction{Predomain}\<%
\\
%
\>[2]\AgdaGeneralizable{D}\AgdaSpace{}%
\AgdaGeneralizable{E}%
\>[7]\AgdaSymbol{:}\AgdaSpace{}%
\AgdaFunction{Domain}\<%
\\
%
\\[\AgdaEmptyExtraSkip]%
\>[0]\AgdaComment{--\ Domains\ are\ pointed}\<%
\\
\>[0]\AgdaKeyword{postulate}\<%
\\
\>[0][@{}l@{\AgdaIndent{0}}]%
\>[2]\AgdaPostulate{⊥}%
\>[12]\AgdaSymbol{:}\AgdaSpace{}%
\AgdaSymbol{\{}\AgdaBound{D}\AgdaSpace{}%
\AgdaSymbol{:}\AgdaSpace{}%
\AgdaFunction{Domain}\AgdaSymbol{\}}\AgdaSpace{}%
\AgdaSymbol{→}\AgdaSpace{}%
\AgdaBound{D}\<%
\\
%
\>[2]\AgdaPostulate{strict}%
\>[12]\AgdaSymbol{:}\AgdaSpace{}%
\AgdaSymbol{\{}\AgdaBound{D}\AgdaSpace{}%
\AgdaBound{E}\AgdaSpace{}%
\AgdaSymbol{:}\AgdaSpace{}%
\AgdaFunction{Domain}\AgdaSymbol{\}}\AgdaSpace{}%
\AgdaSymbol{→}\AgdaSpace{}%
\AgdaSymbol{(}\AgdaBound{D}\AgdaSpace{}%
\AgdaSymbol{→}\AgdaSpace{}%
\AgdaBound{E}\AgdaSymbol{)}\AgdaSpace{}%
\AgdaSymbol{→}\AgdaSpace{}%
\AgdaSymbol{(}\AgdaBound{D}\AgdaSpace{}%
\AgdaSymbol{→}\AgdaSpace{}%
\AgdaBound{E}\AgdaSymbol{)}\<%
\\
%
\\[\AgdaEmptyExtraSkip]%
%
\>[2]\AgdaComment{--\ Properties}\<%
\\
%
\>[2]\AgdaPostulate{strict-⊥}%
\>[12]\AgdaSymbol{:}\AgdaSpace{}%
\AgdaSymbol{∀}%
\>[30I]\AgdaSymbol{\{}\AgdaBound{D}\AgdaSpace{}%
\AgdaBound{E}\AgdaSymbol{\}}\AgdaSpace{}%
\AgdaSymbol{→}\AgdaSpace{}%
\AgdaSymbol{(}\AgdaBound{f}\AgdaSpace{}%
\AgdaSymbol{:}\AgdaSpace{}%
\AgdaBound{D}\AgdaSpace{}%
\AgdaSymbol{→}\AgdaSpace{}%
\AgdaBound{E}\AgdaSymbol{)}\AgdaSpace{}%
\AgdaSymbol{→}\<%
\\
\>[.][@{}l@{}]\<[30I]%
\>[16]\AgdaPostulate{strict}\AgdaSpace{}%
\AgdaBound{f}\AgdaSpace{}%
\AgdaPostulate{⊥}\AgdaSpace{}%
\AgdaOperator{\AgdaDatatype{≡}}\AgdaSpace{}%
\AgdaPostulate{⊥}\<%
\\
%
\\[\AgdaEmptyExtraSkip]%
\>[0]\AgdaComment{------------------------------------------------------------------------}\<%
\\
\>[0]\AgdaComment{--\ Fixed\ points\ of\ endofunctions\ on\ function\ domains}\<%
\\
%
\\[\AgdaEmptyExtraSkip]%
\>[0]\AgdaKeyword{postulate}\<%
\\
\>[0][@{}l@{\AgdaIndent{0}}]%
\>[2]\AgdaPostulate{fix}%
\>[12]\AgdaSymbol{:}\AgdaSpace{}%
\AgdaSymbol{\{}\AgdaBound{D}\AgdaSpace{}%
\AgdaSymbol{:}\AgdaSpace{}%
\AgdaFunction{Domain}\AgdaSymbol{\}}\AgdaSpace{}%
\AgdaSymbol{→}\AgdaSpace{}%
\AgdaSymbol{(}\AgdaBound{D}\AgdaSpace{}%
\AgdaSymbol{→}\AgdaSpace{}%
\AgdaBound{D}\AgdaSymbol{)}\AgdaSpace{}%
\AgdaSymbol{→}\AgdaSpace{}%
\AgdaBound{D}\<%
\\
%
\\[\AgdaEmptyExtraSkip]%
%
\>[2]\AgdaComment{--\ Properties}\<%
\\
%
\>[2]\AgdaPostulate{fix-fix}%
\>[12]\AgdaSymbol{:}%
\>[52I]\AgdaSymbol{∀}\AgdaSpace{}%
\AgdaSymbol{\{}\AgdaBound{D}\AgdaSymbol{\}}\AgdaSpace{}%
\AgdaSymbol{(}\AgdaBound{f}\AgdaSpace{}%
\AgdaSymbol{:}\AgdaSpace{}%
\AgdaBound{D}\AgdaSpace{}%
\AgdaSymbol{→}\AgdaSpace{}%
\AgdaBound{D}\AgdaSymbol{)}\AgdaSpace{}%
\AgdaSymbol{→}\<%
\\
\>[52I][@{}l@{\AgdaIndent{0}}]%
\>[15]\AgdaPostulate{fix}\AgdaSpace{}%
\AgdaBound{f}\AgdaSpace{}%
\AgdaOperator{\AgdaDatatype{≡}}\AgdaSpace{}%
\AgdaBound{f}\AgdaSpace{}%
\AgdaSymbol{(}\AgdaPostulate{fix}\AgdaSpace{}%
\AgdaBound{f}\AgdaSymbol{)}\<%
\\
%
\>[2]\AgdaPostulate{fix-app}%
\>[12]\AgdaSymbol{:}%
\>[65I]\AgdaSymbol{∀}\AgdaSpace{}%
\AgdaSymbol{\{}\AgdaBound{P}\AgdaSpace{}%
\AgdaBound{D}\AgdaSymbol{\}}\AgdaSpace{}%
\AgdaSymbol{(}\AgdaBound{f}\AgdaSpace{}%
\AgdaSymbol{:}\AgdaSpace{}%
\AgdaSymbol{(}\AgdaBound{P}\AgdaSpace{}%
\AgdaSymbol{→}\AgdaSpace{}%
\AgdaBound{D}\AgdaSymbol{)}\AgdaSpace{}%
\AgdaSymbol{→}\AgdaSpace{}%
\AgdaSymbol{(}\AgdaBound{P}\AgdaSpace{}%
\AgdaSymbol{→}\AgdaSpace{}%
\AgdaBound{D}\AgdaSymbol{))}\AgdaSpace{}%
\AgdaSymbol{(}\AgdaBound{p}\AgdaSpace{}%
\AgdaSymbol{:}\AgdaSpace{}%
\AgdaBound{P}\AgdaSymbol{)}\AgdaSpace{}%
\AgdaSymbol{→}\<%
\\
\>[65I][@{}l@{\AgdaIndent{0}}]%
\>[15]\AgdaPostulate{fix}\AgdaSpace{}%
\AgdaBound{f}\AgdaSpace{}%
\AgdaBound{p}\AgdaSpace{}%
\AgdaOperator{\AgdaDatatype{≡}}\AgdaSpace{}%
\AgdaBound{f}\AgdaSpace{}%
\AgdaSymbol{(}\AgdaPostulate{fix}\AgdaSpace{}%
\AgdaBound{f}\AgdaSymbol{)}\AgdaSpace{}%
\AgdaBound{p}\<%
\\
%
\\[\AgdaEmptyExtraSkip]%
\>[0]\AgdaComment{------------------------------------------------------------------------}\<%
\\
\>[0]\AgdaComment{--\ Lifted\ domains}\<%
\\
%
\\[\AgdaEmptyExtraSkip]%
\>[0]\AgdaKeyword{postulate}\<%
\\
\>[0][@{}l@{\AgdaIndent{0}}]%
\>[2]\AgdaPostulate{𝕃}%
\>[12]\AgdaSymbol{:}\AgdaSpace{}%
\AgdaFunction{Predomain}\AgdaSpace{}%
\AgdaSymbol{→}\AgdaSpace{}%
\AgdaFunction{Domain}\<%
\\
%
\>[2]\AgdaPostulate{η}%
\>[12]\AgdaSymbol{:}\AgdaSpace{}%
\AgdaSymbol{\{}\AgdaBound{P}\AgdaSpace{}%
\AgdaSymbol{:}\AgdaSpace{}%
\AgdaFunction{Predomain}\AgdaSymbol{\}}\AgdaSpace{}%
\AgdaSymbol{→}\AgdaSpace{}%
\AgdaBound{P}\AgdaSpace{}%
\AgdaSymbol{→}\AgdaSpace{}%
\AgdaPostulate{𝕃}\AgdaSpace{}%
\AgdaBound{P}\<%
\\
%
\>[2]\AgdaOperator{\AgdaPostulate{\AgdaUnderscore{}♯}}%
\>[12]\AgdaSymbol{:}\AgdaSpace{}%
\AgdaSymbol{\{}\AgdaBound{P}\AgdaSpace{}%
\AgdaSymbol{:}\AgdaSpace{}%
\AgdaFunction{Predomain}\AgdaSymbol{\}}\AgdaSpace{}%
\AgdaSymbol{\{}\AgdaBound{D}\AgdaSpace{}%
\AgdaSymbol{:}\AgdaSpace{}%
\AgdaFunction{Domain}\AgdaSymbol{\}}\AgdaSpace{}%
\AgdaSymbol{→}\AgdaSpace{}%
\AgdaSymbol{(}\AgdaBound{P}\AgdaSpace{}%
\AgdaSymbol{→}\AgdaSpace{}%
\AgdaBound{D}\AgdaSymbol{)}\AgdaSpace{}%
\AgdaSymbol{→}\AgdaSpace{}%
\AgdaSymbol{(}\AgdaPostulate{𝕃}\AgdaSpace{}%
\AgdaBound{P}\AgdaSpace{}%
\AgdaSymbol{→}\AgdaSpace{}%
\AgdaBound{D}\AgdaSymbol{)}\<%
\\
%
\\[\AgdaEmptyExtraSkip]%
%
\>[2]\AgdaComment{--\ Properties}\<%
\\
%
\>[2]\AgdaPostulate{elim-♯-η}%
\>[12]\AgdaSymbol{:}\AgdaSpace{}%
\AgdaSymbol{∀}%
\>[115I]\AgdaSymbol{\{}\AgdaBound{P}\AgdaSpace{}%
\AgdaBound{D}\AgdaSymbol{\}}\AgdaSpace{}%
\AgdaSymbol{(}\AgdaBound{f}\AgdaSpace{}%
\AgdaSymbol{:}\AgdaSpace{}%
\AgdaBound{P}\AgdaSpace{}%
\AgdaSymbol{→}\AgdaSpace{}%
\AgdaBound{D}\AgdaSymbol{)}\AgdaSpace{}%
\AgdaSymbol{(}\AgdaBound{p}\AgdaSpace{}%
\AgdaSymbol{:}\AgdaSpace{}%
\AgdaBound{P}\AgdaSymbol{)}%
\>[43]\AgdaSymbol{→}\<%
\\
\>[.][@{}l@{}]\<[115I]%
\>[16]\AgdaSymbol{(}\AgdaBound{f}\AgdaSpace{}%
\AgdaOperator{\AgdaPostulate{♯}}\AgdaSymbol{)}\AgdaSpace{}%
\AgdaSymbol{(}\AgdaPostulate{η}\AgdaSpace{}%
\AgdaBound{p}\AgdaSymbol{)}\AgdaSpace{}%
\AgdaOperator{\AgdaDatatype{≡}}\AgdaSpace{}%
\AgdaBound{f}\AgdaSpace{}%
\AgdaBound{p}\<%
\\
%
\>[2]\AgdaPostulate{elim-♯-⊥}%
\>[12]\AgdaSymbol{:}\AgdaSpace{}%
\AgdaSymbol{∀}%
\>[132I]\AgdaSymbol{\{}\AgdaBound{P}\AgdaSpace{}%
\AgdaBound{D}\AgdaSymbol{\}}\AgdaSpace{}%
\AgdaSymbol{(}\AgdaBound{f}\AgdaSpace{}%
\AgdaSymbol{:}\AgdaSpace{}%
\AgdaBound{P}\AgdaSpace{}%
\AgdaSymbol{→}\AgdaSpace{}%
\AgdaBound{D}\AgdaSymbol{)}\AgdaSpace{}%
\AgdaSymbol{→}\<%
\\
\>[.][@{}l@{}]\<[132I]%
\>[16]\AgdaSymbol{(}\AgdaBound{f}\AgdaSpace{}%
\AgdaOperator{\AgdaPostulate{♯}}\AgdaSymbol{)}\AgdaSpace{}%
\AgdaPostulate{⊥}\AgdaSpace{}%
\AgdaOperator{\AgdaDatatype{≡}}\AgdaSpace{}%
\AgdaPostulate{⊥}\<%
\end{code}
\clearpage
\begin{code}%
\>[0]\AgdaComment{------------------------------------------------------------------------}\<%
\\
\>[0]\AgdaComment{--\ Flat\ domains}\<%
\\
%
\\[\AgdaEmptyExtraSkip]%
\>[0]\AgdaOperator{\AgdaFunction{\AgdaUnderscore{}+⊥}}%
\>[6]\AgdaSymbol{:}\AgdaSpace{}%
\AgdaPrimitive{Set}\AgdaSpace{}%
\AgdaSymbol{→}\AgdaSpace{}%
\AgdaFunction{Domain}\<%
\\
\>[0]\AgdaBound{S}\AgdaSpace{}%
\AgdaOperator{\AgdaFunction{+⊥}}%
\>[6]\AgdaSymbol{=}\AgdaSpace{}%
\AgdaPostulate{𝕃}\AgdaSpace{}%
\AgdaBound{S}\<%
\\
%
\\[\AgdaEmptyExtraSkip]%
\>[0]\AgdaComment{--\ Lifted\ operations\ on\ ℕ}\<%
\\
%
\\[\AgdaEmptyExtraSkip]%
\>[0]\AgdaKeyword{open}\AgdaSpace{}%
\AgdaKeyword{import}\AgdaSpace{}%
\AgdaModule{Agda.Builtin.Nat}\<%
\\
\>[0][@{}l@{\AgdaIndent{0}}]%
\>[2]\AgdaKeyword{using}\AgdaSpace{}%
\AgdaSymbol{(}\AgdaOperator{\AgdaPrimitive{\AgdaUnderscore{}==\AgdaUnderscore{}}}\AgdaSymbol{;}\AgdaSpace{}%
\AgdaOperator{\AgdaPrimitive{\AgdaUnderscore{}<\AgdaUnderscore{}}}\AgdaSymbol{)}\AgdaSpace{}%
\AgdaKeyword{public}\<%
\\
\>[0]\AgdaKeyword{open}\AgdaSpace{}%
\AgdaKeyword{import}\AgdaSpace{}%
\AgdaModule{Data.Nat.Base}\AgdaSpace{}%
\AgdaSymbol{as}\AgdaSpace{}%
\AgdaModule{Nat}\<%
\\
\>[0][@{}l@{\AgdaIndent{0}}]%
\>[2]\AgdaKeyword{using}\AgdaSpace{}%
\AgdaSymbol{(}\AgdaDatatype{ℕ}\AgdaSymbol{;}\AgdaSpace{}%
\AgdaInductiveConstructor{suc}\AgdaSymbol{;}\AgdaSpace{}%
\AgdaFunction{pred}\AgdaSymbol{)}\AgdaSpace{}%
\AgdaKeyword{public}\<%
\\
\>[0]\AgdaKeyword{open}\AgdaSpace{}%
\AgdaKeyword{import}\AgdaSpace{}%
\AgdaModule{Data.Bool.Base}\<%
\\
\>[0][@{}l@{\AgdaIndent{0}}]%
\>[2]\AgdaKeyword{using}\AgdaSpace{}%
\AgdaSymbol{(}\AgdaDatatype{Bool}\AgdaSymbol{)}\AgdaSpace{}%
\AgdaKeyword{public}\<%
\\
%
\\[\AgdaEmptyExtraSkip]%
\>[0]\AgdaComment{--\ ν\ ==⊥\ n\ :\ Bool\ +⊥}\<%
\\
%
\\[\AgdaEmptyExtraSkip]%
\>[0]\AgdaOperator{\AgdaFunction{\AgdaUnderscore{}==⊥\AgdaUnderscore{}}}\AgdaSpace{}%
\AgdaSymbol{:}\AgdaSpace{}%
\AgdaDatatype{ℕ}\AgdaSpace{}%
\AgdaOperator{\AgdaFunction{+⊥}}\AgdaSpace{}%
\AgdaSymbol{→}\AgdaSpace{}%
\AgdaDatatype{ℕ}\AgdaSpace{}%
\AgdaSymbol{→}\AgdaSpace{}%
\AgdaDatatype{Bool}\AgdaSpace{}%
\AgdaOperator{\AgdaFunction{+⊥}}\<%
\\
\>[0]\AgdaBound{ν}\AgdaSpace{}%
\AgdaOperator{\AgdaFunction{==⊥}}\AgdaSpace{}%
\AgdaBound{n}\AgdaSpace{}%
\AgdaSymbol{=}\AgdaSpace{}%
\AgdaSymbol{((λ}\AgdaSpace{}%
\AgdaBound{m}\AgdaSpace{}%
\AgdaSymbol{→}\AgdaSpace{}%
\AgdaPostulate{η}\AgdaSpace{}%
\AgdaSymbol{(}\AgdaBound{m}\AgdaSpace{}%
\AgdaOperator{\AgdaPrimitive{==}}\AgdaSpace{}%
\AgdaBound{n}\AgdaSymbol{))}\AgdaSpace{}%
\AgdaOperator{\AgdaPostulate{♯}}\AgdaSymbol{)}\AgdaSpace{}%
\AgdaBound{ν}\<%
\\
%
\\[\AgdaEmptyExtraSkip]%
\>[0]\AgdaComment{--\ ν\ >=⊥\ n\ :\ Bool\ +⊥}\<%
\\
%
\\[\AgdaEmptyExtraSkip]%
\>[0]\AgdaOperator{\AgdaFunction{\AgdaUnderscore{}>=⊥\AgdaUnderscore{}}}\AgdaSpace{}%
\AgdaSymbol{:}\AgdaSpace{}%
\AgdaDatatype{ℕ}\AgdaSpace{}%
\AgdaOperator{\AgdaFunction{+⊥}}\AgdaSpace{}%
\AgdaSymbol{→}\AgdaSpace{}%
\AgdaDatatype{ℕ}\AgdaSpace{}%
\AgdaSymbol{→}\AgdaSpace{}%
\AgdaDatatype{Bool}\AgdaSpace{}%
\AgdaOperator{\AgdaFunction{+⊥}}\<%
\\
\>[0]\AgdaBound{ν}\AgdaSpace{}%
\AgdaOperator{\AgdaFunction{>=⊥}}\AgdaSpace{}%
\AgdaBound{n}\AgdaSpace{}%
\AgdaSymbol{=}\AgdaSpace{}%
\AgdaSymbol{((λ}\AgdaSpace{}%
\AgdaBound{m}\AgdaSpace{}%
\AgdaSymbol{→}\AgdaSpace{}%
\AgdaPostulate{η}\AgdaSpace{}%
\AgdaSymbol{(}\AgdaBound{n}\AgdaSpace{}%
\AgdaOperator{\AgdaPrimitive{<}}\AgdaSpace{}%
\AgdaBound{m}\AgdaSymbol{))}\AgdaSpace{}%
\AgdaOperator{\AgdaPostulate{♯}}\AgdaSymbol{)}\AgdaSpace{}%
\AgdaBound{ν}\<%
\\
%
\\[\AgdaEmptyExtraSkip]%
\>[0]\AgdaComment{------------------------------------------------------------------------}\<%
\\
\>[0]\AgdaComment{--\ Products}\<%
\\
%
\\[\AgdaEmptyExtraSkip]%
\>[0]\AgdaComment{--\ Products\ of\ (pre)domains\ are\ Cartesian}\<%
\\
%
\\[\AgdaEmptyExtraSkip]%
\>[0]\AgdaKeyword{open}\AgdaSpace{}%
\AgdaKeyword{import}\AgdaSpace{}%
\AgdaModule{Data.Product.Base}\<%
\\
\>[0][@{}l@{\AgdaIndent{0}}]%
\>[2]\AgdaKeyword{using}\AgdaSpace{}%
\AgdaSymbol{(}\AgdaOperator{\AgdaFunction{\AgdaUnderscore{}×\AgdaUnderscore{}}}\AgdaSymbol{;}\AgdaSpace{}%
\AgdaOperator{\AgdaInductiveConstructor{\AgdaUnderscore{},\AgdaUnderscore{}}}\AgdaSymbol{)}\AgdaSpace{}%
\AgdaKeyword{renaming}\AgdaSpace{}%
\AgdaSymbol{(}\AgdaField{proj₁}\AgdaSpace{}%
\AgdaSymbol{to}\AgdaSpace{}%
\AgdaField{\AgdaUnderscore{}↓1}\AgdaSymbol{;}\AgdaSpace{}%
\AgdaField{proj₂}\AgdaSpace{}%
\AgdaSymbol{to}\AgdaSpace{}%
\AgdaField{\AgdaUnderscore{}↓2}\AgdaSymbol{)}\AgdaSpace{}%
\AgdaKeyword{public}\<%
\\
%
\\[\AgdaEmptyExtraSkip]%
\>[0]\AgdaComment{--\ (p₁\ ,\ ...\ ,\ pₙ)\ :\ P₁\ ×\ ...\ ×\ Pₙ\ \ (n\ ≥\ 2)}\<%
\\
\>[0]\AgdaComment{--\ \AgdaUnderscore{}↓1\ :\ P₁\ ×\ P₂\ →\ P₁}\<%
\\
\>[0]\AgdaComment{--\ \AgdaUnderscore{}↓2\ :\ P₁\ ×\ P₂\ →\ P₂}\<%
\\
%
\\[\AgdaEmptyExtraSkip]%
\>[0]\AgdaComment{------------------------------------------------------------------------}\<%
\\
\>[0]\AgdaComment{--\ Sum\ domains}\<%
\\
%
\\[\AgdaEmptyExtraSkip]%
\>[0]\AgdaComment{--\ Disjoint\ unions\ of\ (pre)domains\ are\ unpointed\ predomains}\<%
\\
\>[0]\AgdaComment{--\ Lifted\ disjoint\ unions\ of\ domains\ are\ separated\ sum\ domains}\<%
\\
%
\\[\AgdaEmptyExtraSkip]%
\>[0]\AgdaKeyword{open}\AgdaSpace{}%
\AgdaKeyword{import}\AgdaSpace{}%
\AgdaModule{Data.Sum.Base}\<%
\\
\>[0][@{}l@{\AgdaIndent{0}}]%
\>[2]\AgdaKeyword{using}\AgdaSpace{}%
\AgdaSymbol{(}\AgdaInductiveConstructor{inj₁}\AgdaSymbol{;}\AgdaSpace{}%
\AgdaInductiveConstructor{inj₂}\AgdaSymbol{)}\AgdaSpace{}%
\AgdaKeyword{renaming}\AgdaSpace{}%
\AgdaSymbol{(}\AgdaOperator{\AgdaDatatype{\AgdaUnderscore{}⊎\AgdaUnderscore{}}}\AgdaSpace{}%
\AgdaSymbol{to}\AgdaSpace{}%
\AgdaOperator{\AgdaDatatype{\AgdaUnderscore{}+\AgdaUnderscore{}}}\AgdaSymbol{;}\AgdaSpace{}%
\AgdaOperator{\AgdaFunction{[\AgdaUnderscore{},\AgdaUnderscore{}]′}}\AgdaSpace{}%
\AgdaSymbol{to}\AgdaSpace{}%
\AgdaOperator{\AgdaFunction{[\AgdaUnderscore{},\AgdaUnderscore{}]}}\AgdaSymbol{)}\AgdaSpace{}%
\AgdaKeyword{public}\<%
\\
%
\\[\AgdaEmptyExtraSkip]%
\>[0]\AgdaComment{--\ inj₁\ :\ P₁\ →\ P₁\ +\ P₂}\<%
\\
\>[0]\AgdaComment{--\ inj₂\ :\ P₂\ →\ P₁\ +\ P₂}\<%
\\
\>[0]\AgdaComment{--\ [\ f₁\ ,\ f₂\ ]\ :\ (P₁\ →\ P)\ →\ (P₂\ →\ P)\ →\ (P₁\ +\ P₂)\ →\ P}\<%
\end{code}
\clearpage
\begin{code}%
\>[0]\AgdaComment{------------------------------------------------------------------------}\<%
\\
\>[0]\AgdaComment{--\ Finite\ sequences}\<%
\\
%
\\[\AgdaEmptyExtraSkip]%
\>[0]\AgdaKeyword{open}\AgdaSpace{}%
\AgdaKeyword{import}\AgdaSpace{}%
\AgdaModule{Data.Vec.Recursive}\<%
\\
\>[0][@{}l@{\AgdaIndent{0}}]%
\>[2]\AgdaKeyword{using}\AgdaSpace{}%
\AgdaSymbol{(}\AgdaOperator{\AgdaFunction{\AgdaUnderscore{}\textasciicircum{}\AgdaUnderscore{}}}\AgdaSymbol{;}\AgdaSpace{}%
\AgdaInductiveConstructor{[]}\AgdaSymbol{;}\AgdaSpace{}%
\AgdaFunction{append}\AgdaSymbol{)}\AgdaSpace{}%
\AgdaKeyword{public}\<%
\\
\>[0]\AgdaKeyword{open}\AgdaSpace{}%
\AgdaKeyword{import}\AgdaSpace{}%
\AgdaModule{Agda.Builtin.Sigma}\<%
\\
\>[0][@{}l@{\AgdaIndent{0}}]%
\>[2]\AgdaKeyword{using}\AgdaSpace{}%
\AgdaSymbol{(}\AgdaRecord{Σ}\AgdaSymbol{)}\<%
\\
%
\\[\AgdaEmptyExtraSkip]%
\>[0]\AgdaComment{--\ Sequence\ predomains}\<%
\\
\>[0]\AgdaComment{--\ P\ \textasciicircum{}\ n\ \ =\ P\ ×\ ...\ ×\ P\ \ (n\ ≥\ 0)}\<%
\\
\>[0]\AgdaComment{--\ P\ *\ \ \ \ =\ (P\ \textasciicircum{}\ 0)\ +\ ...\ +\ (P\ \textasciicircum{}\ n)\ +\ ...}\<%
\\
\>[0]\AgdaComment{--\ (n,\ p₁\ ,\ ...\ ,\ pₙ)\ :\ P\ *}\<%
\\
%
\\[\AgdaEmptyExtraSkip]%
\>[0]\AgdaOperator{\AgdaFunction{\AgdaUnderscore{}*}}%
\>[5]\AgdaSymbol{:}\AgdaSpace{}%
\AgdaPrimitive{Set}\AgdaSpace{}%
\AgdaSymbol{→}\AgdaSpace{}%
\AgdaPrimitive{Set}\<%
\\
\>[0]\AgdaBound{P}\AgdaSpace{}%
\AgdaOperator{\AgdaFunction{*}}%
\>[5]\AgdaSymbol{=}\AgdaSpace{}%
\AgdaRecord{Σ}\AgdaSpace{}%
\AgdaDatatype{ℕ}\AgdaSpace{}%
\AgdaSymbol{(}\AgdaBound{P}\AgdaSpace{}%
\AgdaOperator{\AgdaFunction{\textasciicircum{}\AgdaUnderscore{}}}\AgdaSymbol{)}\<%
\\
%
\\[\AgdaEmptyExtraSkip]%
\>[0]\AgdaComment{--\ \#′\ S\ *\ :\ ℕ}\<%
\\
%
\\[\AgdaEmptyExtraSkip]%
\>[0]\AgdaFunction{\#′}\AgdaSpace{}%
\AgdaSymbol{:}\AgdaSpace{}%
\AgdaSymbol{\{}\AgdaBound{S}\AgdaSpace{}%
\AgdaSymbol{:}\AgdaSpace{}%
\AgdaPrimitive{Set}\AgdaSymbol{\}}\AgdaSpace{}%
\AgdaSymbol{→}\AgdaSpace{}%
\AgdaBound{S}\AgdaSpace{}%
\AgdaOperator{\AgdaFunction{*}}\AgdaSpace{}%
\AgdaSymbol{→}\AgdaSpace{}%
\AgdaDatatype{ℕ}\<%
\\
\>[0]\AgdaFunction{\#′}\AgdaSpace{}%
\AgdaSymbol{(}\AgdaBound{n}\AgdaSpace{}%
\AgdaOperator{\AgdaInductiveConstructor{,}}\AgdaSpace{}%
\AgdaSymbol{\AgdaUnderscore{})}\AgdaSpace{}%
\AgdaSymbol{=}\AgdaSpace{}%
\AgdaBound{n}\<%
\\
%
\\[\AgdaEmptyExtraSkip]%
\>[0]\AgdaOperator{\AgdaFunction{\AgdaUnderscore{}::′\AgdaUnderscore{}}}\AgdaSpace{}%
\AgdaSymbol{:}\AgdaSpace{}%
\AgdaSymbol{∀}\AgdaSpace{}%
\AgdaSymbol{\{}\AgdaBound{P}\AgdaSpace{}%
\AgdaSymbol{:}\AgdaSpace{}%
\AgdaPrimitive{Set}\AgdaSymbol{\}}\AgdaSpace{}%
\AgdaSymbol{→}\AgdaSpace{}%
\AgdaBound{P}\AgdaSpace{}%
\AgdaSymbol{→}\AgdaSpace{}%
\AgdaBound{P}\AgdaSpace{}%
\AgdaOperator{\AgdaFunction{*}}\AgdaSpace{}%
\AgdaSymbol{→}\AgdaSpace{}%
\AgdaBound{P}\AgdaSpace{}%
\AgdaOperator{\AgdaFunction{*}}\<%
\\
\>[0]\AgdaBound{p}\AgdaSpace{}%
\AgdaOperator{\AgdaFunction{::′}}\AgdaSpace{}%
\AgdaSymbol{(}\AgdaNumber{0}%
\>[14]\AgdaOperator{\AgdaInductiveConstructor{,}}\AgdaSpace{}%
\AgdaBound{ps}\AgdaSymbol{)}\AgdaSpace{}%
\AgdaSymbol{=}\AgdaSpace{}%
\AgdaSymbol{(}\AgdaNumber{1}\AgdaSpace{}%
\AgdaOperator{\AgdaInductiveConstructor{,}}\AgdaSpace{}%
\AgdaBound{p}\AgdaSymbol{)}\<%
\\
\>[0]\AgdaBound{p}\AgdaSpace{}%
\AgdaOperator{\AgdaFunction{::′}}\AgdaSpace{}%
\AgdaSymbol{(}\AgdaInductiveConstructor{suc}\AgdaSpace{}%
\AgdaBound{n}%
\>[14]\AgdaOperator{\AgdaInductiveConstructor{,}}\AgdaSpace{}%
\AgdaBound{ps}\AgdaSymbol{)}\AgdaSpace{}%
\AgdaSymbol{=}\AgdaSpace{}%
\AgdaSymbol{(}\AgdaInductiveConstructor{suc}\AgdaSpace{}%
\AgdaSymbol{(}\AgdaInductiveConstructor{suc}\AgdaSpace{}%
\AgdaBound{n}\AgdaSymbol{)}\AgdaSpace{}%
\AgdaOperator{\AgdaInductiveConstructor{,}}\AgdaSpace{}%
\AgdaBound{p}\AgdaSpace{}%
\AgdaOperator{\AgdaInductiveConstructor{,}}\AgdaSpace{}%
\AgdaBound{ps}\AgdaSymbol{)}\<%
\\
%
\\[\AgdaEmptyExtraSkip]%
\>[0]\AgdaOperator{\AgdaFunction{\AgdaUnderscore{}↓′\AgdaUnderscore{}}}\AgdaSpace{}%
\AgdaSymbol{:}\AgdaSpace{}%
\AgdaSymbol{∀}\AgdaSpace{}%
\AgdaSymbol{\{}\AgdaBound{P}\AgdaSpace{}%
\AgdaSymbol{:}\AgdaSpace{}%
\AgdaPrimitive{Set}\AgdaSymbol{\}}\AgdaSpace{}%
\AgdaSymbol{→}\AgdaSpace{}%
\AgdaBound{P}\AgdaSpace{}%
\AgdaOperator{\AgdaFunction{*}}\AgdaSpace{}%
\AgdaSymbol{→}\AgdaSpace{}%
\AgdaDatatype{ℕ}\AgdaSpace{}%
\AgdaSymbol{→}\AgdaSpace{}%
\AgdaPostulate{𝕃}\AgdaSpace{}%
\AgdaBound{P}\<%
\\
\>[0]\AgdaSymbol{(}\AgdaNumber{1}%
\>[14]\AgdaOperator{\AgdaInductiveConstructor{,}}\AgdaSpace{}%
\AgdaBound{p}\AgdaSymbol{)}%
\>[25]\AgdaOperator{\AgdaFunction{↓′}}\AgdaSpace{}%
\AgdaNumber{1}%
\>[41]\AgdaSymbol{=}\AgdaSpace{}%
\AgdaPostulate{η}\AgdaSpace{}%
\AgdaBound{p}\<%
\\
\>[0]\AgdaSymbol{(}\AgdaInductiveConstructor{suc}\AgdaSpace{}%
\AgdaSymbol{(}\AgdaInductiveConstructor{suc}\AgdaSpace{}%
\AgdaBound{n}\AgdaSymbol{)}%
\>[14]\AgdaOperator{\AgdaInductiveConstructor{,}}\AgdaSpace{}%
\AgdaBound{p}\AgdaSpace{}%
\AgdaOperator{\AgdaInductiveConstructor{,}}\AgdaSpace{}%
\AgdaBound{ps}\AgdaSymbol{)}%
\>[25]\AgdaOperator{\AgdaFunction{↓′}}\AgdaSpace{}%
\AgdaNumber{1}%
\>[41]\AgdaSymbol{=}\AgdaSpace{}%
\AgdaPostulate{η}\AgdaSpace{}%
\AgdaBound{p}\<%
\\
\>[0]\AgdaSymbol{(}\AgdaInductiveConstructor{suc}\AgdaSpace{}%
\AgdaSymbol{(}\AgdaInductiveConstructor{suc}\AgdaSpace{}%
\AgdaBound{n}\AgdaSymbol{)}%
\>[14]\AgdaOperator{\AgdaInductiveConstructor{,}}\AgdaSpace{}%
\AgdaBound{p}\AgdaSpace{}%
\AgdaOperator{\AgdaInductiveConstructor{,}}\AgdaSpace{}%
\AgdaBound{ps}\AgdaSymbol{)}%
\>[25]\AgdaOperator{\AgdaFunction{↓′}}\AgdaSpace{}%
\AgdaInductiveConstructor{suc}\AgdaSpace{}%
\AgdaSymbol{(}\AgdaInductiveConstructor{suc}\AgdaSpace{}%
\AgdaBound{i}\AgdaSymbol{)}%
\>[41]\AgdaSymbol{=}\AgdaSpace{}%
\AgdaSymbol{(}\AgdaInductiveConstructor{suc}\AgdaSpace{}%
\AgdaBound{n}\AgdaSpace{}%
\AgdaOperator{\AgdaInductiveConstructor{,}}\AgdaSpace{}%
\AgdaBound{ps}\AgdaSymbol{)}\AgdaSpace{}%
\AgdaOperator{\AgdaFunction{↓′}}\AgdaSpace{}%
\AgdaInductiveConstructor{suc}\AgdaSpace{}%
\AgdaBound{i}\<%
\\
\>[0]\AgdaCatchallClause{\AgdaSymbol{(\AgdaUnderscore{}}}%
\>[14]\AgdaCatchallClause{\AgdaOperator{\AgdaInductiveConstructor{,}}}\AgdaSpace{}%
\AgdaCatchallClause{\AgdaSymbol{\AgdaUnderscore{})}}%
\>[25]\AgdaCatchallClause{\AgdaOperator{\AgdaFunction{↓′}}}\AgdaSpace{}%
\AgdaCatchallClause{\AgdaSymbol{\AgdaUnderscore{}}}%
\>[41]\AgdaSymbol{=}\AgdaSpace{}%
\AgdaPostulate{⊥}\<%
\\
%
\\[\AgdaEmptyExtraSkip]%
\>[0]\AgdaOperator{\AgdaFunction{\AgdaUnderscore{}†′\AgdaUnderscore{}}}\AgdaSpace{}%
\AgdaSymbol{:}\AgdaSpace{}%
\AgdaSymbol{∀}\AgdaSpace{}%
\AgdaSymbol{\{}\AgdaBound{P}\AgdaSpace{}%
\AgdaSymbol{:}\AgdaSpace{}%
\AgdaPrimitive{Set}\AgdaSymbol{\}}\AgdaSpace{}%
\AgdaSymbol{→}\AgdaSpace{}%
\AgdaBound{P}\AgdaSpace{}%
\AgdaOperator{\AgdaFunction{*}}\AgdaSpace{}%
\AgdaSymbol{→}\AgdaSpace{}%
\AgdaDatatype{ℕ}\AgdaSpace{}%
\AgdaSymbol{→}\AgdaSpace{}%
\AgdaPostulate{𝕃}\AgdaSpace{}%
\AgdaSymbol{(}\AgdaBound{P}\AgdaSpace{}%
\AgdaOperator{\AgdaFunction{*}}\AgdaSymbol{)}\<%
\\
\>[0]\AgdaSymbol{(}\AgdaNumber{1}%
\>[14]\AgdaOperator{\AgdaInductiveConstructor{,}}\AgdaSpace{}%
\AgdaBound{p}\AgdaSymbol{)}%
\>[25]\AgdaOperator{\AgdaFunction{†′}}\AgdaSpace{}%
\AgdaNumber{1}%
\>[41]\AgdaSymbol{=}\AgdaSpace{}%
\AgdaPostulate{η}\AgdaSpace{}%
\AgdaSymbol{(}\AgdaNumber{0}\AgdaSpace{}%
\AgdaOperator{\AgdaInductiveConstructor{,}}\AgdaSpace{}%
\AgdaInductiveConstructor{[]}\AgdaSymbol{)}\<%
\\
\>[0]\AgdaSymbol{(}\AgdaInductiveConstructor{suc}\AgdaSpace{}%
\AgdaSymbol{(}\AgdaInductiveConstructor{suc}\AgdaSpace{}%
\AgdaBound{n}\AgdaSymbol{)}%
\>[14]\AgdaOperator{\AgdaInductiveConstructor{,}}\AgdaSpace{}%
\AgdaBound{p}\AgdaSpace{}%
\AgdaOperator{\AgdaInductiveConstructor{,}}\AgdaSpace{}%
\AgdaBound{ps}\AgdaSymbol{)}%
\>[25]\AgdaOperator{\AgdaFunction{†′}}\AgdaSpace{}%
\AgdaNumber{1}%
\>[41]\AgdaSymbol{=}\AgdaSpace{}%
\AgdaPostulate{η}\AgdaSpace{}%
\AgdaSymbol{(}\AgdaInductiveConstructor{suc}\AgdaSpace{}%
\AgdaBound{n}\AgdaSpace{}%
\AgdaOperator{\AgdaInductiveConstructor{,}}\AgdaSpace{}%
\AgdaBound{ps}\AgdaSymbol{)}\<%
\\
\>[0]\AgdaSymbol{(}\AgdaInductiveConstructor{suc}\AgdaSpace{}%
\AgdaSymbol{(}\AgdaInductiveConstructor{suc}\AgdaSpace{}%
\AgdaBound{n}\AgdaSymbol{)}%
\>[14]\AgdaOperator{\AgdaInductiveConstructor{,}}\AgdaSpace{}%
\AgdaBound{p}\AgdaSpace{}%
\AgdaOperator{\AgdaInductiveConstructor{,}}\AgdaSpace{}%
\AgdaBound{ps}\AgdaSymbol{)}%
\>[25]\AgdaOperator{\AgdaFunction{†′}}\AgdaSpace{}%
\AgdaInductiveConstructor{suc}\AgdaSpace{}%
\AgdaSymbol{(}\AgdaInductiveConstructor{suc}\AgdaSpace{}%
\AgdaBound{i}\AgdaSymbol{)}%
\>[41]\AgdaSymbol{=}\AgdaSpace{}%
\AgdaSymbol{(}\AgdaInductiveConstructor{suc}\AgdaSpace{}%
\AgdaBound{n}\AgdaSpace{}%
\AgdaOperator{\AgdaInductiveConstructor{,}}\AgdaSpace{}%
\AgdaBound{ps}\AgdaSymbol{)}\AgdaSpace{}%
\AgdaOperator{\AgdaFunction{†′}}\AgdaSpace{}%
\AgdaInductiveConstructor{suc}\AgdaSpace{}%
\AgdaBound{i}\<%
\\
\>[0]\AgdaCatchallClause{\AgdaSymbol{(\AgdaUnderscore{}}}%
\>[14]\AgdaCatchallClause{\AgdaOperator{\AgdaInductiveConstructor{,}}}\AgdaSpace{}%
\AgdaCatchallClause{\AgdaSymbol{\AgdaUnderscore{})}}%
\>[25]\AgdaCatchallClause{\AgdaOperator{\AgdaFunction{†′}}}\AgdaSpace{}%
\AgdaCatchallClause{\AgdaSymbol{\AgdaUnderscore{}}}%
\>[41]\AgdaSymbol{=}\AgdaSpace{}%
\AgdaPostulate{⊥}\<%
\\
%
\\[\AgdaEmptyExtraSkip]%
\>[0]\AgdaOperator{\AgdaFunction{\AgdaUnderscore{}§′\AgdaUnderscore{}}}\AgdaSpace{}%
\AgdaSymbol{:}\AgdaSpace{}%
\AgdaSymbol{∀}\AgdaSpace{}%
\AgdaSymbol{\{}\AgdaBound{P}\AgdaSpace{}%
\AgdaSymbol{:}\AgdaSpace{}%
\AgdaPrimitive{Set}\AgdaSymbol{\}}\AgdaSpace{}%
\AgdaSymbol{→}\AgdaSpace{}%
\AgdaBound{P}\AgdaSpace{}%
\AgdaOperator{\AgdaFunction{*}}\AgdaSpace{}%
\AgdaSymbol{→}\AgdaSpace{}%
\AgdaBound{P}\AgdaSpace{}%
\AgdaOperator{\AgdaFunction{*}}\AgdaSpace{}%
\AgdaSymbol{→}\AgdaSpace{}%
\AgdaBound{P}\AgdaSpace{}%
\AgdaOperator{\AgdaFunction{*}}\<%
\\
\>[0]\AgdaSymbol{(}\AgdaBound{m}\AgdaSpace{}%
\AgdaOperator{\AgdaInductiveConstructor{,}}\AgdaSpace{}%
\AgdaBound{pm}\AgdaSymbol{)}\AgdaSpace{}%
\AgdaOperator{\AgdaFunction{§′}}\AgdaSpace{}%
\AgdaSymbol{(}\AgdaBound{n}\AgdaSpace{}%
\AgdaOperator{\AgdaInductiveConstructor{,}}\AgdaSpace{}%
\AgdaBound{pn}\AgdaSymbol{)}\AgdaSpace{}%
\AgdaSymbol{=}\AgdaSpace{}%
\AgdaSymbol{((}\AgdaBound{m}\AgdaSpace{}%
\AgdaOperator{\AgdaPrimitive{Nat.+}}\AgdaSpace{}%
\AgdaBound{n}\AgdaSymbol{)}\AgdaSpace{}%
\AgdaOperator{\AgdaInductiveConstructor{,}}\AgdaSpace{}%
\AgdaFunction{append}\AgdaSpace{}%
\AgdaBound{m}\AgdaSpace{}%
\AgdaBound{n}\AgdaSpace{}%
\AgdaBound{pm}\AgdaSpace{}%
\AgdaBound{pn}\AgdaSymbol{)}\<%
\\
%
\\[\AgdaEmptyExtraSkip]%
\>[0]\AgdaComment{--\ Sequence\ domains}\<%
\\
\>[0]\AgdaComment{--\ D\ ⋆\ =\ 𝕃\ ((D\ \textasciicircum{}\ 0)\ +\ ...\ +\ (D\ \textasciicircum{}\ n)\ +\ ...)}\<%
\\
%
\\[\AgdaEmptyExtraSkip]%
\>[0]\AgdaOperator{\AgdaFunction{\AgdaUnderscore{}⋆}}%
\>[5]\AgdaSymbol{:}\AgdaSpace{}%
\AgdaFunction{Domain}\AgdaSpace{}%
\AgdaSymbol{→}\AgdaSpace{}%
\AgdaFunction{Domain}\<%
\\
\>[0]\AgdaBound{D}\AgdaSpace{}%
\AgdaOperator{\AgdaFunction{⋆}}%
\>[5]\AgdaSymbol{=}\AgdaSpace{}%
\AgdaPostulate{𝕃}\AgdaSpace{}%
\AgdaSymbol{(}\AgdaRecord{Σ}\AgdaSpace{}%
\AgdaDatatype{ℕ}\AgdaSpace{}%
\AgdaSymbol{(}\AgdaBound{D}\AgdaSpace{}%
\AgdaOperator{\AgdaFunction{\textasciicircum{}\AgdaUnderscore{}}}\AgdaSymbol{))}\<%
\\
%
\\[\AgdaEmptyExtraSkip]%
\>[0]\AgdaComment{--\ ⟨⟩\ :\ D\ ⋆}\<%
\\
%
\\[\AgdaEmptyExtraSkip]%
\>[0]\AgdaFunction{⟨⟩}\AgdaSpace{}%
\AgdaSymbol{:}\AgdaSpace{}%
\AgdaSymbol{∀}\AgdaSpace{}%
\AgdaSymbol{\{}\AgdaBound{D}\AgdaSymbol{\}}\AgdaSpace{}%
\AgdaSymbol{→}\AgdaSpace{}%
\AgdaBound{D}\AgdaSpace{}%
\AgdaOperator{\AgdaFunction{⋆}}\<%
\\
\>[0]\AgdaFunction{⟨⟩}\AgdaSpace{}%
\AgdaSymbol{=}\AgdaSpace{}%
\AgdaPostulate{η}\AgdaSpace{}%
\AgdaSymbol{(}\AgdaNumber{0}\AgdaSpace{}%
\AgdaOperator{\AgdaInductiveConstructor{,}}\AgdaSpace{}%
\AgdaInductiveConstructor{[]}\AgdaSymbol{)}\<%
\\
%
\\[\AgdaEmptyExtraSkip]%
\>[0]\AgdaComment{--\ ⟨\ d₁\ ,\ ...\ ,\ dₙ\ ⟩\ :\ D\ ⋆}\<%
\\
%
\\[\AgdaEmptyExtraSkip]%
\>[0]\AgdaOperator{\AgdaFunction{⟨\AgdaUnderscore{}⟩}}\AgdaSpace{}%
\AgdaSymbol{:}\AgdaSpace{}%
\AgdaSymbol{∀}\AgdaSpace{}%
\AgdaSymbol{\{}\AgdaBound{n}\AgdaSpace{}%
\AgdaBound{D}\AgdaSymbol{\}}\AgdaSpace{}%
\AgdaSymbol{→}\AgdaSpace{}%
\AgdaBound{D}\AgdaSpace{}%
\AgdaOperator{\AgdaFunction{\textasciicircum{}}}\AgdaSpace{}%
\AgdaInductiveConstructor{suc}\AgdaSpace{}%
\AgdaBound{n}\AgdaSpace{}%
\AgdaSymbol{→}\AgdaSpace{}%
\AgdaBound{D}\AgdaSpace{}%
\AgdaOperator{\AgdaFunction{⋆}}\<%
\\
\>[0]\AgdaOperator{\AgdaFunction{⟨\AgdaUnderscore{}⟩}}\AgdaSpace{}%
\AgdaSymbol{\{}\AgdaArgument{n}\AgdaSpace{}%
\AgdaSymbol{=}\AgdaSpace{}%
\AgdaBound{n}\AgdaSymbol{\}}\AgdaSpace{}%
\AgdaBound{ds}\AgdaSpace{}%
\AgdaSymbol{=}\AgdaSpace{}%
\AgdaPostulate{η}\AgdaSpace{}%
\AgdaSymbol{(}\AgdaInductiveConstructor{suc}\AgdaSpace{}%
\AgdaBound{n}\AgdaSpace{}%
\AgdaOperator{\AgdaInductiveConstructor{,}}\AgdaSpace{}%
\AgdaBound{ds}\AgdaSymbol{)}\<%
\end{code}
\clearpage
\begin{code}%
\>[0]\AgdaComment{--\ \#\ D\ ⋆\ :\ ℕ\ +⊥}\<%
\\
%
\\[\AgdaEmptyExtraSkip]%
\>[0]\AgdaFunction{\#}\AgdaSpace{}%
\AgdaSymbol{:}\AgdaSpace{}%
\AgdaSymbol{∀}\AgdaSpace{}%
\AgdaSymbol{\{}\AgdaBound{D}\AgdaSymbol{\}}\AgdaSpace{}%
\AgdaSymbol{→}\AgdaSpace{}%
\AgdaBound{D}\AgdaSpace{}%
\AgdaOperator{\AgdaFunction{⋆}}\AgdaSpace{}%
\AgdaSymbol{→}\AgdaSpace{}%
\AgdaDatatype{ℕ}\AgdaSpace{}%
\AgdaOperator{\AgdaFunction{+⊥}}\<%
\\
\>[0]\AgdaFunction{\#}\AgdaSpace{}%
\AgdaBound{d⋆}\AgdaSpace{}%
\AgdaSymbol{=}\AgdaSpace{}%
\AgdaSymbol{((λ}\AgdaSpace{}%
\AgdaBound{p*}\AgdaSpace{}%
\AgdaSymbol{→}\AgdaSpace{}%
\AgdaPostulate{η}\AgdaSpace{}%
\AgdaSymbol{(}\AgdaFunction{\#′}\AgdaSpace{}%
\AgdaBound{p*}\AgdaSymbol{))}\AgdaSpace{}%
\AgdaOperator{\AgdaPostulate{♯}}\AgdaSymbol{)}\AgdaSpace{}%
\AgdaBound{d⋆}\<%
\\
%
\\[\AgdaEmptyExtraSkip]%
\>[0]\AgdaComment{--\ d⋆₁\ §\ d⋆₂\ :\ D\ ⋆}\<%
\\
%
\\[\AgdaEmptyExtraSkip]%
\>[0]\AgdaOperator{\AgdaFunction{\AgdaUnderscore{}§\AgdaUnderscore{}}}\AgdaSpace{}%
\AgdaSymbol{:}\AgdaSpace{}%
\AgdaSymbol{∀}\AgdaSpace{}%
\AgdaSymbol{\{}\AgdaBound{D}\AgdaSymbol{\}}\AgdaSpace{}%
\AgdaSymbol{→}\AgdaSpace{}%
\AgdaBound{D}\AgdaSpace{}%
\AgdaOperator{\AgdaFunction{⋆}}\AgdaSpace{}%
\AgdaSymbol{→}\AgdaSpace{}%
\AgdaBound{D}\AgdaSpace{}%
\AgdaOperator{\AgdaFunction{⋆}}\AgdaSpace{}%
\AgdaSymbol{→}\AgdaSpace{}%
\AgdaBound{D}\AgdaSpace{}%
\AgdaOperator{\AgdaFunction{⋆}}\<%
\\
\>[0]\AgdaBound{d⋆₁}\AgdaSpace{}%
\AgdaOperator{\AgdaFunction{§}}\AgdaSpace{}%
\AgdaBound{d⋆₂}\AgdaSpace{}%
\AgdaSymbol{=}\AgdaSpace{}%
\AgdaSymbol{((λ}\AgdaSpace{}%
\AgdaBound{p*₁}\AgdaSpace{}%
\AgdaSymbol{→}\AgdaSpace{}%
\AgdaSymbol{((λ}\AgdaSpace{}%
\AgdaBound{p*₂}\AgdaSpace{}%
\AgdaSymbol{→}\AgdaSpace{}%
\AgdaPostulate{η}\AgdaSpace{}%
\AgdaSymbol{(}\AgdaBound{p*₁}\AgdaSpace{}%
\AgdaOperator{\AgdaFunction{§′}}\AgdaSpace{}%
\AgdaBound{p*₂}\AgdaSymbol{))}\AgdaSpace{}%
\AgdaOperator{\AgdaPostulate{♯}}\AgdaSymbol{)}\AgdaSpace{}%
\AgdaBound{d⋆₂}\AgdaSymbol{)}\AgdaSpace{}%
\AgdaOperator{\AgdaPostulate{♯}}\AgdaSymbol{)}\AgdaSpace{}%
\AgdaBound{d⋆₁}\<%
\\
%
\\[\AgdaEmptyExtraSkip]%
\>[0]\AgdaKeyword{open}\AgdaSpace{}%
\AgdaKeyword{import}\AgdaSpace{}%
\AgdaModule{Function}\<%
\\
\>[0][@{}l@{\AgdaIndent{0}}]%
\>[2]\AgdaKeyword{using}\AgdaSpace{}%
\AgdaSymbol{(}\AgdaFunction{id}\AgdaSymbol{;}\AgdaSpace{}%
\AgdaOperator{\AgdaFunction{\AgdaUnderscore{}∘\AgdaUnderscore{}}}\AgdaSymbol{)}\AgdaSpace{}%
\AgdaKeyword{public}\<%
\\
%
\\[\AgdaEmptyExtraSkip]%
\>[0]\AgdaComment{--\ d⋆\ ↓\ k\ :\ 𝕃\ D\ \ (k\ ≥\ 1;\ k\ <\ \#\ d⋆)}\<%
\\
%
\\[\AgdaEmptyExtraSkip]%
\>[0]\AgdaOperator{\AgdaFunction{\AgdaUnderscore{}↓\AgdaUnderscore{}}}\AgdaSpace{}%
\AgdaSymbol{:}\AgdaSpace{}%
\AgdaSymbol{∀}\AgdaSpace{}%
\AgdaSymbol{\{}\AgdaBound{D}\AgdaSymbol{\}}\AgdaSpace{}%
\AgdaSymbol{→}\AgdaSpace{}%
\AgdaBound{D}\AgdaSpace{}%
\AgdaOperator{\AgdaFunction{⋆}}\AgdaSpace{}%
\AgdaSymbol{→}\AgdaSpace{}%
\AgdaDatatype{ℕ}\AgdaSpace{}%
\AgdaSymbol{→}\AgdaSpace{}%
\AgdaBound{D}\<%
\\
\>[0]\AgdaBound{d⋆}\AgdaSpace{}%
\AgdaOperator{\AgdaFunction{↓}}\AgdaSpace{}%
\AgdaBound{n}\AgdaSpace{}%
\AgdaSymbol{=}\AgdaSpace{}%
\AgdaSymbol{(}\AgdaFunction{id}\AgdaSpace{}%
\AgdaOperator{\AgdaPostulate{♯}}\AgdaSymbol{)}\AgdaSpace{}%
\AgdaSymbol{(((λ}\AgdaSpace{}%
\AgdaBound{p*}\AgdaSpace{}%
\AgdaSymbol{→}\AgdaSpace{}%
\AgdaBound{p*}\AgdaSpace{}%
\AgdaOperator{\AgdaFunction{↓′}}\AgdaSpace{}%
\AgdaBound{n}\AgdaSymbol{)}\AgdaSpace{}%
\AgdaOperator{\AgdaPostulate{♯}}\AgdaSymbol{)}\AgdaSpace{}%
\AgdaBound{d⋆}\AgdaSymbol{)}\<%
\\
%
\\[\AgdaEmptyExtraSkip]%
\>[0]\AgdaComment{--\ d⋆\ †\ k\ :\ D\ ⋆\ \ (k\ ≥\ 1)}\<%
\\
%
\\[\AgdaEmptyExtraSkip]%
\>[0]\AgdaOperator{\AgdaFunction{\AgdaUnderscore{}†\AgdaUnderscore{}}}\AgdaSpace{}%
\AgdaSymbol{:}\AgdaSpace{}%
\AgdaSymbol{∀}\AgdaSpace{}%
\AgdaSymbol{\{}\AgdaBound{D}\AgdaSymbol{\}}\AgdaSpace{}%
\AgdaSymbol{→}\AgdaSpace{}%
\AgdaBound{D}\AgdaSpace{}%
\AgdaOperator{\AgdaFunction{⋆}}\AgdaSpace{}%
\AgdaSymbol{→}\AgdaSpace{}%
\AgdaDatatype{ℕ}\AgdaSpace{}%
\AgdaSymbol{→}\AgdaSpace{}%
\AgdaBound{D}\AgdaSpace{}%
\AgdaOperator{\AgdaFunction{⋆}}\<%
\\
\>[0]\AgdaBound{d⋆}\AgdaSpace{}%
\AgdaOperator{\AgdaFunction{†}}\AgdaSpace{}%
\AgdaBound{n}\AgdaSpace{}%
\AgdaSymbol{=}\AgdaSpace{}%
\AgdaSymbol{(}\AgdaFunction{id}\AgdaSpace{}%
\AgdaOperator{\AgdaPostulate{♯}}\AgdaSymbol{)}\AgdaSpace{}%
\AgdaSymbol{(((λ}\AgdaSpace{}%
\AgdaBound{p*}\AgdaSpace{}%
\AgdaSymbol{→}\AgdaSpace{}%
\AgdaPostulate{η}\AgdaSpace{}%
\AgdaSymbol{(}\AgdaBound{p*}\AgdaSpace{}%
\AgdaOperator{\AgdaFunction{†′}}\AgdaSpace{}%
\AgdaBound{n}\AgdaSymbol{))}\AgdaSpace{}%
\AgdaOperator{\AgdaPostulate{♯}}\AgdaSymbol{)}\AgdaSpace{}%
\AgdaBound{d⋆}\AgdaSymbol{)}\<%
\\
%
\\[\AgdaEmptyExtraSkip]%
\>[0]\AgdaComment{------------------------------------------------------------------------}\<%
\\
\>[0]\AgdaComment{--\ McCarthy\ conditional}\<%
\\
%
\\[\AgdaEmptyExtraSkip]%
\>[0]\AgdaComment{--\ t\ ⟶\ d₁\ ,\ d₂\ :\ D\ \ (t\ :\ Bool\ +⊥\ ;\ d₁,\ d₂\ :\ D)}\<%
\\
%
\\[\AgdaEmptyExtraSkip]%
\>[0]\AgdaKeyword{open}\AgdaSpace{}%
\AgdaKeyword{import}\AgdaSpace{}%
\AgdaModule{Data.Bool.Base}\<%
\\
\>[0][@{}l@{\AgdaIndent{0}}]%
\>[2]\AgdaKeyword{using}\AgdaSpace{}%
\AgdaSymbol{(}\AgdaDatatype{Bool}\AgdaSymbol{;}\AgdaSpace{}%
\AgdaInductiveConstructor{true}\AgdaSymbol{;}\AgdaSpace{}%
\AgdaInductiveConstructor{false}\AgdaSymbol{;}\AgdaSpace{}%
\AgdaOperator{\AgdaFunction{if\AgdaUnderscore{}then\AgdaUnderscore{}else\AgdaUnderscore{}}}\AgdaSymbol{)}\AgdaSpace{}%
\AgdaKeyword{public}\<%
\\
%
\\[\AgdaEmptyExtraSkip]%
\>[0]\AgdaKeyword{postulate}\<%
\\
\>[0][@{}l@{\AgdaIndent{0}}]%
\>[2]\AgdaOperator{\AgdaPostulate{\AgdaUnderscore{}⟶\AgdaUnderscore{},\AgdaUnderscore{}}}\AgdaSpace{}%
\AgdaSymbol{:}\AgdaSpace{}%
\AgdaSymbol{\{}\AgdaBound{D}\AgdaSpace{}%
\AgdaSymbol{:}\AgdaSpace{}%
\AgdaFunction{Domain}\AgdaSymbol{\}}\AgdaSpace{}%
\AgdaSymbol{→}\AgdaSpace{}%
\AgdaDatatype{Bool}\AgdaSpace{}%
\AgdaOperator{\AgdaFunction{+⊥}}\AgdaSpace{}%
\AgdaSymbol{→}\AgdaSpace{}%
\AgdaBound{D}\AgdaSpace{}%
\AgdaSymbol{→}\AgdaSpace{}%
\AgdaBound{D}\AgdaSpace{}%
\AgdaSymbol{→}\AgdaSpace{}%
\AgdaBound{D}\<%
\\
%
\\[\AgdaEmptyExtraSkip]%
%
\>[2]\AgdaComment{--\ Properties}\<%
\\
%
\>[2]\AgdaPostulate{true-cond}%
\>[15]\AgdaSymbol{:}\AgdaSpace{}%
\AgdaSymbol{∀}\AgdaSpace{}%
\AgdaSymbol{\{}\AgdaBound{D}\AgdaSymbol{\}}\AgdaSpace{}%
\AgdaSymbol{\{}\AgdaBound{d₁}\AgdaSpace{}%
\AgdaBound{d₂}\AgdaSpace{}%
\AgdaSymbol{:}\AgdaSpace{}%
\AgdaBound{D}\AgdaSymbol{\}}\AgdaSpace{}%
\AgdaSymbol{→}\AgdaSpace{}%
\AgdaSymbol{(}\AgdaPostulate{η}\AgdaSpace{}%
\AgdaInductiveConstructor{true}\AgdaSpace{}%
\AgdaOperator{\AgdaPostulate{⟶}}\AgdaSpace{}%
\AgdaBound{d₁}\AgdaSpace{}%
\AgdaOperator{\AgdaPostulate{,}}\AgdaSpace{}%
\AgdaBound{d₂}\AgdaSymbol{)}%
\>[57]\AgdaOperator{\AgdaDatatype{≡}}\AgdaSpace{}%
\AgdaBound{d₁}\<%
\\
%
\>[2]\AgdaPostulate{false-cond}%
\>[15]\AgdaSymbol{:}\AgdaSpace{}%
\AgdaSymbol{∀}\AgdaSpace{}%
\AgdaSymbol{\{}\AgdaBound{D}\AgdaSymbol{\}}\AgdaSpace{}%
\AgdaSymbol{\{}\AgdaBound{d₁}\AgdaSpace{}%
\AgdaBound{d₂}\AgdaSpace{}%
\AgdaSymbol{:}\AgdaSpace{}%
\AgdaBound{D}\AgdaSymbol{\}}\AgdaSpace{}%
\AgdaSymbol{→}\AgdaSpace{}%
\AgdaSymbol{(}\AgdaPostulate{η}\AgdaSpace{}%
\AgdaInductiveConstructor{false}\AgdaSpace{}%
\AgdaOperator{\AgdaPostulate{⟶}}\AgdaSpace{}%
\AgdaBound{d₁}\AgdaSpace{}%
\AgdaOperator{\AgdaPostulate{,}}\AgdaSpace{}%
\AgdaBound{d₂}\AgdaSymbol{)}\AgdaSpace{}%
\AgdaOperator{\AgdaDatatype{≡}}\AgdaSpace{}%
\AgdaBound{d₂}\<%
\\
%
\>[2]\AgdaPostulate{bottom-cond}%
\>[15]\AgdaSymbol{:}\AgdaSpace{}%
\AgdaSymbol{∀}\AgdaSpace{}%
\AgdaSymbol{\{}\AgdaBound{D}\AgdaSymbol{\}}\AgdaSpace{}%
\AgdaSymbol{\{}\AgdaBound{d₁}\AgdaSpace{}%
\AgdaBound{d₂}\AgdaSpace{}%
\AgdaSymbol{:}\AgdaSpace{}%
\AgdaBound{D}\AgdaSymbol{\}}\AgdaSpace{}%
\AgdaSymbol{→}\AgdaSpace{}%
\AgdaSymbol{(}\AgdaPostulate{⊥}\AgdaSpace{}%
\AgdaOperator{\AgdaPostulate{⟶}}\AgdaSpace{}%
\AgdaBound{d₁}\AgdaSpace{}%
\AgdaOperator{\AgdaPostulate{,}}\AgdaSpace{}%
\AgdaBound{d₂}\AgdaSymbol{)}%
\>[57]\AgdaOperator{\AgdaDatatype{≡}}\AgdaSpace{}%
\AgdaPostulate{⊥}\<%
\\
%
\\[\AgdaEmptyExtraSkip]%
\>[0]\AgdaComment{------------------------------------------------------------------------}\<%
\\
\>[0]\AgdaComment{--\ Meta-Strings}\<%
\\
%
\\[\AgdaEmptyExtraSkip]%
\>[0]\AgdaKeyword{open}\AgdaSpace{}%
\AgdaKeyword{import}\AgdaSpace{}%
\AgdaModule{Data.String.Base}\<%
\\
\>[0][@{}l@{\AgdaIndent{0}}]%
\>[2]\AgdaKeyword{using}\AgdaSpace{}%
\AgdaSymbol{(}\AgdaPostulate{String}\AgdaSymbol{)}\AgdaSpace{}%
\AgdaKeyword{public}\<%
\\
\>[0]\<%
\end{code}  

\clearpage

\begin{code}%
\>[0]\<%
\\
\>[0]\AgdaKeyword{module}\AgdaSpace{}%
\AgdaModule{Scheme.Abstract-Syntax}\AgdaSpace{}%
\AgdaKeyword{where}\<%
\\
%
\\[\AgdaEmptyExtraSkip]%
\>[0]\AgdaKeyword{open}\AgdaSpace{}%
\AgdaKeyword{import}\AgdaSpace{}%
\AgdaModule{Scheme.Domain-Notation}\AgdaSpace{}%
\AgdaKeyword{using}\AgdaSpace{}%
\AgdaSymbol{(}\AgdaOperator{\AgdaFunction{\AgdaUnderscore{}*}}\AgdaSymbol{)}\<%
\\
%
\\[\AgdaEmptyExtraSkip]%
\>[0]\AgdaComment{--\ 7.2.1.\ Abstract\ syntax}\<%
\\
%
\\[\AgdaEmptyExtraSkip]%
\>[0]\AgdaKeyword{postulate}\AgdaSpace{}%
\AgdaPostulate{Con}%
\>[15]\AgdaSymbol{:}\AgdaSpace{}%
\AgdaPrimitive{Set}%
\>[23]\AgdaComment{--\ constants,\ including\ quotations}\<%
\\
\>[0]\AgdaKeyword{postulate}\AgdaSpace{}%
\AgdaPostulate{Ide}%
\>[15]\AgdaSymbol{:}\AgdaSpace{}%
\AgdaPrimitive{Set}%
\>[23]\AgdaComment{--\ identifiers\ (variables)}\<%
\\
\>[0]\AgdaKeyword{data}%
\>[10]\AgdaDatatype{Exp}%
\>[15]\AgdaSymbol{:}\AgdaSpace{}%
\AgdaPrimitive{Set}%
\>[23]\AgdaComment{--\ expressions}\<%
\\
\>[0]\AgdaFunction{Com}%
\>[15]\AgdaSymbol{=}\AgdaSpace{}%
\AgdaDatatype{Exp}%
\>[23]\AgdaComment{--\ commands}\<%
\\
%
\\[\AgdaEmptyExtraSkip]%
\>[0]\AgdaKeyword{data}\AgdaSpace{}%
\AgdaDatatype{Exp}\AgdaSpace{}%
\AgdaKeyword{where}\<%
\\
\>[0][@{}l@{\AgdaIndent{0}}]%
\>[2]\AgdaInductiveConstructor{con}%
\>[21]\AgdaSymbol{:}\AgdaSpace{}%
\AgdaPostulate{Con}\AgdaSpace{}%
\AgdaSymbol{→}\AgdaSpace{}%
\AgdaDatatype{Exp}%
\>[56]\AgdaComment{--\ K}\<%
\\
%
\>[2]\AgdaInductiveConstructor{ide}%
\>[21]\AgdaSymbol{:}\AgdaSpace{}%
\AgdaPostulate{Ide}\AgdaSpace{}%
\AgdaSymbol{→}\AgdaSpace{}%
\AgdaDatatype{Exp}%
\>[56]\AgdaComment{--\ I}\<%
\\
%
\>[2]\AgdaOperator{\AgdaInductiveConstructor{⦅\AgdaUnderscore{}⦆}}%
\>[21]\AgdaSymbol{:}\AgdaSpace{}%
\AgdaDatatype{Exp}\AgdaSpace{}%
\AgdaOperator{\AgdaFunction{*}}\AgdaSpace{}%
\AgdaSymbol{→}\AgdaSpace{}%
\AgdaDatatype{Exp}%
\>[56]\AgdaComment{--\ (E₀\ E*)}\<%
\\
%
\>[2]\AgdaOperator{\AgdaInductiveConstructor{⦅lambda␣⦅\AgdaUnderscore{}⦆\AgdaUnderscore{}␣\AgdaUnderscore{}⦆}}%
\>[21]\AgdaSymbol{:}\AgdaSpace{}%
\AgdaPostulate{Ide}\AgdaSpace{}%
\AgdaOperator{\AgdaFunction{*}}\AgdaSpace{}%
\AgdaSymbol{→}\AgdaSpace{}%
\AgdaFunction{Com}\AgdaSpace{}%
\AgdaOperator{\AgdaFunction{*}}\AgdaSpace{}%
\AgdaSymbol{→}\AgdaSpace{}%
\AgdaDatatype{Exp}\AgdaSpace{}%
\AgdaSymbol{→}\AgdaSpace{}%
\AgdaDatatype{Exp}%
\>[56]\AgdaComment{--\ (lambda\ (I*)\ Γ*\ E₀)}\<%
\\
%
\>[2]\AgdaOperator{\AgdaInductiveConstructor{⦅lambda␣⦅\AgdaUnderscore{}·\AgdaUnderscore{}⦆\AgdaUnderscore{}␣\AgdaUnderscore{}⦆}}%
\>[21]\AgdaSymbol{:}\AgdaSpace{}%
\AgdaPostulate{Ide}\AgdaSpace{}%
\AgdaOperator{\AgdaFunction{*}}\AgdaSpace{}%
\AgdaSymbol{→}\AgdaSpace{}%
\AgdaPostulate{Ide}\AgdaSpace{}%
\AgdaSymbol{→}\AgdaSpace{}%
\AgdaFunction{Com}\AgdaSpace{}%
\AgdaOperator{\AgdaFunction{*}}\AgdaSpace{}%
\AgdaSymbol{→}\AgdaSpace{}%
\AgdaDatatype{Exp}\AgdaSpace{}%
\AgdaSymbol{→}\AgdaSpace{}%
\AgdaDatatype{Exp}%
\>[56]\AgdaComment{--\ (lambda\ (I*\ .\ I)\ Γ*\ E₀)}\<%
\\
%
\>[2]\AgdaOperator{\AgdaInductiveConstructor{⦅lambda\AgdaUnderscore{}␣\AgdaUnderscore{}␣\AgdaUnderscore{}⦆}}%
\>[21]\AgdaSymbol{:}\AgdaSpace{}%
\AgdaPostulate{Ide}\AgdaSpace{}%
\AgdaSymbol{→}\AgdaSpace{}%
\AgdaFunction{Com}\AgdaSpace{}%
\AgdaOperator{\AgdaFunction{*}}\AgdaSpace{}%
\AgdaSymbol{→}\AgdaSpace{}%
\AgdaDatatype{Exp}\AgdaSpace{}%
\AgdaSymbol{→}\AgdaSpace{}%
\AgdaDatatype{Exp}%
\>[56]\AgdaComment{--\ (lambda\ I\ Γ*\ E₀)}\<%
\\
%
\>[2]\AgdaOperator{\AgdaInductiveConstructor{⦅if\AgdaUnderscore{}␣\AgdaUnderscore{}␣\AgdaUnderscore{}⦆}}%
\>[21]\AgdaSymbol{:}\AgdaSpace{}%
\AgdaDatatype{Exp}\AgdaSpace{}%
\AgdaSymbol{→}\AgdaSpace{}%
\AgdaDatatype{Exp}\AgdaSpace{}%
\AgdaSymbol{→}\AgdaSpace{}%
\AgdaDatatype{Exp}\AgdaSpace{}%
\AgdaSymbol{→}\AgdaSpace{}%
\AgdaDatatype{Exp}%
\>[56]\AgdaComment{--\ (if\ E₀\ E₁\ E₂)}\<%
\\
%
\>[2]\AgdaOperator{\AgdaInductiveConstructor{⦅if\AgdaUnderscore{}␣\AgdaUnderscore{}⦆}}%
\>[21]\AgdaSymbol{:}\AgdaSpace{}%
\AgdaDatatype{Exp}\AgdaSpace{}%
\AgdaSymbol{→}\AgdaSpace{}%
\AgdaDatatype{Exp}\AgdaSpace{}%
\AgdaSymbol{→}\AgdaSpace{}%
\AgdaDatatype{Exp}%
\>[56]\AgdaComment{--\ (if\ E₀\ E₁)}\<%
\\
%
\>[2]\AgdaOperator{\AgdaInductiveConstructor{⦅set!\AgdaUnderscore{}␣\AgdaUnderscore{}⦆}}%
\>[21]\AgdaSymbol{:}\AgdaSpace{}%
\AgdaPostulate{Ide}\AgdaSpace{}%
\AgdaSymbol{→}\AgdaSpace{}%
\AgdaDatatype{Exp}\AgdaSpace{}%
\AgdaSymbol{→}\AgdaSpace{}%
\AgdaDatatype{Exp}%
\>[56]\AgdaComment{--\ (set!\ I\ E)}\<%
\\
%
\\[\AgdaEmptyExtraSkip]%
\>[0]\AgdaKeyword{variable}\<%
\\
\>[0][@{}l@{\AgdaIndent{0}}]%
\>[2]\AgdaGeneralizable{K}%
\>[6]\AgdaSymbol{:}\AgdaSpace{}%
\AgdaPostulate{Con}\<%
\\
%
\>[2]\AgdaGeneralizable{I}%
\>[6]\AgdaSymbol{:}\AgdaSpace{}%
\AgdaPostulate{Ide}\<%
\\
%
\>[2]\AgdaGeneralizable{I*}%
\>[6]\AgdaSymbol{:}\AgdaSpace{}%
\AgdaPostulate{Ide}\AgdaSpace{}%
\AgdaOperator{\AgdaFunction{*}}\<%
\\
%
\>[2]\AgdaGeneralizable{E}%
\>[6]\AgdaSymbol{:}\AgdaSpace{}%
\AgdaDatatype{Exp}\<%
\\
%
\>[2]\AgdaGeneralizable{E*}%
\>[6]\AgdaSymbol{:}\AgdaSpace{}%
\AgdaDatatype{Exp}\AgdaSpace{}%
\AgdaOperator{\AgdaFunction{*}}\<%
\\
%
\>[2]\AgdaGeneralizable{Γ}%
\>[6]\AgdaSymbol{:}\AgdaSpace{}%
\AgdaFunction{Com}\<%
\\
%
\>[2]\AgdaGeneralizable{Γ*}%
\>[6]\AgdaSymbol{:}\AgdaSpace{}%
\AgdaFunction{Com}\AgdaSpace{}%
\AgdaOperator{\AgdaFunction{*}}\<%
\\
\>[0]\<%
\end{code}

\clearpage

\begin{code}%
\>[0]\AgdaKeyword{module}\AgdaSpace{}%
\AgdaModule{Scheme.Domain-Equations}\AgdaSpace{}%
\AgdaKeyword{where}\<%
\\
%
\\[\AgdaEmptyExtraSkip]%
\>[0]\AgdaKeyword{open}\AgdaSpace{}%
\AgdaKeyword{import}\AgdaSpace{}%
\AgdaModule{Scheme.Domain-Notation}\<%
\\
\>[0]\AgdaKeyword{open}\AgdaSpace{}%
\AgdaKeyword{import}\AgdaSpace{}%
\AgdaModule{Scheme.Abstract-Syntax}\<%
\\
\>[0][@{}l@{\AgdaIndent{0}}]%
\>[2]\AgdaKeyword{using}\AgdaSpace{}%
\AgdaSymbol{(}\AgdaPostulate{Ide}\AgdaSymbol{)}\<%
\\
%
\\[\AgdaEmptyExtraSkip]%
\>[0]\AgdaComment{--\ 7.2.2.\ Domain\ equations}\<%
\\
%
\\[\AgdaEmptyExtraSkip]%
\>[0]\AgdaComment{--\ Domain\ definitions}\<%
\\
%
\\[\AgdaEmptyExtraSkip]%
\>[0]\AgdaKeyword{postulate}\AgdaSpace{}%
\AgdaPostulate{Loc}%
\>[15]\AgdaSymbol{:}%
\>[18]\AgdaPrimitive{Set}\<%
\\
\>[0]\AgdaFunction{𝐋}%
\>[15]\AgdaSymbol{=}%
\>[18]\AgdaPostulate{Loc}\AgdaSpace{}%
\AgdaOperator{\AgdaFunction{+⊥}}%
\>[31]\AgdaComment{--\ locations}\<%
\\
\>[0]\AgdaFunction{𝐍}%
\>[15]\AgdaSymbol{=}%
\>[18]\AgdaDatatype{ℕ}\AgdaSpace{}%
\AgdaOperator{\AgdaFunction{+⊥}}%
\>[31]\AgdaComment{--\ natural\ numbers}\<%
\\
\>[0]\AgdaFunction{𝐓}%
\>[15]\AgdaSymbol{=}%
\>[18]\AgdaDatatype{Bool}\AgdaSpace{}%
\AgdaOperator{\AgdaFunction{+⊥}}%
\>[31]\AgdaComment{--\ booleans}\<%
\\
\>[0]\AgdaKeyword{postulate}\AgdaSpace{}%
\AgdaPostulate{𝐐}%
\>[15]\AgdaSymbol{:}%
\>[18]\AgdaFunction{Domain}%
\>[31]\AgdaComment{--\ symbols}\<%
\\
\>[0]\AgdaKeyword{postulate}\AgdaSpace{}%
\AgdaPostulate{𝐇}%
\>[15]\AgdaSymbol{:}%
\>[18]\AgdaFunction{Domain}%
\>[31]\AgdaComment{--\ characters}\<%
\\
\>[0]\AgdaKeyword{postulate}\AgdaSpace{}%
\AgdaPostulate{𝐑}%
\>[15]\AgdaSymbol{:}%
\>[18]\AgdaFunction{Domain}%
\>[31]\AgdaComment{--\ numbers}\<%
\\
\>[0]\AgdaFunction{𝐄𝐩}%
\>[15]\AgdaSymbol{=}%
\>[18]\AgdaSymbol{(}\AgdaFunction{𝐋}\AgdaSpace{}%
\AgdaOperator{\AgdaFunction{×}}\AgdaSpace{}%
\AgdaFunction{𝐋}\AgdaSpace{}%
\AgdaOperator{\AgdaFunction{×}}\AgdaSpace{}%
\AgdaFunction{𝐓}\AgdaSymbol{)}%
\>[31]\AgdaComment{--\ pairs}\<%
\\
\>[0]\AgdaFunction{𝐄𝐯}%
\>[15]\AgdaSymbol{=}%
\>[18]\AgdaSymbol{(}\AgdaFunction{𝐋}\AgdaSpace{}%
\AgdaOperator{\AgdaFunction{⋆}}\AgdaSpace{}%
\AgdaOperator{\AgdaFunction{×}}\AgdaSpace{}%
\AgdaFunction{𝐓}\AgdaSymbol{)}%
\>[31]\AgdaComment{--\ vectors}\<%
\\
\>[0]\AgdaFunction{𝐄𝐬}%
\>[15]\AgdaSymbol{=}%
\>[18]\AgdaSymbol{(}\AgdaFunction{𝐋}\AgdaSpace{}%
\AgdaOperator{\AgdaFunction{⋆}}\AgdaSpace{}%
\AgdaOperator{\AgdaFunction{×}}\AgdaSpace{}%
\AgdaFunction{𝐓}\AgdaSymbol{)}%
\>[31]\AgdaComment{--\ strings}\<%
\\
\>[0]\AgdaKeyword{data}\AgdaSpace{}%
\AgdaDatatype{Misc}%
\>[15]\AgdaSymbol{:}%
\>[18]\AgdaPrimitive{Set}\AgdaSpace{}%
\AgdaKeyword{where}%
\>[31]\AgdaInductiveConstructor{false}\AgdaSpace{}%
\AgdaInductiveConstructor{true}\AgdaSpace{}%
\AgdaInductiveConstructor{null}\AgdaSpace{}%
\AgdaInductiveConstructor{undefined}\AgdaSpace{}%
\AgdaInductiveConstructor{unspecified}\AgdaSpace{}%
\AgdaSymbol{:}\AgdaSpace{}%
\AgdaDatatype{Misc}\<%
\\
\>[0]\AgdaFunction{𝐌}%
\>[15]\AgdaSymbol{=}%
\>[18]\AgdaDatatype{Misc}\AgdaSpace{}%
\AgdaOperator{\AgdaFunction{+⊥}}%
\>[31]\AgdaComment{--\ miscellaneous}\<%
\\
\>[0]\AgdaFunction{𝐗}%
\>[15]\AgdaSymbol{=}%
\>[18]\AgdaPostulate{String}\AgdaSpace{}%
\AgdaOperator{\AgdaFunction{+⊥}}%
\>[31]\AgdaComment{--\ errors}\<%
\\
%
\\[\AgdaEmptyExtraSkip]%
\>[0]\AgdaComment{--\ Domain\ isomorphisms}\<%
\\
%
\\[\AgdaEmptyExtraSkip]%
\>[0]\AgdaKeyword{open}\AgdaSpace{}%
\AgdaKeyword{import}\AgdaSpace{}%
\AgdaModule{Function}\<%
\\
\>[0][@{}l@{\AgdaIndent{0}}]%
\>[2]\AgdaKeyword{using}\AgdaSpace{}%
\AgdaSymbol{(}\AgdaOperator{\AgdaFunction{\AgdaUnderscore{}↔\AgdaUnderscore{}}}\AgdaSymbol{)}\AgdaSpace{}%
\AgdaKeyword{public}\<%
\\
%
\\[\AgdaEmptyExtraSkip]%
\>[0]\AgdaKeyword{postulate}\<%
\\
\>[0][@{}l@{\AgdaIndent{0}}]%
\>[2]\AgdaPostulate{𝐅}%
\>[15]\AgdaSymbol{:}%
\>[18]\AgdaFunction{Domain}%
\>[31]\AgdaComment{--\ procedure\ values}\<%
\\
%
\>[2]\AgdaPostulate{𝐄}%
\>[15]\AgdaSymbol{:}%
\>[18]\AgdaFunction{Domain}%
\>[31]\AgdaComment{--\ expressed\ values}\<%
\\
%
\>[2]\AgdaPostulate{𝐒}%
\>[15]\AgdaSymbol{:}%
\>[18]\AgdaFunction{Domain}%
\>[31]\AgdaComment{--\ stores}\<%
\\
%
\>[2]\AgdaPostulate{𝐔}%
\>[15]\AgdaSymbol{:}%
\>[18]\AgdaFunction{Domain}%
\>[31]\AgdaComment{--\ environments}\<%
\\
%
\>[2]\AgdaPostulate{𝐂}%
\>[15]\AgdaSymbol{:}%
\>[18]\AgdaFunction{Domain}%
\>[31]\AgdaComment{--\ command\ continuations}\<%
\\
%
\>[2]\AgdaPostulate{𝐊}%
\>[15]\AgdaSymbol{:}%
\>[18]\AgdaFunction{Domain}%
\>[31]\AgdaComment{--\ expression\ continuations}\<%
\\
%
\>[2]\AgdaPostulate{𝐀}%
\>[15]\AgdaSymbol{:}%
\>[18]\AgdaFunction{Domain}%
\>[31]\AgdaComment{--\ answers}\<%
\\
%
\\[\AgdaEmptyExtraSkip]%
\>[0]\AgdaKeyword{postulate}\AgdaSpace{}%
\AgdaKeyword{instance}\<%
\\
\>[0][@{}l@{\AgdaIndent{0}}]%
\>[2]\AgdaPostulate{iso-𝐅}%
\>[15]\AgdaSymbol{:}\AgdaSpace{}%
\AgdaPostulate{𝐅}%
\>[20]\AgdaOperator{\AgdaFunction{↔}}%
\>[23]\AgdaSymbol{(}\AgdaFunction{𝐋}\AgdaSpace{}%
\AgdaOperator{\AgdaFunction{×}}\AgdaSpace{}%
\AgdaSymbol{(}\AgdaPostulate{𝐄}\AgdaSpace{}%
\AgdaOperator{\AgdaFunction{⋆}}\AgdaSpace{}%
\AgdaSymbol{→}\AgdaSpace{}%
\AgdaPostulate{𝐊}\AgdaSpace{}%
\AgdaSymbol{→}\AgdaSpace{}%
\AgdaPostulate{𝐂}\AgdaSymbol{))}\<%
\\
%
\>[2]\AgdaPostulate{iso-𝐄}%
\>[15]\AgdaSymbol{:}\AgdaSpace{}%
\AgdaPostulate{𝐄}%
\>[20]\AgdaOperator{\AgdaFunction{↔}}%
\>[23]\AgdaSymbol{(}\AgdaPostulate{𝕃}\AgdaSpace{}%
\AgdaSymbol{(}\AgdaPostulate{𝐐}\AgdaSpace{}%
\AgdaOperator{\AgdaDatatype{+}}\AgdaSpace{}%
\AgdaPostulate{𝐇}\AgdaSpace{}%
\AgdaOperator{\AgdaDatatype{+}}\AgdaSpace{}%
\AgdaPostulate{𝐑}\AgdaSpace{}%
\AgdaOperator{\AgdaDatatype{+}}\AgdaSpace{}%
\AgdaFunction{𝐄𝐩}\AgdaSpace{}%
\AgdaOperator{\AgdaDatatype{+}}\AgdaSpace{}%
\AgdaFunction{𝐄𝐯}\AgdaSpace{}%
\AgdaOperator{\AgdaDatatype{+}}\AgdaSpace{}%
\AgdaFunction{𝐄𝐬}\AgdaSpace{}%
\AgdaOperator{\AgdaDatatype{+}}\AgdaSpace{}%
\AgdaFunction{𝐌}\AgdaSpace{}%
\AgdaOperator{\AgdaDatatype{+}}\AgdaSpace{}%
\AgdaPostulate{𝐅}\AgdaSymbol{))}\<%
\\
%
\>[2]\AgdaPostulate{iso-𝐒}%
\>[15]\AgdaSymbol{:}\AgdaSpace{}%
\AgdaPostulate{𝐒}%
\>[20]\AgdaOperator{\AgdaFunction{↔}}%
\>[23]\AgdaSymbol{(}\AgdaFunction{𝐋}\AgdaSpace{}%
\AgdaSymbol{→}\AgdaSpace{}%
\AgdaPostulate{𝐄}\AgdaSpace{}%
\AgdaOperator{\AgdaFunction{×}}\AgdaSpace{}%
\AgdaFunction{𝐓}\AgdaSymbol{)}\<%
\\
%
\>[2]\AgdaPostulate{iso-𝐔}%
\>[15]\AgdaSymbol{:}\AgdaSpace{}%
\AgdaPostulate{𝐔}%
\>[20]\AgdaOperator{\AgdaFunction{↔}}%
\>[23]\AgdaSymbol{(}\AgdaPostulate{Ide}\AgdaSpace{}%
\AgdaSymbol{→}\AgdaSpace{}%
\AgdaFunction{𝐋}\AgdaSymbol{)}\<%
\\
%
\>[2]\AgdaPostulate{iso-𝐂}%
\>[15]\AgdaSymbol{:}\AgdaSpace{}%
\AgdaPostulate{𝐂}%
\>[20]\AgdaOperator{\AgdaFunction{↔}}%
\>[23]\AgdaSymbol{(}\AgdaPostulate{𝐒}\AgdaSpace{}%
\AgdaSymbol{→}\AgdaSpace{}%
\AgdaPostulate{𝐀}\AgdaSymbol{)}\<%
\\
%
\>[2]\AgdaPostulate{iso-𝐊}%
\>[15]\AgdaSymbol{:}\AgdaSpace{}%
\AgdaPostulate{𝐊}%
\>[20]\AgdaOperator{\AgdaFunction{↔}}%
\>[23]\AgdaSymbol{(}\AgdaPostulate{𝐄}\AgdaSpace{}%
\AgdaOperator{\AgdaFunction{⋆}}\AgdaSpace{}%
\AgdaSymbol{→}\AgdaSpace{}%
\AgdaPostulate{𝐂}\AgdaSymbol{)}\<%
\\
%
\\[\AgdaEmptyExtraSkip]%
\>[0]\AgdaKeyword{open}\AgdaSpace{}%
\AgdaModule{Function.Inverse}\AgdaSpace{}%
\AgdaSymbol{\{\{}\AgdaSpace{}%
\AgdaSymbol{...}\AgdaSpace{}%
\AgdaSymbol{\}\}}\<%
\\
\>[0][@{}l@{\AgdaIndent{0}}]%
\>[2]\AgdaKeyword{renaming}\AgdaSpace{}%
\AgdaSymbol{(}\AgdaField{to}\AgdaSpace{}%
\AgdaSymbol{to}\AgdaSpace{}%
\AgdaField{▻}\AgdaSpace{}%
\AgdaSymbol{;}\AgdaSpace{}%
\AgdaField{from}\AgdaSpace{}%
\AgdaSymbol{to}\AgdaSpace{}%
\AgdaField{◅}\AgdaSpace{}%
\AgdaSymbol{)}\AgdaSpace{}%
\AgdaKeyword{public}\<%
\\
%
\>[2]\AgdaComment{--\ iso-D\ :\ D\ ↔\ D′\ declares\ ▻\ :\ D\ →\ D′\ and\ ◅\ :\ D′\ →\ D}\<%
\end{code}
\clearpage
\begin{code}%
\>[0]\AgdaKeyword{variable}\<%
\\
\>[0][@{}l@{\AgdaIndent{0}}]%
\>[2]\AgdaGeneralizable{α}%
\>[6]\AgdaSymbol{:}\AgdaSpace{}%
\AgdaFunction{𝐋}\<%
\\
%
\>[2]\AgdaGeneralizable{α⋆}%
\>[6]\AgdaSymbol{:}\AgdaSpace{}%
\AgdaFunction{𝐋}\AgdaSpace{}%
\AgdaOperator{\AgdaFunction{⋆}}\<%
\\
%
\>[2]\AgdaGeneralizable{ν}%
\>[6]\AgdaSymbol{:}\AgdaSpace{}%
\AgdaFunction{𝐍}\<%
\\
%
\>[2]\AgdaGeneralizable{μ}%
\>[6]\AgdaSymbol{:}\AgdaSpace{}%
\AgdaFunction{𝐌}\<%
\\
%
\>[2]\AgdaGeneralizable{φ}%
\>[6]\AgdaSymbol{:}\AgdaSpace{}%
\AgdaPostulate{𝐅}\<%
\\
%
\>[2]\AgdaGeneralizable{ϵ}%
\>[6]\AgdaSymbol{:}\AgdaSpace{}%
\AgdaPostulate{𝐄}\<%
\\
%
\>[2]\AgdaGeneralizable{ϵ⋆}%
\>[6]\AgdaSymbol{:}\AgdaSpace{}%
\AgdaPostulate{𝐄}\AgdaSpace{}%
\AgdaOperator{\AgdaFunction{⋆}}\<%
\\
%
\>[2]\AgdaGeneralizable{σ}%
\>[6]\AgdaSymbol{:}\AgdaSpace{}%
\AgdaPostulate{𝐒}\<%
\\
%
\>[2]\AgdaGeneralizable{ρ}%
\>[6]\AgdaSymbol{:}\AgdaSpace{}%
\AgdaPostulate{𝐔}\<%
\\
%
\>[2]\AgdaGeneralizable{θ}%
\>[6]\AgdaSymbol{:}\AgdaSpace{}%
\AgdaPostulate{𝐂}\<%
\\
%
\>[2]\AgdaGeneralizable{κ}%
\>[6]\AgdaSymbol{:}\AgdaSpace{}%
\AgdaPostulate{𝐊}\<%
\\
%
\\[\AgdaEmptyExtraSkip]%
\>[0]\AgdaKeyword{pattern}\<%
\\
\>[0][@{}l@{\AgdaIndent{0}}]%
\>[2]\AgdaInductiveConstructor{inj-𝐄𝐩}\AgdaSpace{}%
\AgdaBound{ep}%
\>[13]\AgdaSymbol{=}\AgdaSpace{}%
\AgdaInductiveConstructor{inj₂}\AgdaSpace{}%
\AgdaSymbol{(}\AgdaInductiveConstructor{inj₂}\AgdaSpace{}%
\AgdaSymbol{(}\AgdaInductiveConstructor{inj₂}\AgdaSpace{}%
\AgdaSymbol{(}\AgdaInductiveConstructor{inj₁}\AgdaSpace{}%
\AgdaBound{ep}\AgdaSymbol{)))}\<%
\\
\>[0]\AgdaKeyword{pattern}\<%
\\
\>[0][@{}l@{\AgdaIndent{0}}]%
\>[2]\AgdaInductiveConstructor{inj-𝐌}\AgdaSpace{}%
\AgdaBound{μ}%
\>[13]\AgdaSymbol{=}\AgdaSpace{}%
\AgdaInductiveConstructor{inj₂}\AgdaSpace{}%
\AgdaSymbol{(}\AgdaInductiveConstructor{inj₂}\AgdaSpace{}%
\AgdaSymbol{(}\AgdaInductiveConstructor{inj₂}\AgdaSpace{}%
\AgdaSymbol{(}\AgdaInductiveConstructor{inj₂}\AgdaSpace{}%
\AgdaSymbol{(}\AgdaInductiveConstructor{inj₂}\AgdaSpace{}%
\AgdaSymbol{(}\AgdaInductiveConstructor{inj₂}\AgdaSpace{}%
\AgdaSymbol{(}\AgdaInductiveConstructor{inj₁}\AgdaSpace{}%
\AgdaBound{μ}\AgdaSymbol{))))))}\<%
\\
\>[0]\AgdaKeyword{pattern}\<%
\\
\>[0][@{}l@{\AgdaIndent{0}}]%
\>[2]\AgdaInductiveConstructor{inj-𝐅}\AgdaSpace{}%
\AgdaBound{φ}%
\>[13]\AgdaSymbol{=}\AgdaSpace{}%
\AgdaInductiveConstructor{inj₂}\AgdaSpace{}%
\AgdaSymbol{(}\AgdaInductiveConstructor{inj₂}\AgdaSpace{}%
\AgdaSymbol{(}\AgdaInductiveConstructor{inj₂}\AgdaSpace{}%
\AgdaSymbol{(}\AgdaInductiveConstructor{inj₂}\AgdaSpace{}%
\AgdaSymbol{(}\AgdaInductiveConstructor{inj₂}\AgdaSpace{}%
\AgdaSymbol{(}\AgdaInductiveConstructor{inj₂}\AgdaSpace{}%
\AgdaSymbol{(}\AgdaInductiveConstructor{inj₂}\AgdaSpace{}%
\AgdaBound{φ}\AgdaSymbol{))))))}\<%
\\
%
\\[\AgdaEmptyExtraSkip]%
\>[0]\AgdaOperator{\AgdaFunction{\AgdaUnderscore{}∈𝐅}}%
\>[13]\AgdaSymbol{:}\AgdaSpace{}%
\AgdaPostulate{𝐄}\AgdaSpace{}%
\AgdaSymbol{→}\AgdaSpace{}%
\AgdaDatatype{Bool}\AgdaSpace{}%
\AgdaOperator{\AgdaFunction{+⊥}}\<%
\\
\>[0]\AgdaBound{ϵ}\AgdaSpace{}%
\AgdaOperator{\AgdaFunction{∈𝐅}}%
\>[13]\AgdaSymbol{=}\AgdaSpace{}%
\AgdaSymbol{((λ}\AgdaSpace{}%
\AgdaSymbol{\{}\AgdaSpace{}%
\AgdaSymbol{(}\AgdaInductiveConstructor{inj-𝐅}\AgdaSpace{}%
\AgdaSymbol{\AgdaUnderscore{})}\AgdaSpace{}%
\AgdaSymbol{→}\AgdaSpace{}%
\AgdaPostulate{η}\AgdaSpace{}%
\AgdaInductiveConstructor{true}\AgdaSpace{}%
\AgdaSymbol{;}\AgdaSpace{}%
\AgdaCatchallClause{\AgdaSymbol{\AgdaUnderscore{}}}\AgdaSpace{}%
\AgdaSymbol{→}\AgdaSpace{}%
\AgdaPostulate{η}\AgdaSpace{}%
\AgdaInductiveConstructor{false}\AgdaSpace{}%
\AgdaSymbol{\})}\AgdaSpace{}%
\AgdaOperator{\AgdaPostulate{♯}}\AgdaSymbol{)}\AgdaSpace{}%
\AgdaSymbol{(}\AgdaField{▻}\AgdaSpace{}%
\AgdaBound{ϵ}\AgdaSymbol{)}\<%
\\
%
\\[\AgdaEmptyExtraSkip]%
\>[0]\AgdaOperator{\AgdaFunction{\AgdaUnderscore{}|𝐅}}%
\>[13]\AgdaSymbol{:}\AgdaSpace{}%
\AgdaPostulate{𝐄}\AgdaSpace{}%
\AgdaSymbol{→}\AgdaSpace{}%
\AgdaPostulate{𝐅}\<%
\\
\>[0]\AgdaBound{ϵ}\AgdaSpace{}%
\AgdaOperator{\AgdaFunction{|𝐅}}%
\>[13]\AgdaSymbol{=}\AgdaSpace{}%
\AgdaSymbol{((λ}\AgdaSpace{}%
\AgdaSymbol{\{}\AgdaSpace{}%
\AgdaSymbol{(}\AgdaInductiveConstructor{inj-𝐅}\AgdaSpace{}%
\AgdaBound{φ}\AgdaSymbol{)}\AgdaSpace{}%
\AgdaSymbol{→}\AgdaSpace{}%
\AgdaBound{φ}\AgdaSpace{}%
\AgdaSymbol{;}\AgdaSpace{}%
\AgdaCatchallClause{\AgdaSymbol{\AgdaUnderscore{}}}\AgdaSpace{}%
\AgdaSymbol{→}\AgdaSpace{}%
\AgdaPostulate{⊥}\AgdaSpace{}%
\AgdaSymbol{\})}\AgdaSpace{}%
\AgdaOperator{\AgdaPostulate{♯}}\AgdaSymbol{)}\AgdaSpace{}%
\AgdaSymbol{(}\AgdaField{▻}\AgdaSpace{}%
\AgdaBound{ϵ}\AgdaSymbol{)}\<%
\\
%
\\[\AgdaEmptyExtraSkip]%
\>[0]\AgdaOperator{\AgdaFunction{\AgdaUnderscore{}∈𝐋}}%
\>[13]\AgdaSymbol{:}\AgdaSpace{}%
\AgdaPostulate{𝕃}\AgdaSpace{}%
\AgdaSymbol{(}\AgdaFunction{𝐋}\AgdaSpace{}%
\AgdaOperator{\AgdaDatatype{+}}\AgdaSpace{}%
\AgdaFunction{𝐗}\AgdaSymbol{)}\AgdaSpace{}%
\AgdaSymbol{→}\AgdaSpace{}%
\AgdaDatatype{Bool}\AgdaSpace{}%
\AgdaOperator{\AgdaFunction{+⊥}}\<%
\\
\>[0]\AgdaOperator{\AgdaFunction{\AgdaUnderscore{}∈𝐋}}%
\>[13]\AgdaSymbol{=}\AgdaSpace{}%
\AgdaOperator{\AgdaFunction{[}}\AgdaSpace{}%
\AgdaSymbol{(λ}\AgdaSpace{}%
\AgdaBound{\AgdaUnderscore{}}\AgdaSpace{}%
\AgdaSymbol{→}\AgdaSpace{}%
\AgdaPostulate{η}\AgdaSpace{}%
\AgdaInductiveConstructor{true}\AgdaSymbol{)}\AgdaOperator{\AgdaFunction{,}}\AgdaSpace{}%
\AgdaSymbol{(λ}\AgdaSpace{}%
\AgdaBound{\AgdaUnderscore{}}\AgdaSpace{}%
\AgdaSymbol{→}\AgdaSpace{}%
\AgdaPostulate{η}\AgdaSpace{}%
\AgdaInductiveConstructor{false}\AgdaSymbol{)}\AgdaSpace{}%
\AgdaOperator{\AgdaFunction{]}}\AgdaSpace{}%
\AgdaOperator{\AgdaPostulate{♯}}\<%
\\
%
\\[\AgdaEmptyExtraSkip]%
\>[0]\AgdaOperator{\AgdaFunction{\AgdaUnderscore{}|𝐋}}%
\>[13]\AgdaSymbol{:}\AgdaSpace{}%
\AgdaPostulate{𝕃}\AgdaSpace{}%
\AgdaSymbol{(}\AgdaFunction{𝐋}\AgdaSpace{}%
\AgdaOperator{\AgdaDatatype{+}}\AgdaSpace{}%
\AgdaFunction{𝐗}\AgdaSymbol{)}\AgdaSpace{}%
\AgdaSymbol{→}\AgdaSpace{}%
\AgdaFunction{𝐋}\<%
\\
\>[0]\AgdaOperator{\AgdaFunction{\AgdaUnderscore{}|𝐋}}%
\>[13]\AgdaSymbol{=}\AgdaSpace{}%
\AgdaOperator{\AgdaFunction{[}}\AgdaSpace{}%
\AgdaFunction{id}\AgdaSpace{}%
\AgdaOperator{\AgdaFunction{,}}\AgdaSpace{}%
\AgdaSymbol{(λ}\AgdaSpace{}%
\AgdaBound{\AgdaUnderscore{}}\AgdaSpace{}%
\AgdaSymbol{→}\AgdaSpace{}%
\AgdaPostulate{⊥}\AgdaSymbol{)}\AgdaSpace{}%
\AgdaOperator{\AgdaFunction{]}}\AgdaSpace{}%
\AgdaOperator{\AgdaPostulate{♯}}\<%
\\
%
\\[\AgdaEmptyExtraSkip]%
\>[0]\AgdaOperator{\AgdaFunction{\AgdaUnderscore{}𝐄𝐩-in-𝐄}}%
\>[18]\AgdaSymbol{:}\AgdaSpace{}%
\AgdaFunction{𝐄𝐩}\AgdaSpace{}%
\AgdaSymbol{→}\AgdaSpace{}%
\AgdaPostulate{𝐄}\<%
\\
\>[0]\AgdaBound{ep}\AgdaSpace{}%
\AgdaOperator{\AgdaFunction{𝐄𝐩-in-𝐄}}%
\>[18]\AgdaSymbol{=}\AgdaSpace{}%
\AgdaField{◅}\AgdaSpace{}%
\AgdaSymbol{(}\AgdaPostulate{η}\AgdaSpace{}%
\AgdaSymbol{(}\AgdaInductiveConstructor{inj-𝐄𝐩}\AgdaSpace{}%
\AgdaBound{ep}\AgdaSymbol{))}\<%
\\
%
\\[\AgdaEmptyExtraSkip]%
\>[0]\AgdaOperator{\AgdaFunction{\AgdaUnderscore{}𝐅-in-𝐄}}%
\>[18]\AgdaSymbol{:}\AgdaSpace{}%
\AgdaPostulate{𝐅}\AgdaSpace{}%
\AgdaSymbol{→}\AgdaSpace{}%
\AgdaPostulate{𝐄}\<%
\\
\>[0]\AgdaBound{φ}\AgdaSpace{}%
\AgdaOperator{\AgdaFunction{𝐅-in-𝐄}}%
\>[18]\AgdaSymbol{=}\AgdaSpace{}%
\AgdaField{◅}\AgdaSpace{}%
\AgdaSymbol{(}\AgdaPostulate{η}\AgdaSpace{}%
\AgdaSymbol{(}\AgdaInductiveConstructor{inj-𝐅}\AgdaSpace{}%
\AgdaBound{φ}\AgdaSymbol{))}\<%
\\
%
\\[\AgdaEmptyExtraSkip]%
\>[0]\AgdaFunction{unspecified-in-𝐄}%
\>[18]\AgdaSymbol{:}\AgdaSpace{}%
\AgdaPostulate{𝐄}\<%
\\
\>[0]\AgdaFunction{unspecified-in-𝐄}%
\>[18]\AgdaSymbol{=}\AgdaSpace{}%
\AgdaField{◅}\AgdaSpace{}%
\AgdaSymbol{(}\AgdaPostulate{η}\AgdaSpace{}%
\AgdaSymbol{(}\AgdaInductiveConstructor{inj-𝐌}\AgdaSpace{}%
\AgdaSymbol{(}\AgdaPostulate{η}\AgdaSpace{}%
\AgdaInductiveConstructor{unspecified}\AgdaSymbol{)))}\<%
\end{code} 

\clearpage

\begin{code}%
\>[0]\<%
\\
\>[0]\AgdaKeyword{module}\AgdaSpace{}%
\AgdaModule{Scheme.Auxiliary-Functions}\AgdaSpace{}%
\AgdaKeyword{where}\<%
\\
%
\\[\AgdaEmptyExtraSkip]%
\>[0]\AgdaKeyword{open}\AgdaSpace{}%
\AgdaKeyword{import}\AgdaSpace{}%
\AgdaModule{Scheme.Domain-Notation}\<%
\\
\>[0]\AgdaKeyword{open}\AgdaSpace{}%
\AgdaKeyword{import}\AgdaSpace{}%
\AgdaModule{Scheme.Domain-Equations}\<%
\\
\>[0]\AgdaKeyword{open}\AgdaSpace{}%
\AgdaKeyword{import}\AgdaSpace{}%
\AgdaModule{Scheme.Abstract-Syntax}\AgdaSpace{}%
\AgdaKeyword{using}\AgdaSpace{}%
\AgdaSymbol{(}\AgdaPostulate{Ide}\AgdaSymbol{)}\<%
\\
%
\\[\AgdaEmptyExtraSkip]%
\>[0]\AgdaKeyword{open}\AgdaSpace{}%
\AgdaKeyword{import}\AgdaSpace{}%
\AgdaModule{Data.Nat.Base}\<%
\\
\>[0][@{}l@{\AgdaIndent{0}}]%
\>[2]\AgdaKeyword{using}\AgdaSpace{}%
\AgdaSymbol{(}\AgdaRecord{NonZero}\AgdaSymbol{;}\AgdaSpace{}%
\AgdaFunction{pred}\AgdaSymbol{)}\AgdaSpace{}%
\AgdaKeyword{public}\<%
\\
%
\\[\AgdaEmptyExtraSkip]%
\>[0]\AgdaComment{--\ 7.2.4.\ Auxiliary\ functions}\<%
\\
%
\\[\AgdaEmptyExtraSkip]%
\>[0]\AgdaKeyword{postulate}\AgdaSpace{}%
\AgdaOperator{\AgdaPostulate{\AgdaUnderscore{}==ᴵ\AgdaUnderscore{}}}\AgdaSpace{}%
\AgdaSymbol{:}\AgdaSpace{}%
\AgdaPostulate{Ide}\AgdaSpace{}%
\AgdaSymbol{→}\AgdaSpace{}%
\AgdaPostulate{Ide}\AgdaSpace{}%
\AgdaSymbol{→}\AgdaSpace{}%
\AgdaDatatype{Bool}\<%
\\
%
\\[\AgdaEmptyExtraSkip]%
\>[0]\AgdaOperator{\AgdaFunction{\AgdaUnderscore{}[\AgdaUnderscore{}/\AgdaUnderscore{}]}}\AgdaSpace{}%
\AgdaSymbol{:}\AgdaSpace{}%
\AgdaPostulate{𝐔}\AgdaSpace{}%
\AgdaSymbol{→}\AgdaSpace{}%
\AgdaFunction{𝐋}\AgdaSpace{}%
\AgdaSymbol{→}\AgdaSpace{}%
\AgdaPostulate{Ide}\AgdaSpace{}%
\AgdaSymbol{→}\AgdaSpace{}%
\AgdaPostulate{𝐔}\<%
\\
\>[0]\AgdaBound{ρ}\AgdaSpace{}%
\AgdaOperator{\AgdaFunction{[}}\AgdaSpace{}%
\AgdaBound{α}\AgdaSpace{}%
\AgdaOperator{\AgdaFunction{/}}\AgdaSpace{}%
\AgdaBound{I}\AgdaSpace{}%
\AgdaOperator{\AgdaFunction{]}}\AgdaSpace{}%
\AgdaSymbol{=}\AgdaSpace{}%
\AgdaField{◅}\AgdaSpace{}%
\AgdaSymbol{λ}\AgdaSpace{}%
\AgdaBound{I′}\AgdaSpace{}%
\AgdaSymbol{→}\AgdaSpace{}%
\AgdaOperator{\AgdaFunction{if}}\AgdaSpace{}%
\AgdaBound{I}\AgdaSpace{}%
\AgdaOperator{\AgdaPostulate{==ᴵ}}\AgdaSpace{}%
\AgdaBound{I′}\AgdaSpace{}%
\AgdaOperator{\AgdaFunction{then}}\AgdaSpace{}%
\AgdaBound{α}\AgdaSpace{}%
\AgdaOperator{\AgdaFunction{else}}\AgdaSpace{}%
\AgdaField{▻}\AgdaSpace{}%
\AgdaBound{ρ}\AgdaSpace{}%
\AgdaBound{I′}\<%
\\
%
\\[\AgdaEmptyExtraSkip]%
\>[0]\AgdaFunction{lookup}\AgdaSpace{}%
\AgdaSymbol{:}\AgdaSpace{}%
\AgdaPostulate{𝐔}\AgdaSpace{}%
\AgdaSymbol{→}\AgdaSpace{}%
\AgdaPostulate{Ide}\AgdaSpace{}%
\AgdaSymbol{→}\AgdaSpace{}%
\AgdaFunction{𝐋}\<%
\\
\>[0]\AgdaFunction{lookup}\AgdaSpace{}%
\AgdaSymbol{=}\AgdaSpace{}%
\AgdaSymbol{λ}\AgdaSpace{}%
\AgdaBound{ρ}\AgdaSpace{}%
\AgdaBound{I}\AgdaSpace{}%
\AgdaSymbol{→}\AgdaSpace{}%
\AgdaField{▻}\AgdaSpace{}%
\AgdaBound{ρ}\AgdaSpace{}%
\AgdaBound{I}\<%
\\
%
\\[\AgdaEmptyExtraSkip]%
\>[0]\AgdaFunction{extends}\AgdaSpace{}%
\AgdaSymbol{:}\AgdaSpace{}%
\AgdaPostulate{𝐔}\AgdaSpace{}%
\AgdaSymbol{→}\AgdaSpace{}%
\AgdaPostulate{Ide}\AgdaSpace{}%
\AgdaOperator{\AgdaFunction{*}}\AgdaSpace{}%
\AgdaSymbol{→}\AgdaSpace{}%
\AgdaFunction{𝐋}\AgdaSpace{}%
\AgdaOperator{\AgdaFunction{⋆}}\AgdaSpace{}%
\AgdaSymbol{→}\AgdaSpace{}%
\AgdaPostulate{𝐔}\<%
\\
\>[0]\AgdaFunction{extends}\AgdaSpace{}%
\AgdaSymbol{=}\AgdaSpace{}%
\AgdaPostulate{fix}\AgdaSpace{}%
\AgdaSymbol{λ}\AgdaSpace{}%
\AgdaBound{extends′}\AgdaSpace{}%
\AgdaSymbol{→}\<%
\\
\>[0][@{}l@{\AgdaIndent{0}}]%
\>[2]\AgdaSymbol{λ}%
\>[79I]\AgdaBound{ρ}\AgdaSpace{}%
\AgdaBound{I*}\AgdaSpace{}%
\AgdaBound{α⋆}\AgdaSpace{}%
\AgdaSymbol{→}\<%
\\
\>[.][@{}l@{}]\<[79I]%
\>[4]\AgdaPostulate{η}%
\>[83I]\AgdaSymbol{(}\AgdaFunction{\#′}\AgdaSpace{}%
\AgdaBound{I*}\AgdaSpace{}%
\AgdaOperator{\AgdaPrimitive{==}}\AgdaSpace{}%
\AgdaNumber{0}\AgdaSymbol{)}\AgdaSpace{}%
\AgdaOperator{\AgdaPostulate{⟶}}\AgdaSpace{}%
\AgdaBound{ρ}\AgdaSpace{}%
\AgdaOperator{\AgdaPostulate{,}}\<%
\\
\>[.][@{}l@{}]\<[83I]%
\>[6]\AgdaSymbol{(}\AgdaSpace{}%
\AgdaSymbol{(}%
\>[91I]\AgdaSymbol{(}%
\>[92I]\AgdaSymbol{(λ}\AgdaSpace{}%
\AgdaBound{I}\AgdaSpace{}%
\AgdaSymbol{→}\AgdaSpace{}%
\AgdaSymbol{λ}\AgdaSpace{}%
\AgdaBound{I*′}\AgdaSpace{}%
\AgdaSymbol{→}\<%
\\
\>[92I][@{}l@{\AgdaIndent{0}}]%
\>[14]\AgdaBound{extends′}\AgdaSpace{}%
\AgdaSymbol{(}\AgdaBound{ρ}\AgdaSpace{}%
\AgdaOperator{\AgdaFunction{[}}\AgdaSpace{}%
\AgdaSymbol{(}\AgdaBound{α⋆}\AgdaSpace{}%
\AgdaOperator{\AgdaFunction{↓}}\AgdaSpace{}%
\AgdaNumber{1}\AgdaSymbol{)}\AgdaSpace{}%
\AgdaOperator{\AgdaFunction{/}}\AgdaSpace{}%
\AgdaBound{I}\AgdaSpace{}%
\AgdaOperator{\AgdaFunction{]}}\AgdaSymbol{)}\AgdaSpace{}%
\AgdaBound{I*′}\AgdaSpace{}%
\AgdaSymbol{(}\AgdaBound{α⋆}\AgdaSpace{}%
\AgdaOperator{\AgdaFunction{†}}\AgdaSpace{}%
\AgdaNumber{1}\AgdaSymbol{))}\AgdaSpace{}%
\AgdaOperator{\AgdaPostulate{♯}}\AgdaSymbol{)}\<%
\\
\>[.][@{}l@{}]\<[91I]%
\>[10]\AgdaSymbol{(}\AgdaBound{I*}\AgdaSpace{}%
\AgdaOperator{\AgdaFunction{↓′}}\AgdaSpace{}%
\AgdaNumber{1}\AgdaSymbol{))}\AgdaSpace{}%
\AgdaOperator{\AgdaPostulate{♯}}\AgdaSymbol{)}\AgdaSpace{}%
\AgdaSymbol{(}\AgdaBound{I*}\AgdaSpace{}%
\AgdaOperator{\AgdaFunction{†′}}\AgdaSpace{}%
\AgdaNumber{1}\AgdaSymbol{)}\<%
\\
%
\\[\AgdaEmptyExtraSkip]%
\>[0]\AgdaKeyword{postulate}\<%
\\
\>[0][@{}l@{\AgdaIndent{0}}]%
\>[2]\AgdaPostulate{wrong}\AgdaSpace{}%
\AgdaSymbol{:}\AgdaSpace{}%
\AgdaPostulate{String}\AgdaSpace{}%
\AgdaSymbol{→}\AgdaSpace{}%
\AgdaPostulate{𝐂}\<%
\\
%
\>[2]\AgdaComment{--\ wrong\ :\ 𝐗\ →\ 𝐂\ --\ implementation-dependent}\<%
\\
%
\\[\AgdaEmptyExtraSkip]%
\>[0]\AgdaFunction{send}\AgdaSpace{}%
\AgdaSymbol{:}\AgdaSpace{}%
\AgdaPostulate{𝐄}\AgdaSpace{}%
\AgdaSymbol{→}\AgdaSpace{}%
\AgdaPostulate{𝐊}\AgdaSpace{}%
\AgdaSymbol{→}\AgdaSpace{}%
\AgdaPostulate{𝐂}\<%
\\
\>[0]\AgdaFunction{send}\AgdaSpace{}%
\AgdaSymbol{=}\AgdaSpace{}%
\AgdaSymbol{λ}\AgdaSpace{}%
\AgdaBound{ϵ}\AgdaSpace{}%
\AgdaBound{κ}\AgdaSpace{}%
\AgdaSymbol{→}\AgdaSpace{}%
\AgdaField{▻}\AgdaSpace{}%
\AgdaBound{κ}\AgdaSpace{}%
\AgdaOperator{\AgdaFunction{⟨}}\AgdaSpace{}%
\AgdaBound{ϵ}\AgdaSpace{}%
\AgdaOperator{\AgdaFunction{⟩}}\<%
\\
%
\\[\AgdaEmptyExtraSkip]%
\>[0]\AgdaFunction{single}\AgdaSpace{}%
\AgdaSymbol{:}\AgdaSpace{}%
\AgdaSymbol{(}\AgdaPostulate{𝐄}\AgdaSpace{}%
\AgdaSymbol{→}\AgdaSpace{}%
\AgdaPostulate{𝐂}\AgdaSymbol{)}\AgdaSpace{}%
\AgdaSymbol{→}\AgdaSpace{}%
\AgdaPostulate{𝐊}\<%
\\
\>[0]\AgdaFunction{single}\AgdaSpace{}%
\AgdaSymbol{=}\<%
\\
\>[0][@{}l@{\AgdaIndent{0}}]%
\>[2]\AgdaSymbol{λ}%
\>[144I]\AgdaBound{ψ}\AgdaSpace{}%
\AgdaSymbol{→}\AgdaSpace{}%
\AgdaField{◅}\AgdaSpace{}%
\AgdaSymbol{λ}\AgdaSpace{}%
\AgdaBound{ϵ⋆}\AgdaSpace{}%
\AgdaSymbol{→}\<%
\\
\>[.][@{}l@{}]\<[144I]%
\>[4]\AgdaSymbol{(}\AgdaFunction{\#}\AgdaSpace{}%
\AgdaBound{ϵ⋆}\AgdaSpace{}%
\AgdaOperator{\AgdaFunction{==⊥}}\AgdaSpace{}%
\AgdaNumber{1}\AgdaSymbol{)}\AgdaSpace{}%
\AgdaOperator{\AgdaPostulate{⟶}}\AgdaSpace{}%
\AgdaBound{ψ}\AgdaSpace{}%
\AgdaSymbol{(}\AgdaBound{ϵ⋆}\AgdaSpace{}%
\AgdaOperator{\AgdaFunction{↓}}\AgdaSpace{}%
\AgdaNumber{1}\AgdaSymbol{)}\AgdaSpace{}%
\AgdaOperator{\AgdaPostulate{,}}\<%
\\
\>[4][@{}l@{\AgdaIndent{0}}]%
\>[6]\AgdaPostulate{wrong}\AgdaSpace{}%
\AgdaString{"wrong\ number\ of\ return\ values"}\<%
\\
%
\\[\AgdaEmptyExtraSkip]%
\>[0]\AgdaKeyword{postulate}\<%
\\
\>[0][@{}l@{\AgdaIndent{0}}]%
\>[2]\AgdaPostulate{new}\AgdaSpace{}%
\AgdaSymbol{:}\AgdaSpace{}%
\AgdaPostulate{𝐒}\AgdaSpace{}%
\AgdaSymbol{→}\AgdaSpace{}%
\AgdaPostulate{𝕃}\AgdaSpace{}%
\AgdaSymbol{(}\AgdaFunction{𝐋}\AgdaSpace{}%
\AgdaOperator{\AgdaDatatype{+}}\AgdaSpace{}%
\AgdaFunction{𝐗}\AgdaSymbol{)}\<%
\\
\>[0]\AgdaComment{--\ new\ :\ 𝐒\ →\ (𝐋\ +\ \{error\})\ --\ implementation-dependent}\<%
\\
%
\\[\AgdaEmptyExtraSkip]%
\>[0]\AgdaFunction{hold}\AgdaSpace{}%
\AgdaSymbol{:}\AgdaSpace{}%
\AgdaFunction{𝐋}\AgdaSpace{}%
\AgdaSymbol{→}\AgdaSpace{}%
\AgdaPostulate{𝐊}\AgdaSpace{}%
\AgdaSymbol{→}\AgdaSpace{}%
\AgdaPostulate{𝐂}\<%
\\
\>[0]\AgdaFunction{hold}\AgdaSpace{}%
\AgdaSymbol{=}\AgdaSpace{}%
\AgdaSymbol{λ}\AgdaSpace{}%
\AgdaBound{α}\AgdaSpace{}%
\AgdaBound{κ}\AgdaSpace{}%
\AgdaSymbol{→}\AgdaSpace{}%
\AgdaField{◅}\AgdaSpace{}%
\AgdaSymbol{λ}\AgdaSpace{}%
\AgdaBound{σ}\AgdaSpace{}%
\AgdaSymbol{→}\AgdaSpace{}%
\AgdaField{▻}\AgdaSpace{}%
\AgdaSymbol{(}\AgdaFunction{send}\AgdaSpace{}%
\AgdaSymbol{(}\AgdaField{▻}\AgdaSpace{}%
\AgdaBound{σ}\AgdaSpace{}%
\AgdaBound{α}\AgdaSpace{}%
\AgdaOperator{\AgdaField{↓1}}\AgdaSymbol{)}\AgdaSpace{}%
\AgdaBound{κ}\AgdaSymbol{)}\AgdaSpace{}%
\AgdaBound{σ}\<%
\\
%
\\[\AgdaEmptyExtraSkip]%
\>[0]\AgdaComment{--\ assign\ :\ 𝐋\ →\ 𝐄\ →\ 𝐂\ →\ 𝐂}\<%
\\
\>[0]\AgdaComment{--\ assign\ =\ λ\ α\ ϵ\ θ\ σ\ →\ θ\ (update\ α\ ϵ\ σ)}\<%
\\
\>[0]\AgdaComment{--\ forward\ reference\ to\ update}\<%
\\
%
\\[\AgdaEmptyExtraSkip]%
\>[0]\AgdaKeyword{postulate}\<%
\\
\>[0][@{}l@{\AgdaIndent{0}}]%
\>[2]\AgdaOperator{\AgdaPostulate{\AgdaUnderscore{}==ᴸ\AgdaUnderscore{}}}\AgdaSpace{}%
\AgdaSymbol{:}\AgdaSpace{}%
\AgdaFunction{𝐋}\AgdaSpace{}%
\AgdaSymbol{→}\AgdaSpace{}%
\AgdaFunction{𝐋}\AgdaSpace{}%
\AgdaSymbol{→}\AgdaSpace{}%
\AgdaFunction{𝐓}\<%
\\
%
\\[\AgdaEmptyExtraSkip]%
\>[0]\AgdaComment{--\ R5RS\ and\ [Stoy]\ explain\ \AgdaUnderscore{}[\AgdaUnderscore{}/\AgdaUnderscore{}]\ only\ in\ connection\ with\ environments}\<%
\\
\>[0]\AgdaOperator{\AgdaFunction{\AgdaUnderscore{}[\AgdaUnderscore{}/\AgdaUnderscore{}]′}}\AgdaSpace{}%
\AgdaSymbol{:}\AgdaSpace{}%
\AgdaPostulate{𝐒}\AgdaSpace{}%
\AgdaSymbol{→}\AgdaSpace{}%
\AgdaSymbol{(}\AgdaPostulate{𝐄}\AgdaSpace{}%
\AgdaOperator{\AgdaFunction{×}}\AgdaSpace{}%
\AgdaFunction{𝐓}\AgdaSymbol{)}\AgdaSpace{}%
\AgdaSymbol{→}\AgdaSpace{}%
\AgdaFunction{𝐋}\AgdaSpace{}%
\AgdaSymbol{→}\AgdaSpace{}%
\AgdaPostulate{𝐒}\<%
\\
\>[0]\AgdaBound{σ}\AgdaSpace{}%
\AgdaOperator{\AgdaFunction{[}}\AgdaSpace{}%
\AgdaBound{z}\AgdaSpace{}%
\AgdaOperator{\AgdaFunction{/}}\AgdaSpace{}%
\AgdaBound{α}\AgdaSpace{}%
\AgdaOperator{\AgdaFunction{]′}}\AgdaSpace{}%
\AgdaSymbol{=}\AgdaSpace{}%
\AgdaField{◅}\AgdaSpace{}%
\AgdaSymbol{λ}\AgdaSpace{}%
\AgdaBound{α′}\AgdaSpace{}%
\AgdaSymbol{→}\AgdaSpace{}%
\AgdaSymbol{(}\AgdaBound{α}\AgdaSpace{}%
\AgdaOperator{\AgdaPostulate{==ᴸ}}\AgdaSpace{}%
\AgdaBound{α′}\AgdaSymbol{)}\AgdaSpace{}%
\AgdaOperator{\AgdaPostulate{⟶}}\AgdaSpace{}%
\AgdaBound{z}\AgdaSpace{}%
\AgdaOperator{\AgdaPostulate{,}}\AgdaSpace{}%
\AgdaField{▻}\AgdaSpace{}%
\AgdaBound{σ}\AgdaSpace{}%
\AgdaBound{α′}\<%
\\
%
\\[\AgdaEmptyExtraSkip]%
\>[0]\AgdaFunction{update}\AgdaSpace{}%
\AgdaSymbol{:}\AgdaSpace{}%
\AgdaFunction{𝐋}\AgdaSpace{}%
\AgdaSymbol{→}\AgdaSpace{}%
\AgdaPostulate{𝐄}\AgdaSpace{}%
\AgdaSymbol{→}\AgdaSpace{}%
\AgdaPostulate{𝐒}\AgdaSpace{}%
\AgdaSymbol{→}\AgdaSpace{}%
\AgdaPostulate{𝐒}\<%
\\
\>[0]\AgdaFunction{update}\AgdaSpace{}%
\AgdaSymbol{=}\AgdaSpace{}%
\AgdaSymbol{λ}\AgdaSpace{}%
\AgdaBound{α}\AgdaSpace{}%
\AgdaBound{ϵ}\AgdaSpace{}%
\AgdaBound{σ}\AgdaSpace{}%
\AgdaSymbol{→}\AgdaSpace{}%
\AgdaBound{σ}\AgdaSpace{}%
\AgdaOperator{\AgdaFunction{[}}\AgdaSpace{}%
\AgdaSymbol{(}\AgdaBound{ϵ}\AgdaSpace{}%
\AgdaOperator{\AgdaInductiveConstructor{,}}\AgdaSpace{}%
\AgdaPostulate{η}\AgdaSpace{}%
\AgdaInductiveConstructor{true}\AgdaSymbol{)}\AgdaSpace{}%
\AgdaOperator{\AgdaFunction{/}}\AgdaSpace{}%
\AgdaBound{α}\AgdaSpace{}%
\AgdaOperator{\AgdaFunction{]′}}\<%
\\
%
\\[\AgdaEmptyExtraSkip]%
\>[0]\AgdaFunction{assign}\AgdaSpace{}%
\AgdaSymbol{:}\AgdaSpace{}%
\AgdaFunction{𝐋}\AgdaSpace{}%
\AgdaSymbol{→}\AgdaSpace{}%
\AgdaPostulate{𝐄}\AgdaSpace{}%
\AgdaSymbol{→}\AgdaSpace{}%
\AgdaPostulate{𝐂}\AgdaSpace{}%
\AgdaSymbol{→}\AgdaSpace{}%
\AgdaPostulate{𝐂}\<%
\\
\>[0]\AgdaFunction{assign}\AgdaSpace{}%
\AgdaSymbol{=}\AgdaSpace{}%
\AgdaSymbol{λ}\AgdaSpace{}%
\AgdaBound{α}\AgdaSpace{}%
\AgdaBound{ϵ}\AgdaSpace{}%
\AgdaBound{θ}\AgdaSpace{}%
\AgdaSymbol{→}\AgdaSpace{}%
\AgdaField{◅}\AgdaSpace{}%
\AgdaSymbol{λ}\AgdaSpace{}%
\AgdaBound{σ}\AgdaSpace{}%
\AgdaSymbol{→}\AgdaSpace{}%
\AgdaField{▻}\AgdaSpace{}%
\AgdaBound{θ}\AgdaSpace{}%
\AgdaSymbol{(}\AgdaFunction{update}\AgdaSpace{}%
\AgdaBound{α}\AgdaSpace{}%
\AgdaBound{ϵ}\AgdaSpace{}%
\AgdaBound{σ}\AgdaSymbol{)}\<%
\\
%
\\[\AgdaEmptyExtraSkip]%
\>[0]\AgdaFunction{tievals}\AgdaSpace{}%
\AgdaSymbol{:}\AgdaSpace{}%
\AgdaSymbol{(}\AgdaFunction{𝐋}\AgdaSpace{}%
\AgdaOperator{\AgdaFunction{⋆}}\AgdaSpace{}%
\AgdaSymbol{→}\AgdaSpace{}%
\AgdaPostulate{𝐂}\AgdaSymbol{)}\AgdaSpace{}%
\AgdaSymbol{→}\AgdaSpace{}%
\AgdaPostulate{𝐄}\AgdaSpace{}%
\AgdaOperator{\AgdaFunction{⋆}}\AgdaSpace{}%
\AgdaSymbol{→}\AgdaSpace{}%
\AgdaPostulate{𝐂}\<%
\\
\>[0]\AgdaFunction{tievals}\AgdaSpace{}%
\AgdaSymbol{=}\AgdaSpace{}%
\AgdaPostulate{fix}\AgdaSpace{}%
\AgdaSymbol{λ}\AgdaSpace{}%
\AgdaBound{tievals′}\AgdaSpace{}%
\AgdaSymbol{→}\<%
\\
\>[0][@{}l@{\AgdaIndent{0}}]%
\>[2]\AgdaSymbol{λ}%
\>[287I]\AgdaBound{ψ}\AgdaSpace{}%
\AgdaBound{ϵ⋆}\AgdaSpace{}%
\AgdaSymbol{→}\AgdaSpace{}%
\AgdaField{◅}\AgdaSpace{}%
\AgdaSymbol{λ}\AgdaSpace{}%
\AgdaBound{σ}\AgdaSpace{}%
\AgdaSymbol{→}\<%
\\
\>[.][@{}l@{}]\<[287I]%
\>[4]\AgdaSymbol{(}\AgdaFunction{\#}\AgdaSpace{}%
\AgdaBound{ϵ⋆}\AgdaSpace{}%
\AgdaOperator{\AgdaFunction{==⊥}}\AgdaSpace{}%
\AgdaNumber{0}\AgdaSymbol{)}\AgdaSpace{}%
\AgdaOperator{\AgdaPostulate{⟶}}\AgdaSpace{}%
\AgdaField{▻}\AgdaSpace{}%
\AgdaSymbol{(}\AgdaBound{ψ}\AgdaSpace{}%
\AgdaFunction{⟨⟩}\AgdaSymbol{)}\AgdaSpace{}%
\AgdaBound{σ}\AgdaSpace{}%
\AgdaOperator{\AgdaPostulate{,}}\<%
\\
\>[4][@{}l@{\AgdaIndent{0}}]%
\>[6]\AgdaSymbol{((}\AgdaPostulate{new}\AgdaSpace{}%
\AgdaBound{σ}\AgdaSpace{}%
\AgdaOperator{\AgdaFunction{∈𝐋}}\AgdaSymbol{)}\AgdaSpace{}%
\AgdaOperator{\AgdaPostulate{⟶}}\<%
\\
\>[6][@{}l@{\AgdaIndent{0}}]%
\>[10]\AgdaField{▻}%
\>[13]\AgdaSymbol{(}\AgdaBound{tievals′}\AgdaSpace{}%
\AgdaSymbol{(λ}\AgdaSpace{}%
\AgdaBound{α⋆}\AgdaSpace{}%
\AgdaSymbol{→}\AgdaSpace{}%
\AgdaBound{ψ}\AgdaSpace{}%
\AgdaSymbol{(}\AgdaOperator{\AgdaFunction{⟨}}\AgdaSpace{}%
\AgdaPostulate{new}\AgdaSpace{}%
\AgdaBound{σ}\AgdaSpace{}%
\AgdaOperator{\AgdaFunction{|𝐋}}\AgdaSpace{}%
\AgdaOperator{\AgdaFunction{⟩}}\AgdaSpace{}%
\AgdaOperator{\AgdaFunction{§}}\AgdaSpace{}%
\AgdaBound{α⋆}\AgdaSymbol{))}\AgdaSpace{}%
\AgdaSymbol{(}\AgdaBound{ϵ⋆}\AgdaSpace{}%
\AgdaOperator{\AgdaFunction{†}}\AgdaSpace{}%
\AgdaNumber{1}\AgdaSymbol{))}\<%
\\
%
\>[13]\AgdaSymbol{(}\AgdaFunction{update}\AgdaSpace{}%
\AgdaSymbol{(}\AgdaPostulate{new}\AgdaSpace{}%
\AgdaBound{σ}\AgdaSpace{}%
\AgdaOperator{\AgdaFunction{|𝐋}}\AgdaSymbol{)}\AgdaSpace{}%
\AgdaSymbol{(}\AgdaBound{ϵ⋆}\AgdaSpace{}%
\AgdaOperator{\AgdaFunction{↓}}\AgdaSpace{}%
\AgdaNumber{1}\AgdaSymbol{)}\AgdaSpace{}%
\AgdaBound{σ}\AgdaSymbol{)}\AgdaSpace{}%
\AgdaOperator{\AgdaPostulate{,}}\<%
\\
\>[6][@{}l@{\AgdaIndent{0}}]%
\>[8]\AgdaField{▻}\AgdaSpace{}%
\AgdaSymbol{(}\AgdaPostulate{wrong}\AgdaSpace{}%
\AgdaString{"out\ of\ memory"}\AgdaSymbol{)}\AgdaSpace{}%
\AgdaBound{σ}\AgdaSpace{}%
\AgdaSymbol{)}\<%
\\
%
\\[\AgdaEmptyExtraSkip]%
\>[0]\AgdaFunction{list}\AgdaSpace{}%
\AgdaSymbol{:}\AgdaSpace{}%
\AgdaPostulate{𝐄}\AgdaSpace{}%
\AgdaOperator{\AgdaFunction{⋆}}\AgdaSpace{}%
\AgdaSymbol{→}\AgdaSpace{}%
\AgdaPostulate{𝐊}\AgdaSpace{}%
\AgdaSymbol{→}\AgdaSpace{}%
\AgdaPostulate{𝐂}\<%
\\
\>[0]\AgdaComment{--\ Add\ declarations:}\<%
\\
\>[0]\AgdaFunction{dropfirst}\AgdaSpace{}%
\AgdaSymbol{:}\AgdaSpace{}%
\AgdaPostulate{𝐄}\AgdaSpace{}%
\AgdaOperator{\AgdaFunction{⋆}}\AgdaSpace{}%
\AgdaSymbol{→}\AgdaSpace{}%
\AgdaFunction{𝐍}\AgdaSpace{}%
\AgdaSymbol{→}\AgdaSpace{}%
\AgdaPostulate{𝐄}\AgdaSpace{}%
\AgdaOperator{\AgdaFunction{⋆}}\<%
\\
\>[0]\AgdaFunction{takefirst}\AgdaSpace{}%
\AgdaSymbol{:}\AgdaSpace{}%
\AgdaPostulate{𝐄}\AgdaSpace{}%
\AgdaOperator{\AgdaFunction{⋆}}\AgdaSpace{}%
\AgdaSymbol{→}\AgdaSpace{}%
\AgdaFunction{𝐍}\AgdaSpace{}%
\AgdaSymbol{→}\AgdaSpace{}%
\AgdaPostulate{𝐄}\AgdaSpace{}%
\AgdaOperator{\AgdaFunction{⋆}}\<%
\\
%
\\[\AgdaEmptyExtraSkip]%
\>[0]\AgdaFunction{tievalsrest}\AgdaSpace{}%
\AgdaSymbol{:}\AgdaSpace{}%
\AgdaSymbol{(}\AgdaFunction{𝐋}\AgdaSpace{}%
\AgdaOperator{\AgdaFunction{⋆}}\AgdaSpace{}%
\AgdaSymbol{→}\AgdaSpace{}%
\AgdaPostulate{𝐂}\AgdaSymbol{)}\AgdaSpace{}%
\AgdaSymbol{→}\AgdaSpace{}%
\AgdaPostulate{𝐄}\AgdaSpace{}%
\AgdaOperator{\AgdaFunction{⋆}}\AgdaSpace{}%
\AgdaSymbol{→}\AgdaSpace{}%
\AgdaFunction{𝐍}\AgdaSpace{}%
\AgdaSymbol{→}\AgdaSpace{}%
\AgdaPostulate{𝐂}\<%
\\
\>[0]\AgdaFunction{tievalsrest}\AgdaSpace{}%
\AgdaSymbol{=}\<%
\\
\>[0][@{}l@{\AgdaIndent{0}}]%
\>[2]\AgdaSymbol{λ}\AgdaSpace{}%
\AgdaBound{ψ}\AgdaSpace{}%
\AgdaBound{ϵ⋆}\AgdaSpace{}%
\AgdaBound{ν}\AgdaSpace{}%
\AgdaSymbol{→}\AgdaSpace{}%
\AgdaFunction{list}%
\>[19]\AgdaSymbol{(}\AgdaFunction{dropfirst}\AgdaSpace{}%
\AgdaBound{ϵ⋆}\AgdaSpace{}%
\AgdaBound{ν}\AgdaSymbol{)}\<%
\\
%
\>[19]\AgdaSymbol{(}\AgdaFunction{single}\AgdaSpace{}%
\AgdaSymbol{(λ}\AgdaSpace{}%
\AgdaBound{ϵ}\AgdaSpace{}%
\AgdaSymbol{→}\AgdaSpace{}%
\AgdaFunction{tievals}\AgdaSpace{}%
\AgdaBound{ψ}\AgdaSpace{}%
\AgdaSymbol{((}\AgdaFunction{takefirst}\AgdaSpace{}%
\AgdaBound{ϵ⋆}\AgdaSpace{}%
\AgdaBound{ν}\AgdaSymbol{)}\AgdaSpace{}%
\AgdaOperator{\AgdaFunction{§}}\AgdaSpace{}%
\AgdaOperator{\AgdaFunction{⟨}}\AgdaSpace{}%
\AgdaBound{ϵ}\AgdaSpace{}%
\AgdaOperator{\AgdaFunction{⟩}}\AgdaSymbol{)))}\<%
\\
%
\\[\AgdaEmptyExtraSkip]%
\>[0]\AgdaFunction{dropfirst}\AgdaSpace{}%
\AgdaSymbol{=}\AgdaSpace{}%
\AgdaPostulate{fix}\AgdaSpace{}%
\AgdaSymbol{λ}\AgdaSpace{}%
\AgdaBound{dropfirst′}\AgdaSpace{}%
\AgdaSymbol{→}\<%
\\
\>[0][@{}l@{\AgdaIndent{0}}]%
\>[2]\AgdaSymbol{λ}%
\>[392I]\AgdaBound{ϵ⋆}\AgdaSpace{}%
\AgdaBound{ν}\AgdaSpace{}%
\AgdaSymbol{→}\<%
\\
\>[.][@{}l@{}]\<[392I]%
\>[4]\AgdaSymbol{(}\AgdaBound{ν}\AgdaSpace{}%
\AgdaOperator{\AgdaFunction{==⊥}}\AgdaSpace{}%
\AgdaNumber{0}\AgdaSymbol{)}\AgdaSpace{}%
\AgdaOperator{\AgdaPostulate{⟶}}\AgdaSpace{}%
\AgdaBound{ϵ⋆}\AgdaSpace{}%
\AgdaOperator{\AgdaPostulate{,}}\<%
\\
\>[4][@{}l@{\AgdaIndent{0}}]%
\>[6]\AgdaBound{dropfirst′}\AgdaSpace{}%
\AgdaSymbol{(}\AgdaBound{ϵ⋆}\AgdaSpace{}%
\AgdaOperator{\AgdaFunction{†}}\AgdaSpace{}%
\AgdaNumber{1}\AgdaSymbol{)}\AgdaSpace{}%
\AgdaSymbol{(((}\AgdaPostulate{η}\AgdaSpace{}%
\AgdaOperator{\AgdaFunction{∘}}\AgdaSpace{}%
\AgdaFunction{pred}\AgdaSymbol{)}\AgdaSpace{}%
\AgdaOperator{\AgdaPostulate{♯}}\AgdaSymbol{)}\AgdaSpace{}%
\AgdaBound{ν}\AgdaSymbol{)}\<%
\\
%
\\[\AgdaEmptyExtraSkip]%
\>[0]\AgdaFunction{takefirst}\AgdaSpace{}%
\AgdaSymbol{=}\AgdaSpace{}%
\AgdaPostulate{fix}\AgdaSpace{}%
\AgdaSymbol{λ}\AgdaSpace{}%
\AgdaBound{takefirst′}\AgdaSpace{}%
\AgdaSymbol{→}\<%
\\
\>[0][@{}l@{\AgdaIndent{0}}]%
\>[2]\AgdaSymbol{λ}%
\>[413I]\AgdaBound{ϵ⋆}\AgdaSpace{}%
\AgdaBound{ν}\AgdaSpace{}%
\AgdaSymbol{→}\<%
\\
\>[.][@{}l@{}]\<[413I]%
\>[4]\AgdaSymbol{(}\AgdaBound{ν}\AgdaSpace{}%
\AgdaOperator{\AgdaFunction{==⊥}}\AgdaSpace{}%
\AgdaNumber{0}\AgdaSymbol{)}\AgdaSpace{}%
\AgdaOperator{\AgdaPostulate{⟶}}\AgdaSpace{}%
\AgdaFunction{⟨⟩}\AgdaSpace{}%
\AgdaOperator{\AgdaPostulate{,}}\<%
\\
\>[4][@{}l@{\AgdaIndent{0}}]%
\>[6]\AgdaSymbol{(}\AgdaSpace{}%
\AgdaOperator{\AgdaFunction{⟨}}\AgdaSpace{}%
\AgdaBound{ϵ⋆}\AgdaSpace{}%
\AgdaOperator{\AgdaFunction{↓}}\AgdaSpace{}%
\AgdaNumber{1}\AgdaSpace{}%
\AgdaOperator{\AgdaFunction{⟩}}\AgdaSpace{}%
\AgdaOperator{\AgdaFunction{§}}\AgdaSpace{}%
\AgdaSymbol{(}\AgdaBound{takefirst′}\AgdaSpace{}%
\AgdaSymbol{(}\AgdaBound{ϵ⋆}\AgdaSpace{}%
\AgdaOperator{\AgdaFunction{†}}\AgdaSpace{}%
\AgdaNumber{1}\AgdaSymbol{)}\AgdaSpace{}%
\AgdaSymbol{(((}\AgdaPostulate{η}\AgdaSpace{}%
\AgdaOperator{\AgdaFunction{∘}}\AgdaSpace{}%
\AgdaFunction{pred}\AgdaSymbol{)}\AgdaSpace{}%
\AgdaOperator{\AgdaPostulate{♯}}\AgdaSymbol{)}\AgdaSpace{}%
\AgdaBound{ν}\AgdaSymbol{))}\AgdaSpace{}%
\AgdaSymbol{)}\<%
\\
%
\\[\AgdaEmptyExtraSkip]%
\>[0]\AgdaFunction{truish}\AgdaSpace{}%
\AgdaSymbol{:}\AgdaSpace{}%
\AgdaPostulate{𝐄}\AgdaSpace{}%
\AgdaSymbol{→}\AgdaSpace{}%
\AgdaFunction{𝐓}\<%
\\
\>[0]\AgdaComment{--\ truish\ =\ λ\ ϵ\ →\ ϵ\ =\ false\ ⟶\ false\ ,\ true}\<%
\\
\>[0]\AgdaFunction{truish}\AgdaSpace{}%
\AgdaSymbol{=}\AgdaSpace{}%
\AgdaSymbol{λ}\AgdaSpace{}%
\AgdaBound{ϵ}\AgdaSpace{}%
\AgdaSymbol{→}\AgdaSpace{}%
\AgdaSymbol{(}\AgdaFunction{misc-false}\AgdaSpace{}%
\AgdaOperator{\AgdaPostulate{♯}}\AgdaSymbol{)}\AgdaSpace{}%
\AgdaSymbol{(}\AgdaField{▻}\AgdaSpace{}%
\AgdaBound{ϵ}\AgdaSymbol{)}\AgdaSpace{}%
\AgdaOperator{\AgdaPostulate{⟶}}\AgdaSpace{}%
\AgdaSymbol{(}\AgdaPostulate{η}\AgdaSpace{}%
\AgdaInductiveConstructor{false}\AgdaSymbol{)}\AgdaSpace{}%
\AgdaOperator{\AgdaPostulate{,}}\AgdaSpace{}%
\AgdaSymbol{(}\AgdaPostulate{η}\AgdaSpace{}%
\AgdaInductiveConstructor{true}\AgdaSymbol{)}\AgdaSpace{}%
\AgdaKeyword{where}\<%
\\
\>[0][@{}l@{\AgdaIndent{0}}]%
\>[2]\AgdaFunction{misc-false}\AgdaSpace{}%
\AgdaSymbol{:}\AgdaSpace{}%
\AgdaSymbol{(}\AgdaPostulate{𝐐}\AgdaSpace{}%
\AgdaOperator{\AgdaDatatype{+}}\AgdaSpace{}%
\AgdaPostulate{𝐇}\AgdaSpace{}%
\AgdaOperator{\AgdaDatatype{+}}\AgdaSpace{}%
\AgdaPostulate{𝐑}\AgdaSpace{}%
\AgdaOperator{\AgdaDatatype{+}}\AgdaSpace{}%
\AgdaFunction{𝐄𝐩}\AgdaSpace{}%
\AgdaOperator{\AgdaDatatype{+}}\AgdaSpace{}%
\AgdaFunction{𝐄𝐯}\AgdaSpace{}%
\AgdaOperator{\AgdaDatatype{+}}\AgdaSpace{}%
\AgdaFunction{𝐄𝐬}\AgdaSpace{}%
\AgdaOperator{\AgdaDatatype{+}}\AgdaSpace{}%
\AgdaFunction{𝐌}\AgdaSpace{}%
\AgdaOperator{\AgdaDatatype{+}}\AgdaSpace{}%
\AgdaPostulate{𝐅}\AgdaSymbol{)}\AgdaSpace{}%
\AgdaSymbol{→}\AgdaSpace{}%
\AgdaPostulate{𝕃}\AgdaSpace{}%
\AgdaDatatype{Bool}\<%
\\
%
\>[2]\AgdaFunction{misc-false}\AgdaSpace{}%
\AgdaSymbol{(}\AgdaInductiveConstructor{inj-𝐌}\AgdaSpace{}%
\AgdaBound{μ}\AgdaSymbol{)}%
\>[24]\AgdaSymbol{=}\AgdaSpace{}%
\AgdaSymbol{((λ}\AgdaSpace{}%
\AgdaSymbol{\{}\AgdaSpace{}%
\AgdaInductiveConstructor{false}\AgdaSpace{}%
\AgdaSymbol{→}\AgdaSpace{}%
\AgdaPostulate{η}\AgdaSpace{}%
\AgdaInductiveConstructor{true}\AgdaSpace{}%
\AgdaSymbol{;}\AgdaSpace{}%
\AgdaCatchallClause{\AgdaSymbol{\AgdaUnderscore{}}}\AgdaSpace{}%
\AgdaSymbol{→}\AgdaSpace{}%
\AgdaPostulate{η}\AgdaSpace{}%
\AgdaInductiveConstructor{false}\AgdaSpace{}%
\AgdaSymbol{\})}\AgdaSpace{}%
\AgdaOperator{\AgdaPostulate{♯}}\AgdaSymbol{)}\AgdaSpace{}%
\AgdaSymbol{(}\AgdaBound{μ}\AgdaSymbol{)}\<%
\\
%
\>[2]\AgdaFunction{misc-false}\AgdaSpace{}%
\AgdaSymbol{(}\AgdaInductiveConstructor{inj₁}\AgdaSpace{}%
\AgdaSymbol{\AgdaUnderscore{})}%
\>[24]\AgdaSymbol{=}\AgdaSpace{}%
\AgdaPostulate{η}\AgdaSpace{}%
\AgdaInductiveConstructor{false}\<%
\\
%
\>[2]\AgdaCatchallClause{\AgdaFunction{misc-false}}\AgdaSpace{}%
\AgdaCatchallClause{\AgdaSymbol{(}}\AgdaCatchallClause{\AgdaInductiveConstructor{inj₂}}\AgdaSpace{}%
\AgdaCatchallClause{\AgdaSymbol{\AgdaUnderscore{})}}%
\>[24]\AgdaSymbol{=}\AgdaSpace{}%
\AgdaPostulate{η}\AgdaSpace{}%
\AgdaInductiveConstructor{false}\<%
\\
%
\\[\AgdaEmptyExtraSkip]%
\>[0]\AgdaComment{--\ Added:}\<%
\\
\>[0]\AgdaFunction{misc-undefined}\AgdaSpace{}%
\AgdaSymbol{:}\AgdaSpace{}%
\AgdaSymbol{(}\AgdaPostulate{𝐐}\AgdaSpace{}%
\AgdaOperator{\AgdaDatatype{+}}\AgdaSpace{}%
\AgdaPostulate{𝐇}\AgdaSpace{}%
\AgdaOperator{\AgdaDatatype{+}}\AgdaSpace{}%
\AgdaPostulate{𝐑}\AgdaSpace{}%
\AgdaOperator{\AgdaDatatype{+}}\AgdaSpace{}%
\AgdaFunction{𝐄𝐩}\AgdaSpace{}%
\AgdaOperator{\AgdaDatatype{+}}\AgdaSpace{}%
\AgdaFunction{𝐄𝐯}\AgdaSpace{}%
\AgdaOperator{\AgdaDatatype{+}}\AgdaSpace{}%
\AgdaFunction{𝐄𝐬}\AgdaSpace{}%
\AgdaOperator{\AgdaDatatype{+}}\AgdaSpace{}%
\AgdaFunction{𝐌}\AgdaSpace{}%
\AgdaOperator{\AgdaDatatype{+}}\AgdaSpace{}%
\AgdaPostulate{𝐅}\AgdaSymbol{)}\AgdaSpace{}%
\AgdaSymbol{→}\AgdaSpace{}%
\AgdaPostulate{𝕃}\AgdaSpace{}%
\AgdaDatatype{Bool}\<%
\\
\>[0]\AgdaFunction{misc-undefined}\AgdaSpace{}%
\AgdaSymbol{(}\AgdaInductiveConstructor{inj-𝐌}\AgdaSpace{}%
\AgdaBound{μ}\AgdaSymbol{)}%
\>[26]\AgdaSymbol{=}\AgdaSpace{}%
\AgdaSymbol{((λ}\AgdaSpace{}%
\AgdaSymbol{\{}\AgdaSpace{}%
\AgdaInductiveConstructor{undefined}\AgdaSpace{}%
\AgdaSymbol{→}\AgdaSpace{}%
\AgdaPostulate{η}\AgdaSpace{}%
\AgdaInductiveConstructor{true}\AgdaSpace{}%
\AgdaSymbol{;}\AgdaSpace{}%
\AgdaCatchallClause{\AgdaSymbol{\AgdaUnderscore{}}}\AgdaSpace{}%
\AgdaSymbol{→}\AgdaSpace{}%
\AgdaPostulate{η}\AgdaSpace{}%
\AgdaInductiveConstructor{false}\AgdaSpace{}%
\AgdaSymbol{\})}\AgdaSpace{}%
\AgdaOperator{\AgdaPostulate{♯}}\AgdaSymbol{)}\AgdaSpace{}%
\AgdaSymbol{(}\AgdaBound{μ}\AgdaSymbol{)}\<%
\\
\>[0]\AgdaFunction{misc-undefined}\AgdaSpace{}%
\AgdaSymbol{(}\AgdaInductiveConstructor{inj₁}\AgdaSpace{}%
\AgdaSymbol{\AgdaUnderscore{})}%
\>[27]\AgdaSymbol{=}\AgdaSpace{}%
\AgdaPostulate{η}\AgdaSpace{}%
\AgdaInductiveConstructor{false}\<%
\\
\>[0]\AgdaCatchallClause{\AgdaFunction{misc-undefined}}\AgdaSpace{}%
\AgdaCatchallClause{\AgdaSymbol{(}}\AgdaCatchallClause{\AgdaInductiveConstructor{inj₂}}\AgdaSpace{}%
\AgdaCatchallClause{\AgdaSymbol{\AgdaUnderscore{})}}%
\>[27]\AgdaSymbol{=}\AgdaSpace{}%
\AgdaPostulate{η}\AgdaSpace{}%
\AgdaInductiveConstructor{false}\<%
\\
%
\\[\AgdaEmptyExtraSkip]%
\>[0]\AgdaComment{--\ permute\ \ \ \ :\ Exp\ *\ →\ Exp\ *\ \ --\ implementation-dependent}\<%
\\
\>[0]\AgdaComment{--\ unpermute\ \ :\ 𝐄\ ⋆\ →\ 𝐄\ ⋆\ \ \ \ \ \ --\ inverse\ of\ permute}\<%
\\
%
\\[\AgdaEmptyExtraSkip]%
\>[0]\AgdaFunction{applicate}\AgdaSpace{}%
\AgdaSymbol{:}\AgdaSpace{}%
\AgdaPostulate{𝐄}\AgdaSpace{}%
\AgdaSymbol{→}\AgdaSpace{}%
\AgdaPostulate{𝐄}\AgdaSpace{}%
\AgdaOperator{\AgdaFunction{⋆}}\AgdaSpace{}%
\AgdaSymbol{→}\AgdaSpace{}%
\AgdaPostulate{𝐊}\AgdaSpace{}%
\AgdaSymbol{→}\AgdaSpace{}%
\AgdaPostulate{𝐂}\<%
\\
\>[0]\AgdaFunction{applicate}\AgdaSpace{}%
\AgdaSymbol{=}\<%
\\
\>[0][@{}l@{\AgdaIndent{0}}]%
\>[2]\AgdaSymbol{λ}%
\>[552I]\AgdaBound{ϵ}\AgdaSpace{}%
\AgdaBound{ϵ⋆}\AgdaSpace{}%
\AgdaBound{κ}\AgdaSpace{}%
\AgdaSymbol{→}\<%
\\
\>[.][@{}l@{}]\<[552I]%
\>[4]\AgdaSymbol{(}\AgdaBound{ϵ}\AgdaSpace{}%
\AgdaOperator{\AgdaFunction{∈𝐅}}\AgdaSymbol{)}\AgdaSpace{}%
\AgdaOperator{\AgdaPostulate{⟶}}\AgdaSpace{}%
\AgdaSymbol{(}\AgdaField{▻}\AgdaSpace{}%
\AgdaSymbol{(}\AgdaBound{ϵ}\AgdaSpace{}%
\AgdaOperator{\AgdaFunction{|𝐅}}\AgdaSymbol{)}\AgdaSpace{}%
\AgdaOperator{\AgdaField{↓2}}\AgdaSymbol{)}\AgdaSpace{}%
\AgdaBound{ϵ⋆}\AgdaSpace{}%
\AgdaBound{κ}\AgdaSpace{}%
\AgdaOperator{\AgdaPostulate{,}}\<%
\\
\>[4][@{}l@{\AgdaIndent{0}}]%
\>[6]\AgdaPostulate{wrong}\AgdaSpace{}%
\AgdaString{"bad\ procedure"}\<%
\\
%
\\[\AgdaEmptyExtraSkip]%
\>[0]\AgdaFunction{onearg}\AgdaSpace{}%
\AgdaSymbol{:}\AgdaSpace{}%
\AgdaSymbol{(}\AgdaPostulate{𝐄}\AgdaSpace{}%
\AgdaSymbol{→}\AgdaSpace{}%
\AgdaPostulate{𝐊}\AgdaSpace{}%
\AgdaSymbol{→}\AgdaSpace{}%
\AgdaPostulate{𝐂}\AgdaSymbol{)}\AgdaSpace{}%
\AgdaSymbol{→}\AgdaSpace{}%
\AgdaSymbol{(}\AgdaPostulate{𝐄}\AgdaSpace{}%
\AgdaOperator{\AgdaFunction{⋆}}\AgdaSpace{}%
\AgdaSymbol{→}\AgdaSpace{}%
\AgdaPostulate{𝐊}\AgdaSpace{}%
\AgdaSymbol{→}\AgdaSpace{}%
\AgdaPostulate{𝐂}\AgdaSymbol{)}\<%
\\
\>[0]\AgdaFunction{onearg}\AgdaSpace{}%
\AgdaSymbol{=}\<%
\\
\>[0][@{}l@{\AgdaIndent{0}}]%
\>[2]\AgdaSymbol{λ}%
\>[580I]\AgdaBound{ζ}\AgdaSpace{}%
\AgdaBound{ϵ⋆}\AgdaSpace{}%
\AgdaBound{κ}\AgdaSpace{}%
\AgdaSymbol{→}\<%
\\
\>[.][@{}l@{}]\<[580I]%
\>[4]\AgdaSymbol{(}\AgdaFunction{\#}\AgdaSpace{}%
\AgdaBound{ϵ⋆}\AgdaSpace{}%
\AgdaOperator{\AgdaFunction{==⊥}}\AgdaSpace{}%
\AgdaNumber{1}\AgdaSymbol{)}\AgdaSpace{}%
\AgdaOperator{\AgdaPostulate{⟶}}\AgdaSpace{}%
\AgdaBound{ζ}\AgdaSpace{}%
\AgdaSymbol{(}\AgdaBound{ϵ⋆}\AgdaSpace{}%
\AgdaOperator{\AgdaFunction{↓}}\AgdaSpace{}%
\AgdaNumber{1}\AgdaSymbol{)}\AgdaSpace{}%
\AgdaBound{κ}\AgdaSpace{}%
\AgdaOperator{\AgdaPostulate{,}}\<%
\\
\>[4][@{}l@{\AgdaIndent{0}}]%
\>[6]\AgdaPostulate{wrong}\AgdaSpace{}%
\AgdaString{"wrong\ number\ of\ arguments"}\<%
\\
%
\\[\AgdaEmptyExtraSkip]%
\>[0]\AgdaFunction{twoarg}\AgdaSpace{}%
\AgdaSymbol{:}\AgdaSpace{}%
\AgdaSymbol{(}\AgdaPostulate{𝐄}\AgdaSpace{}%
\AgdaSymbol{→}\AgdaSpace{}%
\AgdaPostulate{𝐄}\AgdaSpace{}%
\AgdaSymbol{→}\AgdaSpace{}%
\AgdaPostulate{𝐊}\AgdaSpace{}%
\AgdaSymbol{→}\AgdaSpace{}%
\AgdaPostulate{𝐂}\AgdaSymbol{)}\AgdaSpace{}%
\AgdaSymbol{→}\AgdaSpace{}%
\AgdaSymbol{(}\AgdaPostulate{𝐄}\AgdaSpace{}%
\AgdaOperator{\AgdaFunction{⋆}}\AgdaSpace{}%
\AgdaSymbol{→}\AgdaSpace{}%
\AgdaPostulate{𝐊}\AgdaSpace{}%
\AgdaSymbol{→}\AgdaSpace{}%
\AgdaPostulate{𝐂}\AgdaSymbol{)}\<%
\\
\>[0]\AgdaFunction{twoarg}\AgdaSpace{}%
\AgdaSymbol{=}\<%
\\
\>[0][@{}l@{\AgdaIndent{0}}]%
\>[2]\AgdaSymbol{λ}%
\>[611I]\AgdaBound{ζ}\AgdaSpace{}%
\AgdaBound{ϵ⋆}\AgdaSpace{}%
\AgdaBound{κ}\AgdaSpace{}%
\AgdaSymbol{→}\<%
\\
\>[.][@{}l@{}]\<[611I]%
\>[4]\AgdaSymbol{(}\AgdaFunction{\#}\AgdaSpace{}%
\AgdaBound{ϵ⋆}\AgdaSpace{}%
\AgdaOperator{\AgdaFunction{==⊥}}\AgdaSpace{}%
\AgdaNumber{2}\AgdaSymbol{)}\AgdaSpace{}%
\AgdaOperator{\AgdaPostulate{⟶}}\AgdaSpace{}%
\AgdaBound{ζ}\AgdaSpace{}%
\AgdaSymbol{(}\AgdaBound{ϵ⋆}\AgdaSpace{}%
\AgdaOperator{\AgdaFunction{↓}}\AgdaSpace{}%
\AgdaNumber{1}\AgdaSymbol{)}\AgdaSpace{}%
\AgdaSymbol{(}\AgdaBound{ϵ⋆}\AgdaSpace{}%
\AgdaOperator{\AgdaFunction{↓}}\AgdaSpace{}%
\AgdaNumber{2}\AgdaSymbol{)}\AgdaSpace{}%
\AgdaBound{κ}\AgdaSpace{}%
\AgdaOperator{\AgdaPostulate{,}}\<%
\\
\>[4][@{}l@{\AgdaIndent{0}}]%
\>[6]\AgdaPostulate{wrong}\AgdaSpace{}%
\AgdaString{"wrong\ number\ of\ arguments"}\<%
\\
%
\\[\AgdaEmptyExtraSkip]%
\>[0]\AgdaFunction{cons}\AgdaSpace{}%
\AgdaSymbol{:}\AgdaSpace{}%
\AgdaPostulate{𝐄}\AgdaSpace{}%
\AgdaOperator{\AgdaFunction{⋆}}\AgdaSpace{}%
\AgdaSymbol{→}\AgdaSpace{}%
\AgdaPostulate{𝐊}\AgdaSpace{}%
\AgdaSymbol{→}\AgdaSpace{}%
\AgdaPostulate{𝐂}\<%
\\
%
\\[\AgdaEmptyExtraSkip]%
\>[0]\AgdaComment{--\ list\ :\ 𝐄\ ⋆\ →\ 𝐊\ →\ 𝐂}\<%
\\
\>[0]\AgdaFunction{list}\AgdaSpace{}%
\AgdaSymbol{=}\AgdaSpace{}%
\AgdaPostulate{fix}\AgdaSpace{}%
\AgdaSymbol{λ}\AgdaSpace{}%
\AgdaBound{list′}\AgdaSpace{}%
\AgdaSymbol{→}\<%
\\
\>[0][@{}l@{\AgdaIndent{0}}]%
\>[2]\AgdaSymbol{λ}%
\>[641I]\AgdaBound{ϵ⋆}\AgdaSpace{}%
\AgdaBound{κ}\AgdaSpace{}%
\AgdaSymbol{→}\<%
\\
\>[.][@{}l@{}]\<[641I]%
\>[4]\AgdaSymbol{(}\AgdaFunction{\#}\AgdaSpace{}%
\AgdaBound{ϵ⋆}\AgdaSpace{}%
\AgdaOperator{\AgdaFunction{==⊥}}\AgdaSpace{}%
\AgdaNumber{0}\AgdaSymbol{)}\AgdaSpace{}%
\AgdaOperator{\AgdaPostulate{⟶}}\AgdaSpace{}%
\AgdaFunction{send}\AgdaSpace{}%
\AgdaSymbol{(}\AgdaField{◅}\AgdaSpace{}%
\AgdaSymbol{(}\AgdaPostulate{η}\AgdaSpace{}%
\AgdaSymbol{(}\AgdaInductiveConstructor{inj-𝐌}\AgdaSpace{}%
\AgdaSymbol{(}\AgdaPostulate{η}\AgdaSpace{}%
\AgdaInductiveConstructor{null}\AgdaSymbol{))))}\AgdaSpace{}%
\AgdaBound{κ}\AgdaSpace{}%
\AgdaOperator{\AgdaPostulate{,}}\<%
\\
\>[4][@{}l@{\AgdaIndent{0}}]%
\>[6]\AgdaBound{list′}\AgdaSpace{}%
\AgdaSymbol{(}\AgdaBound{ϵ⋆}\AgdaSpace{}%
\AgdaOperator{\AgdaFunction{†}}\AgdaSpace{}%
\AgdaNumber{1}\AgdaSymbol{)}\AgdaSpace{}%
\AgdaSymbol{(}\AgdaFunction{single}\AgdaSpace{}%
\AgdaSymbol{(λ}\AgdaSpace{}%
\AgdaBound{ϵ}\AgdaSpace{}%
\AgdaSymbol{→}\AgdaSpace{}%
\AgdaFunction{cons}\AgdaSpace{}%
\AgdaOperator{\AgdaFunction{⟨}}\AgdaSpace{}%
\AgdaSymbol{(}\AgdaBound{ϵ⋆}\AgdaSpace{}%
\AgdaOperator{\AgdaFunction{↓}}\AgdaSpace{}%
\AgdaNumber{1}\AgdaSymbol{)}\AgdaSpace{}%
\AgdaOperator{\AgdaInductiveConstructor{,}}\AgdaSpace{}%
\AgdaBound{ϵ}\AgdaSpace{}%
\AgdaOperator{\AgdaFunction{⟩}}\AgdaSpace{}%
\AgdaBound{κ}\AgdaSymbol{))}\<%
\\
%
\\[\AgdaEmptyExtraSkip]%
\>[0]\AgdaComment{--\ cons\ :\ 𝐄\ ⋆\ →\ 𝐊\ →\ 𝐂}\<%
\\
\>[0]\AgdaFunction{cons}\AgdaSpace{}%
\AgdaSymbol{=}\AgdaSpace{}%
\AgdaFunction{twoarg}\<%
\\
\>[0][@{}l@{\AgdaIndent{0}}]%
\>[2]\AgdaSymbol{λ}%
\>[674I]\AgdaBound{ϵ₁}\AgdaSpace{}%
\AgdaBound{ϵ₂}\AgdaSpace{}%
\AgdaBound{κ}\AgdaSpace{}%
\AgdaSymbol{→}\AgdaSpace{}%
\AgdaField{◅}\AgdaSpace{}%
\AgdaSymbol{λ}\AgdaSpace{}%
\AgdaBound{σ}\AgdaSpace{}%
\AgdaSymbol{→}\<%
\\
\>[.][@{}l@{}]\<[674I]%
\>[4]\AgdaSymbol{(}\AgdaPostulate{new}\AgdaSpace{}%
\AgdaBound{σ}\AgdaSpace{}%
\AgdaOperator{\AgdaFunction{∈𝐋}}\AgdaSymbol{)}\AgdaSpace{}%
\AgdaOperator{\AgdaPostulate{⟶}}\<%
\\
\>[4][@{}l@{\AgdaIndent{0}}]%
\>[8]\AgdaSymbol{(λ}\AgdaSpace{}%
\AgdaBound{σ′}\AgdaSpace{}%
\AgdaSymbol{→}%
\>[687I]\AgdaSymbol{(}\AgdaPostulate{new}\AgdaSpace{}%
\AgdaBound{σ′}\AgdaSpace{}%
\AgdaOperator{\AgdaFunction{∈𝐋}}\AgdaSymbol{)}\AgdaSpace{}%
\AgdaOperator{\AgdaPostulate{⟶}}\<%
\\
\>[687I][@{}l@{\AgdaIndent{0}}]%
\>[20]\AgdaField{▻}%
\>[691I]\AgdaSymbol{(}\AgdaFunction{send}\AgdaSpace{}%
\AgdaSymbol{((}\AgdaPostulate{new}\AgdaSpace{}%
\AgdaBound{σ}\AgdaSpace{}%
\AgdaOperator{\AgdaFunction{|𝐋}}\AgdaSpace{}%
\AgdaOperator{\AgdaInductiveConstructor{,}}\AgdaSpace{}%
\AgdaPostulate{new}\AgdaSpace{}%
\AgdaBound{σ′}\AgdaSpace{}%
\AgdaOperator{\AgdaFunction{|𝐋}}\AgdaSpace{}%
\AgdaOperator{\AgdaInductiveConstructor{,}}\AgdaSpace{}%
\AgdaSymbol{(}\AgdaPostulate{η}\AgdaSpace{}%
\AgdaInductiveConstructor{true}\AgdaSymbol{))}\AgdaSpace{}%
\AgdaOperator{\AgdaFunction{𝐄𝐩-in-𝐄}}\AgdaSymbol{)}\AgdaSpace{}%
\AgdaBound{κ}\AgdaSymbol{)}\<%
\\
\>[.][@{}l@{}]\<[691I]%
\>[22]\AgdaSymbol{(}\AgdaFunction{update}\AgdaSpace{}%
\AgdaSymbol{(}\AgdaPostulate{new}\AgdaSpace{}%
\AgdaBound{σ′}\AgdaSpace{}%
\AgdaOperator{\AgdaFunction{|𝐋}}\AgdaSymbol{)}\AgdaSpace{}%
\AgdaBound{ϵ₂}\AgdaSpace{}%
\AgdaBound{σ′}\AgdaSymbol{)}\AgdaSpace{}%
\AgdaOperator{\AgdaPostulate{,}}\<%
\\
\>[687I][@{}l@{\AgdaIndent{0}}]%
\>[18]\AgdaField{▻}\AgdaSpace{}%
\AgdaSymbol{(}\AgdaPostulate{wrong}\AgdaSpace{}%
\AgdaString{"out\ of\ memory"}\AgdaSymbol{)}\AgdaSpace{}%
\AgdaBound{σ′}\AgdaSymbol{)}\<%
\\
%
\>[8]\AgdaSymbol{(}\AgdaFunction{update}\AgdaSpace{}%
\AgdaSymbol{(}\AgdaPostulate{new}\AgdaSpace{}%
\AgdaBound{σ}\AgdaSpace{}%
\AgdaOperator{\AgdaFunction{|𝐋}}\AgdaSymbol{)}\AgdaSpace{}%
\AgdaBound{ϵ₁}\AgdaSpace{}%
\AgdaBound{σ}\AgdaSymbol{)}\AgdaSpace{}%
\AgdaOperator{\AgdaPostulate{,}}\<%
\\
\>[4][@{}l@{\AgdaIndent{0}}]%
\>[6]\AgdaField{▻}\AgdaSpace{}%
\AgdaSymbol{(}\AgdaPostulate{wrong}\AgdaSpace{}%
\AgdaString{"out\ of\ memory"}\AgdaSymbol{)}\AgdaSpace{}%
\AgdaBound{σ}\<%
\end{code} 

\clearpage

\begin{code}%
\>[0]\AgdaSymbol{\{-\#}\AgdaSpace{}%
\AgdaKeyword{OPTIONS}\AgdaSpace{}%
\AgdaPragma{--allow-unsolved-metas}\AgdaSpace{}%
\AgdaSymbol{\#-\}}\<%
\\
%
\\[\AgdaEmptyExtraSkip]%
\>[0]\AgdaKeyword{module}\AgdaSpace{}%
\AgdaModule{Scheme.Semantic-Functions}\AgdaSpace{}%
\AgdaKeyword{where}\<%
\\
%
\\[\AgdaEmptyExtraSkip]%
\>[0]\AgdaKeyword{open}\AgdaSpace{}%
\AgdaKeyword{import}\AgdaSpace{}%
\AgdaModule{Scheme.Domain-Notation}\<%
\\
\>[0]\AgdaKeyword{open}\AgdaSpace{}%
\AgdaKeyword{import}\AgdaSpace{}%
\AgdaModule{Scheme.Abstract-Syntax}\<%
\\
\>[0]\AgdaKeyword{open}\AgdaSpace{}%
\AgdaKeyword{import}\AgdaSpace{}%
\AgdaModule{Scheme.Domain-Equations}\<%
\\
\>[0]\AgdaKeyword{open}\AgdaSpace{}%
\AgdaKeyword{import}\AgdaSpace{}%
\AgdaModule{Scheme.Auxiliary-Functions}\<%
\\
%
\\[\AgdaEmptyExtraSkip]%
\>[0]\AgdaComment{--\ 7.2.3.\ Semantic\ functions}\<%
\\
%
\\[\AgdaEmptyExtraSkip]%
\>[0]\AgdaKeyword{postulate}\AgdaSpace{}%
\AgdaOperator{\AgdaPostulate{𝒦⟦\AgdaUnderscore{}⟧}}\AgdaSpace{}%
\AgdaSymbol{:}\AgdaSpace{}%
\AgdaPostulate{Con}\AgdaSpace{}%
\AgdaSymbol{→}\AgdaSpace{}%
\AgdaPostulate{𝐄}\<%
\\
\>[0]\AgdaOperator{\AgdaFunction{ℰ⟦\AgdaUnderscore{}⟧}}%
\>[7]\AgdaSymbol{:}\AgdaSpace{}%
\AgdaDatatype{Exp}\AgdaSpace{}%
\AgdaSymbol{→}\AgdaSpace{}%
\AgdaPostulate{𝐔}\AgdaSpace{}%
\AgdaSymbol{→}\AgdaSpace{}%
\AgdaPostulate{𝐊}\AgdaSpace{}%
\AgdaSymbol{→}\AgdaSpace{}%
\AgdaPostulate{𝐂}\<%
\\
\>[0]\AgdaOperator{\AgdaFunction{ℰ*⟦\AgdaUnderscore{}⟧}}%
\>[7]\AgdaSymbol{:}\AgdaSpace{}%
\AgdaDatatype{Exp}\AgdaSpace{}%
\AgdaOperator{\AgdaFunction{*}}\AgdaSpace{}%
\AgdaSymbol{→}\AgdaSpace{}%
\AgdaPostulate{𝐔}\AgdaSpace{}%
\AgdaSymbol{→}\AgdaSpace{}%
\AgdaPostulate{𝐊}\AgdaSpace{}%
\AgdaSymbol{→}\AgdaSpace{}%
\AgdaPostulate{𝐂}\<%
\\
\>[0]\AgdaOperator{\AgdaFunction{𝒞*⟦\AgdaUnderscore{}⟧}}%
\>[7]\AgdaSymbol{:}\AgdaSpace{}%
\AgdaFunction{Com}\AgdaSpace{}%
\AgdaOperator{\AgdaFunction{*}}\AgdaSpace{}%
\AgdaSymbol{→}\AgdaSpace{}%
\AgdaPostulate{𝐔}\AgdaSpace{}%
\AgdaSymbol{→}\AgdaSpace{}%
\AgdaPostulate{𝐂}\AgdaSpace{}%
\AgdaSymbol{→}\AgdaSpace{}%
\AgdaPostulate{𝐂}\<%
\\
%
\\[\AgdaEmptyExtraSkip]%
\>[0]\AgdaComment{--\ Definition\ of\ 𝒦\ deliberately\ omitted.}\<%
\\
%
\\[\AgdaEmptyExtraSkip]%
\>[0]\AgdaOperator{\AgdaFunction{ℰ⟦}}\AgdaSpace{}%
\AgdaInductiveConstructor{con}\AgdaSpace{}%
\AgdaBound{K}\AgdaSpace{}%
\AgdaOperator{\AgdaFunction{⟧}}\AgdaSpace{}%
\AgdaSymbol{=}\AgdaSpace{}%
\AgdaSymbol{λ}\AgdaSpace{}%
\AgdaBound{ρ}\AgdaSpace{}%
\AgdaBound{κ}\AgdaSpace{}%
\AgdaSymbol{→}\AgdaSpace{}%
\AgdaFunction{send}\AgdaSpace{}%
\AgdaSymbol{(}\AgdaOperator{\AgdaPostulate{𝒦⟦}}\AgdaSpace{}%
\AgdaBound{K}\AgdaSpace{}%
\AgdaOperator{\AgdaPostulate{⟧}}\AgdaSymbol{)}\AgdaSpace{}%
\AgdaBound{κ}\<%
\\
%
\\[\AgdaEmptyExtraSkip]%
\>[0]\AgdaOperator{\AgdaFunction{ℰ⟦}}\AgdaSpace{}%
\AgdaInductiveConstructor{ide}\AgdaSpace{}%
\AgdaBound{I}\AgdaSpace{}%
\AgdaOperator{\AgdaFunction{⟧}}\AgdaSpace{}%
\AgdaSymbol{=}\AgdaSpace{}%
\AgdaSymbol{λ}\AgdaSpace{}%
\AgdaBound{ρ}\AgdaSpace{}%
\AgdaBound{κ}\AgdaSpace{}%
\AgdaSymbol{→}\<%
\\
\>[0][@{}l@{\AgdaIndent{0}}]%
\>[2]\AgdaFunction{hold}\AgdaSpace{}%
\AgdaSymbol{(}\AgdaFunction{lookup}\AgdaSpace{}%
\AgdaBound{ρ}\AgdaSpace{}%
\AgdaBound{I}\AgdaSymbol{)}\AgdaSpace{}%
\AgdaSymbol{(}\AgdaFunction{single}\AgdaSpace{}%
\AgdaSymbol{(λ}\AgdaSpace{}%
\AgdaBound{ϵ}\AgdaSpace{}%
\AgdaSymbol{→}\<%
\\
\>[2][@{}l@{\AgdaIndent{0}}]%
\>[4]\AgdaSymbol{(}\AgdaFunction{misc-undefined}\AgdaSpace{}%
\AgdaOperator{\AgdaPostulate{♯}}\AgdaSymbol{)}\AgdaSpace{}%
\AgdaSymbol{(}\AgdaField{▻}\AgdaSpace{}%
\AgdaBound{ϵ}\AgdaSymbol{)}\AgdaSpace{}%
\AgdaOperator{\AgdaPostulate{⟶}}\AgdaSpace{}%
\AgdaPostulate{wrong}\AgdaSpace{}%
\AgdaString{"undefined\ variable"}\AgdaSpace{}%
\AgdaOperator{\AgdaPostulate{,}}\<%
\\
\>[4][@{}l@{\AgdaIndent{0}}]%
\>[6]\AgdaFunction{send}\AgdaSpace{}%
\AgdaBound{ϵ}\AgdaSpace{}%
\AgdaBound{κ}\AgdaSymbol{))}\<%
\\
%
\\[\AgdaEmptyExtraSkip]%
\>[0]\AgdaComment{--\ Non-compositional:}\<%
\\
\>[0]\AgdaComment{--\ ℰ⟦\ ⦅\ E₀\ ␣\ E*\ ⦆\ ⟧\ =}\<%
\\
\>[0]\AgdaComment{--\ \ \ λ\ ρ\ κ\ →\ ℰ*⟦\ permute\ (⟨\ E₀\ ⟩\ §\ E*\ )\ ⟧}\<%
\\
\>[0]\AgdaComment{--\ \ \ \ \ \ \ \ \ \ \ ρ}\<%
\\
\>[0]\AgdaComment{--\ \ \ \ \ \ \ \ \ \ \ (λ\ ϵ⋆\ →\ ((λ\ ϵ⋆\ →\ applicate\ (ϵ⋆\ ↓\ 1)\ (ϵ⋆\ †\ 1)\ κ)}\<%
\\
\>[0]\AgdaComment{--\ \ \ \ \ \ \ \ \ \ \ \ \ \ \ \ \ \ \ \ (unpermute\ ϵ⋆)))}\<%
\\
%
\\[\AgdaEmptyExtraSkip]%
\>[0]\AgdaOperator{\AgdaFunction{ℰ⟦}}\AgdaSpace{}%
\AgdaOperator{\AgdaInductiveConstructor{⦅}}\AgdaSpace{}%
\AgdaBound{E*}\AgdaSpace{}%
\AgdaOperator{\AgdaInductiveConstructor{⦆}}\AgdaSpace{}%
\AgdaOperator{\AgdaFunction{⟧}}\AgdaSpace{}%
\AgdaSymbol{=}\AgdaSpace{}%
\AgdaSymbol{λ}\AgdaSpace{}%
\AgdaBound{ρ}\AgdaSpace{}%
\AgdaBound{κ}\AgdaSpace{}%
\AgdaSymbol{→}\<%
\\
\>[0][@{}l@{\AgdaIndent{0}}]%
\>[2]\AgdaOperator{\AgdaFunction{ℰ*⟦}}\AgdaSpace{}%
\AgdaBound{E*}\AgdaSpace{}%
\AgdaOperator{\AgdaFunction{⟧}}\AgdaSpace{}%
\AgdaBound{ρ}\AgdaSpace{}%
\AgdaSymbol{(}\AgdaField{◅}\AgdaSpace{}%
\AgdaSymbol{λ}\AgdaSpace{}%
\AgdaBound{ϵ⋆}\AgdaSpace{}%
\AgdaSymbol{→}\<%
\\
\>[2][@{}l@{\AgdaIndent{0}}]%
\>[4]\AgdaFunction{applicate}\AgdaSpace{}%
\AgdaSymbol{(}\AgdaBound{ϵ⋆}\AgdaSpace{}%
\AgdaOperator{\AgdaFunction{↓}}\AgdaSpace{}%
\AgdaNumber{1}\AgdaSymbol{)}\AgdaSpace{}%
\AgdaSymbol{(}\AgdaBound{ϵ⋆}\AgdaSpace{}%
\AgdaOperator{\AgdaFunction{†}}\AgdaSpace{}%
\AgdaNumber{1}\AgdaSymbol{)}\AgdaSpace{}%
\AgdaBound{κ}\AgdaSymbol{)}\<%
\\
%
\\[\AgdaEmptyExtraSkip]%
\>[0]\AgdaOperator{\AgdaFunction{ℰ⟦}}%
\>[101I]\AgdaOperator{\AgdaInductiveConstructor{⦅lambda␣⦅}}\AgdaSpace{}%
\AgdaBound{I*}\AgdaSpace{}%
\AgdaOperator{\AgdaInductiveConstructor{⦆}}\AgdaSpace{}%
\AgdaBound{Γ*}\AgdaSpace{}%
\AgdaOperator{\AgdaInductiveConstructor{␣}}\AgdaSpace{}%
\AgdaBound{E₀}\AgdaSpace{}%
\AgdaOperator{\AgdaInductiveConstructor{⦆}}\AgdaSpace{}%
\AgdaOperator{\AgdaFunction{⟧}}\AgdaSpace{}%
\AgdaSymbol{=}\AgdaSpace{}%
\AgdaSymbol{λ}\AgdaSpace{}%
\AgdaBound{ρ}\AgdaSpace{}%
\AgdaBound{κ}\AgdaSpace{}%
\AgdaSymbol{→}\AgdaSpace{}%
\AgdaField{◅}\AgdaSpace{}%
\AgdaSymbol{λ}\AgdaSpace{}%
\AgdaBound{σ}\AgdaSpace{}%
\AgdaSymbol{→}\<%
\\
\>[101I][@{}l@{\AgdaIndent{0}}]%
\>[4]\AgdaSymbol{(}\AgdaPostulate{new}\AgdaSpace{}%
\AgdaBound{σ}\AgdaSpace{}%
\AgdaOperator{\AgdaFunction{∈𝐋}}\AgdaSymbol{)}\AgdaSpace{}%
\AgdaOperator{\AgdaPostulate{⟶}}\<%
\\
\>[4][@{}l@{\AgdaIndent{0}}]%
\>[8]\AgdaField{▻}%
\>[121I]\AgdaSymbol{(}\AgdaFunction{send}%
\>[122I]\AgdaSymbol{(}\AgdaField{◅}%
\>[123I]\AgdaSymbol{(}%
\>[124I]\AgdaSymbol{(}\AgdaPostulate{new}\AgdaSpace{}%
\AgdaBound{σ}\AgdaSpace{}%
\AgdaOperator{\AgdaFunction{|𝐋}}\AgdaSymbol{)}\AgdaSpace{}%
\AgdaOperator{\AgdaInductiveConstructor{,}}\<%
\\
\>[.][@{}l@{}]\<[124I]%
\>[21]\AgdaSymbol{(λ}%
\>[128I]\AgdaBound{ϵ⋆}\AgdaSpace{}%
\AgdaBound{κ′}\AgdaSpace{}%
\AgdaSymbol{→}\<%
\\
\>[.][@{}l@{}]\<[128I]%
\>[24]\AgdaSymbol{(}\AgdaFunction{\#}%
\>[131I]\AgdaBound{ϵ⋆}\AgdaSpace{}%
\AgdaOperator{\AgdaFunction{==⊥}}\AgdaSpace{}%
\AgdaFunction{\#′}\AgdaSpace{}%
\AgdaBound{I*}\AgdaSymbol{)}\AgdaSpace{}%
\AgdaOperator{\AgdaPostulate{⟶}}\<%
\\
\>[131I][@{}l@{\AgdaIndent{0}}]%
\>[28]\AgdaFunction{tievals}\<%
\\
\>[28][@{}l@{\AgdaIndent{0}}]%
\>[30]\AgdaSymbol{(λ}\AgdaSpace{}%
\AgdaBound{α⋆}\AgdaSpace{}%
\AgdaSymbol{→}%
\>[138I]\AgdaSymbol{(λ}\AgdaSpace{}%
\AgdaBound{ρ′}\AgdaSpace{}%
\AgdaSymbol{→}\AgdaSpace{}%
\AgdaOperator{\AgdaFunction{𝒞*⟦}}\AgdaSpace{}%
\AgdaBound{Γ*}\AgdaSpace{}%
\AgdaOperator{\AgdaFunction{⟧}}\AgdaSpace{}%
\AgdaBound{ρ′}\AgdaSpace{}%
\AgdaSymbol{(}\AgdaOperator{\AgdaFunction{ℰ⟦}}\AgdaSpace{}%
\AgdaBound{E₀}\AgdaSpace{}%
\AgdaOperator{\AgdaFunction{⟧}}\AgdaSpace{}%
\AgdaBound{ρ′}\AgdaSpace{}%
\AgdaBound{κ′}\AgdaSymbol{))}\<%
\\
\>[.][@{}l@{}]\<[138I]%
\>[38]\AgdaSymbol{(}\AgdaFunction{extends}\AgdaSpace{}%
\AgdaBound{ρ}\AgdaSpace{}%
\AgdaBound{I*}\AgdaSpace{}%
\AgdaBound{α⋆}\AgdaSymbol{))}\<%
\\
%
\>[30]\AgdaBound{ϵ⋆}\AgdaSpace{}%
\AgdaOperator{\AgdaPostulate{,}}\<%
\\
\>[24][@{}l@{\AgdaIndent{0}}]%
\>[26]\AgdaPostulate{wrong}\AgdaSpace{}%
\AgdaString{"wrong\ number\ of\ arguments"}\<%
\\
%
\>[21]\AgdaSymbol{)}\<%
\\
\>[.][@{}l@{}]\<[123I]%
\>[19]\AgdaSymbol{)}\AgdaSpace{}%
\AgdaOperator{\AgdaFunction{𝐅-in-𝐄}}\AgdaSymbol{)}\<%
\\
\>[.][@{}l@{}]\<[122I]%
\>[16]\AgdaBound{κ}\AgdaSymbol{)}\<%
\\
\>[.][@{}l@{}]\<[121I]%
\>[10]\AgdaSymbol{(}\AgdaFunction{update}\AgdaSpace{}%
\AgdaSymbol{(}\AgdaPostulate{new}\AgdaSpace{}%
\AgdaBound{σ}\AgdaSpace{}%
\AgdaOperator{\AgdaFunction{|𝐋}}\AgdaSymbol{)}\AgdaSpace{}%
\AgdaFunction{unspecified-in-𝐄}\AgdaSpace{}%
\AgdaBound{σ}\AgdaSymbol{)}\AgdaSpace{}%
\AgdaOperator{\AgdaPostulate{,}}\<%
\\
\>[4][@{}l@{\AgdaIndent{0}}]%
\>[6]\AgdaField{▻}\AgdaSpace{}%
\AgdaSymbol{(}\AgdaPostulate{wrong}\AgdaSpace{}%
\AgdaString{"out\ of\ memory"}\AgdaSymbol{)}\AgdaSpace{}%
\AgdaBound{σ}\<%
\end{code}
\clearpage
\begin{code}%
\>[0]\AgdaOperator{\AgdaFunction{ℰ⟦}}%
\>[165I]\AgdaOperator{\AgdaInductiveConstructor{⦅lambda␣⦅}}\AgdaSpace{}%
\AgdaBound{I*}\AgdaSpace{}%
\AgdaOperator{\AgdaInductiveConstructor{·}}\AgdaSpace{}%
\AgdaBound{I}\AgdaSpace{}%
\AgdaOperator{\AgdaInductiveConstructor{⦆}}\AgdaSpace{}%
\AgdaBound{Γ*}\AgdaSpace{}%
\AgdaOperator{\AgdaInductiveConstructor{␣}}\AgdaSpace{}%
\AgdaBound{E₀}\AgdaSpace{}%
\AgdaOperator{\AgdaInductiveConstructor{⦆}}\AgdaSpace{}%
\AgdaOperator{\AgdaFunction{⟧}}\AgdaSpace{}%
\AgdaSymbol{=}\AgdaSpace{}%
\AgdaSymbol{λ}\AgdaSpace{}%
\AgdaBound{ρ}\AgdaSpace{}%
\AgdaBound{κ}\AgdaSpace{}%
\AgdaSymbol{→}\AgdaSpace{}%
\AgdaField{◅}\AgdaSpace{}%
\AgdaSymbol{λ}\AgdaSpace{}%
\AgdaBound{σ}\AgdaSpace{}%
\AgdaSymbol{→}\<%
\\
\>[165I][@{}l@{\AgdaIndent{0}}]%
\>[4]\AgdaSymbol{(}\AgdaPostulate{new}\AgdaSpace{}%
\AgdaBound{σ}\AgdaSpace{}%
\AgdaOperator{\AgdaFunction{∈𝐋}}\AgdaSymbol{)}\AgdaSpace{}%
\AgdaOperator{\AgdaPostulate{⟶}}\<%
\\
\>[4][@{}l@{\AgdaIndent{0}}]%
\>[8]\AgdaField{▻}%
\>[187I]\AgdaSymbol{(}\AgdaFunction{send}%
\>[188I]\AgdaSymbol{(}\AgdaField{◅}%
\>[189I]\AgdaSymbol{(}%
\>[190I]\AgdaSymbol{(}\AgdaPostulate{new}\AgdaSpace{}%
\AgdaBound{σ}\AgdaSpace{}%
\AgdaOperator{\AgdaFunction{|𝐋}}\AgdaSymbol{)}\AgdaSpace{}%
\AgdaOperator{\AgdaInductiveConstructor{,}}\<%
\\
\>[.][@{}l@{}]\<[190I]%
\>[21]\AgdaSymbol{(λ}%
\>[194I]\AgdaBound{ϵ⋆}\AgdaSpace{}%
\AgdaBound{κ′}\AgdaSpace{}%
\AgdaSymbol{→}\<%
\\
\>[.][@{}l@{}]\<[194I]%
\>[24]\AgdaSymbol{(}\AgdaFunction{\#}%
\>[197I]\AgdaBound{ϵ⋆}\AgdaSpace{}%
\AgdaOperator{\AgdaFunction{>=⊥}}\AgdaSpace{}%
\AgdaFunction{\#′}\AgdaSpace{}%
\AgdaBound{I*}\AgdaSymbol{)}\AgdaSpace{}%
\AgdaOperator{\AgdaPostulate{⟶}}\<%
\\
\>[.][@{}l@{}]\<[197I]%
\>[27]\AgdaFunction{tievalsrest}\<%
\\
\>[27][@{}l@{\AgdaIndent{0}}]%
\>[30]\AgdaSymbol{(λ}\AgdaSpace{}%
\AgdaBound{α⋆}\AgdaSpace{}%
\AgdaSymbol{→}\AgdaSpace{}%
\AgdaSymbol{(λ}\AgdaSpace{}%
\AgdaBound{ρ′}\AgdaSpace{}%
\AgdaSymbol{→}\AgdaSpace{}%
\AgdaOperator{\AgdaFunction{𝒞*⟦}}\AgdaSpace{}%
\AgdaBound{Γ*}\AgdaSpace{}%
\AgdaOperator{\AgdaFunction{⟧}}\AgdaSpace{}%
\AgdaBound{ρ′}\AgdaSpace{}%
\AgdaSymbol{(}\AgdaOperator{\AgdaFunction{ℰ⟦}}\AgdaSpace{}%
\AgdaBound{E₀}\AgdaSpace{}%
\AgdaOperator{\AgdaFunction{⟧}}\AgdaSpace{}%
\AgdaBound{ρ′}\AgdaSpace{}%
\AgdaBound{κ′}\AgdaSymbol{))}\<%
\\
%
\>[30]\AgdaSymbol{(}\AgdaFunction{extends}\AgdaSpace{}%
\AgdaBound{ρ}\AgdaSpace{}%
\AgdaSymbol{(}\AgdaBound{I*}\AgdaSpace{}%
\AgdaOperator{\AgdaFunction{§′}}\AgdaSpace{}%
\AgdaSymbol{(}\AgdaSpace{}%
\AgdaNumber{1}\AgdaSpace{}%
\AgdaOperator{\AgdaInductiveConstructor{,}}\AgdaSpace{}%
\AgdaBound{I}\AgdaSpace{}%
\AgdaSymbol{))}\AgdaSpace{}%
\AgdaBound{α⋆}\AgdaSymbol{))}\<%
\\
%
\>[30]\AgdaBound{ϵ⋆}\<%
\\
%
\>[30]\AgdaSymbol{(}\AgdaPostulate{η}\AgdaSpace{}%
\AgdaSymbol{(}\AgdaFunction{\#′}\AgdaSpace{}%
\AgdaBound{I*}\AgdaSymbol{))}\AgdaSpace{}%
\AgdaOperator{\AgdaPostulate{,}}\<%
\\
\>[24][@{}l@{\AgdaIndent{0}}]%
\>[26]\AgdaPostulate{wrong}\AgdaSpace{}%
\AgdaString{"too\ few\ arguments"}\<%
\\
%
\>[21]\AgdaSymbol{)}\<%
\\
\>[.][@{}l@{}]\<[189I]%
\>[19]\AgdaSymbol{)}\AgdaSpace{}%
\AgdaOperator{\AgdaFunction{𝐅-in-𝐄}}\AgdaSymbol{)}\<%
\\
\>[.][@{}l@{}]\<[188I]%
\>[16]\AgdaBound{κ}\AgdaSymbol{)}\<%
\\
\>[.][@{}l@{}]\<[187I]%
\>[10]\AgdaSymbol{(}\AgdaFunction{update}\AgdaSpace{}%
\AgdaSymbol{(}\AgdaPostulate{new}\AgdaSpace{}%
\AgdaBound{σ}\AgdaSpace{}%
\AgdaOperator{\AgdaFunction{|𝐋}}\AgdaSymbol{)}\AgdaSpace{}%
\AgdaFunction{unspecified-in-𝐄}\AgdaSpace{}%
\AgdaBound{σ}\AgdaSymbol{)}\AgdaSpace{}%
\AgdaOperator{\AgdaPostulate{,}}\<%
\\
\>[4][@{}l@{\AgdaIndent{0}}]%
\>[6]\AgdaField{▻}\AgdaSpace{}%
\AgdaSymbol{(}\AgdaPostulate{wrong}\AgdaSpace{}%
\AgdaString{"out\ of\ memory"}\AgdaSymbol{)}\AgdaSpace{}%
\AgdaBound{σ}\<%
\\
%
\\[\AgdaEmptyExtraSkip]%
\>[0]\AgdaComment{--\ Non-compositional:}\<%
\\
\>[0]\AgdaComment{--\ ℰ⟦\ ⦅lambda\ I\ ␣\ Γ*\ ␣\ E₀\ ⦆\ ⟧\ =\ ℰ⟦\ ⦅lambda\ ⦅\ ·\ I\ ⦆\ Γ*\ ␣\ E₀\ ⦆\ ⟧}\<%
\\
%
\\[\AgdaEmptyExtraSkip]%
\>[0]\AgdaOperator{\AgdaFunction{ℰ⟦}}%
\>[239I]\AgdaOperator{\AgdaInductiveConstructor{⦅lambda}}\AgdaSpace{}%
\AgdaBound{I}\AgdaSpace{}%
\AgdaOperator{\AgdaInductiveConstructor{␣}}\AgdaSpace{}%
\AgdaBound{Γ*}\AgdaSpace{}%
\AgdaOperator{\AgdaInductiveConstructor{␣}}\AgdaSpace{}%
\AgdaBound{E₀}\AgdaSpace{}%
\AgdaOperator{\AgdaInductiveConstructor{⦆}}\AgdaSpace{}%
\AgdaOperator{\AgdaFunction{⟧}}\AgdaSpace{}%
\AgdaSymbol{=}\AgdaSpace{}%
\AgdaSymbol{λ}\AgdaSpace{}%
\AgdaBound{ρ}\AgdaSpace{}%
\AgdaBound{κ}\AgdaSpace{}%
\AgdaSymbol{→}\AgdaSpace{}%
\AgdaField{◅}\AgdaSpace{}%
\AgdaSymbol{λ}\AgdaSpace{}%
\AgdaBound{σ}\AgdaSpace{}%
\AgdaSymbol{→}\<%
\\
\>[239I][@{}l@{\AgdaIndent{0}}]%
\>[4]\AgdaSymbol{(}\AgdaPostulate{new}\AgdaSpace{}%
\AgdaBound{σ}\AgdaSpace{}%
\AgdaOperator{\AgdaFunction{∈𝐋}}\AgdaSymbol{)}\AgdaSpace{}%
\AgdaOperator{\AgdaPostulate{⟶}}\<%
\\
\>[4][@{}l@{\AgdaIndent{0}}]%
\>[8]\AgdaField{▻}%
\>[259I]\AgdaSymbol{(}\AgdaFunction{send}%
\>[260I]\AgdaSymbol{(}\AgdaField{◅}%
\>[261I]\AgdaSymbol{(}%
\>[262I]\AgdaSymbol{(}\AgdaPostulate{new}\AgdaSpace{}%
\AgdaBound{σ}\AgdaSpace{}%
\AgdaOperator{\AgdaFunction{|𝐋}}\AgdaSymbol{)}\AgdaSpace{}%
\AgdaOperator{\AgdaInductiveConstructor{,}}\<%
\\
\>[.][@{}l@{}]\<[262I]%
\>[21]\AgdaSymbol{(λ}%
\>[266I]\AgdaBound{ϵ⋆}\AgdaSpace{}%
\AgdaBound{κ′}\AgdaSpace{}%
\AgdaSymbol{→}\<%
\\
\>[.][@{}l@{}]\<[266I]%
\>[24]\AgdaFunction{tievalsrest}\<%
\\
\>[24][@{}l@{\AgdaIndent{0}}]%
\>[26]\AgdaSymbol{(λ}\AgdaSpace{}%
\AgdaBound{α⋆}\AgdaSpace{}%
\AgdaSymbol{→}%
\>[271I]\AgdaSymbol{(λ}\AgdaSpace{}%
\AgdaBound{ρ′}\AgdaSpace{}%
\AgdaSymbol{→}\AgdaSpace{}%
\AgdaOperator{\AgdaFunction{𝒞*⟦}}\AgdaSpace{}%
\AgdaBound{Γ*}\AgdaSpace{}%
\AgdaOperator{\AgdaFunction{⟧}}\AgdaSpace{}%
\AgdaBound{ρ′}\AgdaSpace{}%
\AgdaSymbol{(}\AgdaOperator{\AgdaFunction{ℰ⟦}}\AgdaSpace{}%
\AgdaBound{E₀}\AgdaSpace{}%
\AgdaOperator{\AgdaFunction{⟧}}\AgdaSpace{}%
\AgdaBound{ρ′}\AgdaSpace{}%
\AgdaBound{κ′}\AgdaSymbol{))}\<%
\\
\>[.][@{}l@{}]\<[271I]%
\>[34]\AgdaSymbol{(}\AgdaFunction{extends}\AgdaSpace{}%
\AgdaBound{ρ}\AgdaSpace{}%
\AgdaSymbol{(}\AgdaNumber{1}\AgdaSpace{}%
\AgdaOperator{\AgdaInductiveConstructor{,}}\AgdaSpace{}%
\AgdaBound{I}\AgdaSymbol{)}\AgdaSpace{}%
\AgdaBound{α⋆}\AgdaSymbol{))}\<%
\\
%
\>[26]\AgdaBound{ϵ⋆}\<%
\\
%
\>[26]\AgdaSymbol{(}\AgdaPostulate{η}\AgdaSpace{}%
\AgdaNumber{0}\AgdaSymbol{))}\<%
\\
\>[.][@{}l@{}]\<[261I]%
\>[19]\AgdaSymbol{)}\AgdaSpace{}%
\AgdaOperator{\AgdaFunction{𝐅-in-𝐄}}\AgdaSymbol{)}\<%
\\
\>[.][@{}l@{}]\<[260I]%
\>[16]\AgdaBound{κ}\AgdaSymbol{)}\<%
\\
\>[.][@{}l@{}]\<[259I]%
\>[10]\AgdaSymbol{(}\AgdaFunction{update}\AgdaSpace{}%
\AgdaSymbol{(}\AgdaPostulate{new}\AgdaSpace{}%
\AgdaBound{σ}\AgdaSpace{}%
\AgdaOperator{\AgdaFunction{|𝐋}}\AgdaSymbol{)}\AgdaSpace{}%
\AgdaFunction{unspecified-in-𝐄}\AgdaSpace{}%
\AgdaBound{σ}\AgdaSymbol{)}\AgdaSpace{}%
\AgdaOperator{\AgdaPostulate{,}}\<%
\\
\>[4][@{}l@{\AgdaIndent{0}}]%
\>[6]\AgdaField{▻}\AgdaSpace{}%
\AgdaSymbol{(}\AgdaPostulate{wrong}\AgdaSpace{}%
\AgdaString{"out\ of\ memory"}\AgdaSymbol{)}\AgdaSpace{}%
\AgdaBound{σ}\<%
\\
%
\\[\AgdaEmptyExtraSkip]%
%
\\[\AgdaEmptyExtraSkip]%
\>[0]\AgdaOperator{\AgdaFunction{ℰ⟦}}\AgdaSpace{}%
\AgdaOperator{\AgdaInductiveConstructor{⦅if}}\AgdaSpace{}%
\AgdaBound{E₀}\AgdaSpace{}%
\AgdaOperator{\AgdaInductiveConstructor{␣}}\AgdaSpace{}%
\AgdaBound{E₁}\AgdaSpace{}%
\AgdaOperator{\AgdaInductiveConstructor{␣}}\AgdaSpace{}%
\AgdaBound{E₂}\AgdaSpace{}%
\AgdaOperator{\AgdaInductiveConstructor{⦆}}\AgdaSpace{}%
\AgdaOperator{\AgdaFunction{⟧}}\AgdaSpace{}%
\AgdaSymbol{=}\AgdaSpace{}%
\AgdaSymbol{λ}\AgdaSpace{}%
\AgdaBound{ρ}\AgdaSpace{}%
\AgdaBound{κ}\AgdaSpace{}%
\AgdaSymbol{→}\<%
\\
\>[0][@{}l@{\AgdaIndent{0}}]%
\>[2]\AgdaOperator{\AgdaFunction{ℰ⟦}}\AgdaSpace{}%
\AgdaBound{E₀}\AgdaSpace{}%
\AgdaOperator{\AgdaFunction{⟧}}\AgdaSpace{}%
\AgdaBound{ρ}\AgdaSpace{}%
\AgdaSymbol{(}\AgdaFunction{single}\AgdaSpace{}%
\AgdaSymbol{(λ}\AgdaSpace{}%
\AgdaBound{ϵ}\AgdaSpace{}%
\AgdaSymbol{→}\<%
\\
\>[2][@{}l@{\AgdaIndent{0}}]%
\>[4]\AgdaFunction{truish}\AgdaSpace{}%
\AgdaBound{ϵ}\AgdaSpace{}%
\AgdaOperator{\AgdaPostulate{⟶}}\AgdaSpace{}%
\AgdaOperator{\AgdaFunction{ℰ⟦}}\AgdaSpace{}%
\AgdaBound{E₁}\AgdaSpace{}%
\AgdaOperator{\AgdaFunction{⟧}}\AgdaSpace{}%
\AgdaBound{ρ}\AgdaSpace{}%
\AgdaBound{κ}\AgdaSpace{}%
\AgdaOperator{\AgdaPostulate{,}}\<%
\\
\>[4][@{}l@{\AgdaIndent{0}}]%
\>[6]\AgdaOperator{\AgdaFunction{ℰ⟦}}\AgdaSpace{}%
\AgdaBound{E₂}\AgdaSpace{}%
\AgdaOperator{\AgdaFunction{⟧}}\AgdaSpace{}%
\AgdaBound{ρ}\AgdaSpace{}%
\AgdaBound{κ}\AgdaSymbol{))}\<%
\\
%
\\[\AgdaEmptyExtraSkip]%
\>[0]\AgdaOperator{\AgdaFunction{ℰ⟦}}\AgdaSpace{}%
\AgdaOperator{\AgdaInductiveConstructor{⦅if}}\AgdaSpace{}%
\AgdaBound{E₀}\AgdaSpace{}%
\AgdaOperator{\AgdaInductiveConstructor{␣}}\AgdaSpace{}%
\AgdaBound{E₁}\AgdaSpace{}%
\AgdaOperator{\AgdaInductiveConstructor{⦆}}\AgdaSpace{}%
\AgdaOperator{\AgdaFunction{⟧}}\AgdaSpace{}%
\AgdaSymbol{=}\AgdaSpace{}%
\AgdaSymbol{λ}\AgdaSpace{}%
\AgdaBound{ρ}\AgdaSpace{}%
\AgdaBound{κ}\AgdaSpace{}%
\AgdaSymbol{→}\<%
\\
\>[0][@{}l@{\AgdaIndent{0}}]%
\>[2]\AgdaOperator{\AgdaFunction{ℰ⟦}}\AgdaSpace{}%
\AgdaBound{E₀}\AgdaSpace{}%
\AgdaOperator{\AgdaFunction{⟧}}\AgdaSpace{}%
\AgdaBound{ρ}\AgdaSpace{}%
\AgdaSymbol{(}\AgdaFunction{single}\AgdaSpace{}%
\AgdaSymbol{(λ}\AgdaSpace{}%
\AgdaBound{ϵ}\AgdaSpace{}%
\AgdaSymbol{→}\<%
\\
\>[2][@{}l@{\AgdaIndent{0}}]%
\>[4]\AgdaFunction{truish}\AgdaSpace{}%
\AgdaBound{ϵ}\AgdaSpace{}%
\AgdaOperator{\AgdaPostulate{⟶}}\AgdaSpace{}%
\AgdaOperator{\AgdaFunction{ℰ⟦}}\AgdaSpace{}%
\AgdaBound{E₁}\AgdaSpace{}%
\AgdaOperator{\AgdaFunction{⟧}}\AgdaSpace{}%
\AgdaBound{ρ}\AgdaSpace{}%
\AgdaBound{κ}\AgdaSpace{}%
\AgdaOperator{\AgdaPostulate{,}}\<%
\\
\>[4][@{}l@{\AgdaIndent{0}}]%
\>[6]\AgdaFunction{send}\AgdaSpace{}%
\AgdaFunction{unspecified-in-𝐄}\AgdaSpace{}%
\AgdaBound{κ}\AgdaSymbol{))}\<%
\\
%
\\[\AgdaEmptyExtraSkip]%
\>[0]\AgdaComment{--\ Here\ and\ elsewhere,\ any\ expressed\ value\ other\ than\ `undefined`}\<%
\\
\>[0]\AgdaComment{--\ may\ be\ used\ in\ place\ of\ `unspecified`.}\<%
\end{code}
\clearpage
\begin{code}%
\>[0]\AgdaOperator{\AgdaFunction{ℰ⟦}}\AgdaSpace{}%
\AgdaOperator{\AgdaInductiveConstructor{⦅set!}}\AgdaSpace{}%
\AgdaBound{I}\AgdaSpace{}%
\AgdaOperator{\AgdaInductiveConstructor{␣}}\AgdaSpace{}%
\AgdaBound{E}\AgdaSpace{}%
\AgdaOperator{\AgdaInductiveConstructor{⦆}}\AgdaSpace{}%
\AgdaOperator{\AgdaFunction{⟧}}\AgdaSpace{}%
\AgdaSymbol{=}\AgdaSpace{}%
\AgdaSymbol{λ}\AgdaSpace{}%
\AgdaBound{ρ}\AgdaSpace{}%
\AgdaBound{κ}\AgdaSpace{}%
\AgdaSymbol{→}\<%
\\
\>[0][@{}l@{\AgdaIndent{0}}]%
\>[2]\AgdaOperator{\AgdaFunction{ℰ⟦}}\AgdaSpace{}%
\AgdaBound{E}\AgdaSpace{}%
\AgdaOperator{\AgdaFunction{⟧}}\AgdaSpace{}%
\AgdaBound{ρ}\AgdaSpace{}%
\AgdaSymbol{(}\AgdaFunction{single}\AgdaSpace{}%
\AgdaSymbol{(λ}\AgdaSpace{}%
\AgdaBound{ϵ}\AgdaSpace{}%
\AgdaSymbol{→}\<%
\\
\>[2][@{}l@{\AgdaIndent{0}}]%
\>[4]\AgdaFunction{assign}\AgdaSpace{}%
\AgdaSymbol{(}\AgdaFunction{lookup}\AgdaSpace{}%
\AgdaBound{ρ}\AgdaSpace{}%
\AgdaBound{I}\AgdaSymbol{)}\AgdaSpace{}%
\AgdaBound{ϵ}\AgdaSpace{}%
\AgdaSymbol{(}\AgdaFunction{send}\AgdaSpace{}%
\AgdaFunction{unspecified-in-𝐄}\AgdaSpace{}%
\AgdaBound{κ}\AgdaSymbol{)))}\<%
\\
%
\\[\AgdaEmptyExtraSkip]%
\>[0]\AgdaComment{--\ ℰ*⟦\AgdaUnderscore{}⟧\ \ :\ Exp\ *\ →\ 𝐔\ →\ 𝐊\ →\ 𝐂}\<%
\\
%
\\[\AgdaEmptyExtraSkip]%
\>[0]\AgdaOperator{\AgdaFunction{ℰ*⟦}}\AgdaSpace{}%
\AgdaNumber{0}\AgdaSpace{}%
\AgdaOperator{\AgdaInductiveConstructor{,}}\AgdaSpace{}%
\AgdaSymbol{\AgdaUnderscore{}}\AgdaSpace{}%
\AgdaOperator{\AgdaFunction{⟧}}%
\>[13]\AgdaSymbol{=}\AgdaSpace{}%
\AgdaSymbol{λ}\AgdaSpace{}%
\AgdaBound{ρ}\AgdaSpace{}%
\AgdaBound{κ}\AgdaSpace{}%
\AgdaSymbol{→}\AgdaSpace{}%
\AgdaField{▻}\AgdaSpace{}%
\AgdaBound{κ}\AgdaSpace{}%
\AgdaFunction{⟨⟩}\<%
\\
%
\\[\AgdaEmptyExtraSkip]%
\>[0]\AgdaOperator{\AgdaFunction{ℰ*⟦}}\AgdaSpace{}%
\AgdaNumber{1}\AgdaSpace{}%
\AgdaOperator{\AgdaInductiveConstructor{,}}\AgdaSpace{}%
\AgdaBound{E}\AgdaSpace{}%
\AgdaOperator{\AgdaFunction{⟧}}%
\>[13]\AgdaSymbol{=}\AgdaSpace{}%
\AgdaOperator{\AgdaFunction{ℰ⟦}}\AgdaSpace{}%
\AgdaBound{E}\AgdaSpace{}%
\AgdaOperator{\AgdaFunction{⟧}}\<%
\\
%
\\[\AgdaEmptyExtraSkip]%
\>[0]\AgdaOperator{\AgdaFunction{ℰ*⟦}}\AgdaSpace{}%
\AgdaInductiveConstructor{suc}\AgdaSpace{}%
\AgdaSymbol{(}\AgdaInductiveConstructor{suc}\AgdaSpace{}%
\AgdaBound{n}\AgdaSymbol{)}\AgdaSpace{}%
\AgdaOperator{\AgdaInductiveConstructor{,}}\AgdaSpace{}%
\AgdaBound{E}\AgdaSpace{}%
\AgdaOperator{\AgdaInductiveConstructor{,}}\AgdaSpace{}%
\AgdaBound{Es}\AgdaSpace{}%
\AgdaOperator{\AgdaFunction{⟧}}\AgdaSpace{}%
\AgdaSymbol{=}\AgdaSpace{}%
\AgdaSymbol{λ}\AgdaSpace{}%
\AgdaBound{ρ}\AgdaSpace{}%
\AgdaBound{κ}\AgdaSpace{}%
\AgdaSymbol{→}\<%
\\
\>[0][@{}l@{\AgdaIndent{0}}]%
\>[2]\AgdaOperator{\AgdaFunction{ℰ⟦}}\AgdaSpace{}%
\AgdaBound{E}\AgdaSpace{}%
\AgdaOperator{\AgdaFunction{⟧}}\AgdaSpace{}%
\AgdaBound{ρ}\AgdaSpace{}%
\AgdaSymbol{(}\AgdaFunction{single}\AgdaSpace{}%
\AgdaSymbol{(λ}\AgdaSpace{}%
\AgdaBound{ϵ₀}\AgdaSpace{}%
\AgdaSymbol{→}\<%
\\
\>[2][@{}l@{\AgdaIndent{0}}]%
\>[4]\AgdaOperator{\AgdaFunction{ℰ*⟦}}\AgdaSpace{}%
\AgdaInductiveConstructor{suc}\AgdaSpace{}%
\AgdaBound{n}\AgdaSpace{}%
\AgdaOperator{\AgdaInductiveConstructor{,}}\AgdaSpace{}%
\AgdaBound{Es}\AgdaSpace{}%
\AgdaOperator{\AgdaFunction{⟧}}\AgdaSpace{}%
\AgdaBound{ρ}\AgdaSpace{}%
\AgdaSymbol{(}\AgdaField{◅}\AgdaSpace{}%
\AgdaSymbol{λ}\AgdaSpace{}%
\AgdaBound{ϵ⋆}\AgdaSpace{}%
\AgdaSymbol{→}\<%
\\
\>[4][@{}l@{\AgdaIndent{0}}]%
\>[6]\AgdaField{▻}\AgdaSpace{}%
\AgdaBound{κ}\AgdaSpace{}%
\AgdaSymbol{(}\AgdaOperator{\AgdaFunction{⟨}}\AgdaSpace{}%
\AgdaBound{ϵ₀}\AgdaSpace{}%
\AgdaOperator{\AgdaFunction{⟩}}\AgdaSpace{}%
\AgdaOperator{\AgdaFunction{§}}\AgdaSpace{}%
\AgdaBound{ϵ⋆}\AgdaSymbol{))))}\<%
\\
%
\\[\AgdaEmptyExtraSkip]%
\>[0]\AgdaComment{--\ 𝒞*⟦\AgdaUnderscore{}⟧\ \ :\ Com\ *\ →\ 𝐔\ →\ 𝐂\ →\ 𝐂}\<%
\\
%
\\[\AgdaEmptyExtraSkip]%
\>[0]\AgdaOperator{\AgdaFunction{𝒞*⟦}}\AgdaSpace{}%
\AgdaNumber{0}\AgdaSpace{}%
\AgdaOperator{\AgdaInductiveConstructor{,}}\AgdaSpace{}%
\AgdaSymbol{\AgdaUnderscore{}}\AgdaSpace{}%
\AgdaOperator{\AgdaFunction{⟧}}\AgdaSpace{}%
\AgdaSymbol{=}%
\>[15]\AgdaSymbol{λ}\AgdaSpace{}%
\AgdaBound{ρ}\AgdaSpace{}%
\AgdaBound{θ}\AgdaSpace{}%
\AgdaSymbol{→}\AgdaSpace{}%
\AgdaBound{θ}\<%
\\
%
\\[\AgdaEmptyExtraSkip]%
\>[0]\AgdaOperator{\AgdaFunction{𝒞*⟦}}\AgdaSpace{}%
\AgdaNumber{1}\AgdaSpace{}%
\AgdaOperator{\AgdaInductiveConstructor{,}}\AgdaSpace{}%
\AgdaBound{Γ}\AgdaSpace{}%
\AgdaOperator{\AgdaFunction{⟧}}\AgdaSpace{}%
\AgdaSymbol{=}\AgdaSpace{}%
\AgdaSymbol{λ}\AgdaSpace{}%
\AgdaBound{ρ}\AgdaSpace{}%
\AgdaBound{θ}\AgdaSpace{}%
\AgdaSymbol{→}\AgdaSpace{}%
\AgdaOperator{\AgdaFunction{ℰ⟦}}\AgdaSpace{}%
\AgdaBound{Γ}\AgdaSpace{}%
\AgdaOperator{\AgdaFunction{⟧}}\AgdaSpace{}%
\AgdaBound{ρ}\AgdaSpace{}%
\AgdaSymbol{(}\AgdaField{◅}\AgdaSpace{}%
\AgdaSymbol{λ}\AgdaSpace{}%
\AgdaBound{ϵ⋆}\AgdaSpace{}%
\AgdaSymbol{→}\AgdaSpace{}%
\AgdaBound{θ}\AgdaSymbol{)}\<%
\\
%
\\[\AgdaEmptyExtraSkip]%
\>[0]\AgdaOperator{\AgdaFunction{𝒞*⟦}}\AgdaSpace{}%
\AgdaInductiveConstructor{suc}\AgdaSpace{}%
\AgdaSymbol{(}\AgdaInductiveConstructor{suc}\AgdaSpace{}%
\AgdaBound{n}\AgdaSymbol{)}\AgdaSpace{}%
\AgdaOperator{\AgdaInductiveConstructor{,}}\AgdaSpace{}%
\AgdaBound{Γ}\AgdaSpace{}%
\AgdaOperator{\AgdaInductiveConstructor{,}}\AgdaSpace{}%
\AgdaBound{Γs}\AgdaSpace{}%
\AgdaOperator{\AgdaFunction{⟧}}\AgdaSpace{}%
\AgdaSymbol{=}\AgdaSpace{}%
\AgdaSymbol{λ}\AgdaSpace{}%
\AgdaBound{ρ}\AgdaSpace{}%
\AgdaBound{θ}\AgdaSpace{}%
\AgdaSymbol{→}\<%
\\
\>[0][@{}l@{\AgdaIndent{0}}]%
\>[2]\AgdaOperator{\AgdaFunction{ℰ⟦}}\AgdaSpace{}%
\AgdaBound{Γ}\AgdaSpace{}%
\AgdaOperator{\AgdaFunction{⟧}}\AgdaSpace{}%
\AgdaBound{ρ}\AgdaSpace{}%
\AgdaSymbol{(}\AgdaField{◅}\AgdaSpace{}%
\AgdaSymbol{λ}\AgdaSpace{}%
\AgdaBound{ϵ⋆}\AgdaSpace{}%
\AgdaSymbol{→}\<%
\\
\>[2][@{}l@{\AgdaIndent{0}}]%
\>[4]\AgdaOperator{\AgdaFunction{𝒞*⟦}}\AgdaSpace{}%
\AgdaInductiveConstructor{suc}\AgdaSpace{}%
\AgdaBound{n}\AgdaSpace{}%
\AgdaOperator{\AgdaInductiveConstructor{,}}\AgdaSpace{}%
\AgdaBound{Γs}\AgdaSpace{}%
\AgdaOperator{\AgdaFunction{⟧}}\AgdaSpace{}%
\AgdaBound{ρ}\AgdaSpace{}%
\AgdaBound{θ}\AgdaSymbol{)}\<%
\\
\>[0]\<%
\end{code} 

\bibliographystyle{ACM-Reference-Format}
%\bibliography{Scheme}

\end{document}
